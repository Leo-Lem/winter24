\documentclass{article}

\usepackage[]{../.template/xrcise}

\subject{Natural Language Processsing}
\semester{Winter 2024}
\author{Leopold Lemmermann}

\begin{document}\createtitle

\sheet{Lecture questions}

% Q: The web is an application area for NLP, e.g.: [][]
% A: Internet of Services, Community mining, Information retrieval,
% ...
% Q: Web is a resource to improve the quality of NLP, e.g.: [][][]
% A: Web as Corpus, Analyzing web-based knowledge repositories,

% Q: Identify all stems and affixes (prefix, suffix, infix, circumfix) in following words: index, incorrect, interesting
% A: stem:index, prefix:in stem:correct, stem:interest suffix:ing

% Q: In contrast to lemmatization, stemming does not necessarily return a valid word form. Why is stemming still useful?
% A: Faster, easier, applications in IR.

% ▪ Know what a “corpus” is, how to build a corpus, and how to use it for linguistics analysis
% ▪ Understand the notion of NLP datasets
% ▪ Corpus and dataset ”annotation” process and its impact for machine learning applications
% ▪ Main strategies for large-scale dataset collection using crowdsourcing


\setcounter{section}{2022}
\sheet{1st exam}
\begin{exercise}{Tokenization}
  Give at least 3 examples of ambiguity in tokenization.

  \begin{solution}
    % TODO
  \end{solution}
\end{exercise}

\begin{exercise}{Parse trees}
  Given the sentence "John shot the man with the gun" and a grammar: Draw all existing parse trees. Which types of syntactic ambiguity occur here and where?

  \begin{solution}
    % TODO
  \end{solution}
\end{exercise}

\begin{exercise}{Stemming and lemmatization}
  Difference between stemming and lemmatization.

  \begin{solution}
    % TODO
  \end{solution}
\end{exercise}

\begin{exercise}{Web as corpus}
  List 4 issues you have to deal with when using the web as a corpus for NLP purposes. Explain each.

  \begin{solution}
    % TODO
  \end{solution}
\end{exercise}

\begin{exercise}{Cross-language QA}
  How does cross-language QA work? Issues?

  \begin{solution}
    % TODO
  \end{solution}
\end{exercise}

\begin{exercise}{Machine learning pipeline}
  What are the main concepts in a machine learning pipeline?

  \begin{solution}
    % TODO
  \end{solution}
\end{exercise}

\begin{exercise}{IR evaluation}
  What are the problems with precision, recall, and f-measure in IR? How to improve and why is as an example cohen kappa better?

  \begin{solution}
    % TODO
  \end{solution}
\end{exercise}

\begin{exercise}{IR evaluation}
  Calculate P@5, P@10, MAP for the given IR ranked result list (one for IR system A and one for B. Both go up to ten entries and either have a checkmark or an x in each rank).

  \begin{solution}
    % TODO
  \end{solution}
\end{exercise}

\begin{exercise}{Naive Bayes and SVM}
  Which of the following 4 are correct
  \begin{enumerate}
    \item Features for Naive Bayes classifiers do not need to be uncorrelated
    \item Naive Bayes is slow to train
    \item SVM itself is a multi-label classifier
    \item SVM handles sparse data well
  \end{enumerate}

  \begin{solution}
    % TODO
  \end{solution}
\end{exercise}

\begin{exercise}{Corpus construction}
  How to construct a corpus for linguistic purposes from web data and how to make it usable?

  \begin{solution}
    % TODO
  \end{solution}
\end{exercise}

\begin{exercise}{Pragmatics and discourse analysis}
  Why are pragmatics and discourse analysis difficult?

  \begin{solution}
    % TODO
  \end{solution}
\end{exercise}

\begin{exercise}{Hate speech}
  Define hate speech and how to construct a hate speech classifier

  \begin{solution}
    % TODO
  \end{solution}
\end{exercise}

\begin{exercise}{Counter hate speech}
  What is counter hate speech and how to construct such a system

  \begin{solution}
    % TODO
  \end{solution}
\end{exercise}

\begin{exercise}{Crowdsourced annotation}
  Crowdsourced annotation workflow

  \begin{solution}
    % TODO
  \end{solution}
\end{exercise}

\begin{exercise}{Web corpus}
  Why are duplicates on the web and how to detect

  \begin{solution}
    % TODO
  \end{solution}
\end{exercise}

\begin{exercise}{QA vs IR}
  Differences in QA vs IR

  \begin{solution}
    % TODO
  \end{solution}
\end{exercise}

\begin{exercise}{Cross-language QA}
  Explain applications of multi-language QA and difficulties?

  \begin{solution}
    % TODO
  \end{solution}
\end{exercise}

\begin{exercise}{Language detection}
  Explain language detection

  \begin{solution}
    % TODO
  \end{solution}
\end{exercise}

\begin{exercise}{Factual QA vs opinionated QA}
  Differences between factual QA and opinionated QA, and which is easier to evaluate?

  \begin{solution}
    % TODO
  \end{solution}
\end{exercise}

\begin{exercise}{RAG and LangChain}
  What's RAG and how does LangChain help?

  \begin{solution}
    % TODO
  \end{solution}
\end{exercise}

\begin{exercise}{Sentiment analysis}
  Discuss: Sentiment analysis using supervised vs. lexicon based approaches.

  \begin{solution}
    % TODO
  \end{solution}
\end{exercise}

\begin{exercise}{Edit distance}
  Calculate the Minimum Edit Distance for "TAC" vs. "CAT" (given is also a table just like on the slides, just without the numbers filled in). List all possible word outcomes (in this case 4).

  \begin{solution}
    % TODO
  \end{solution}
\end{exercise}


\setcounter{section}{2021}
\sheet{1st exam}

\begin{exercise}{Corpus construction}
  How would you construct a corpus for linguistic purposes and how to make it usable?

  \begin{solution}
    % TODO
  \end{solution}
\end{exercise}

\begin{exercise}{Pragmatics and discourse analysis}
  Why are pragmatics and discourse analysis difficult?

  \begin{solution}
    % TODO
  \end{solution}
\end{exercise}

\begin{exercise}{Sequence tagging and CRF}
  Define sequence tagging and what is CRF

  \begin{solution}
    % TODO
  \end{solution}
\end{exercise}

\begin{exercise}{Minimum edit distance}
  Calculate the Minimum Edit Distance for "TAC" vs. "CAT" (given is also a table just like on the slides, just without the numbers filled in). List all possible word outcomes (in this case 4).

  \begin{solution}
    % TODO
  \end{solution}
\end{exercise}

\begin{exercise}{IR evaluation}
  Calculate P@5 and MAP for the given IR ranked result list (one for IR system A and one for B. Both go up to ten entries and either have a checkmark or an x in each rank).

  \begin{solution}
    % TODO
  \end{solution}
\end{exercise}

\begin{exercise}{Sequence tagging and CRF}
  Define sequence tagging and what is CRF

  \begin{solution}
    % TODO
  \end{solution}
\end{exercise}

\begin{exercise}{Hate speech}
  Define hate speech and how to construct a hate speech classifier

  \begin{solution}
    % TODO
  \end{solution}
\end{exercise}

\begin{exercise}{Counter hate speech}
  What is counter hate speech and how to construct such a system

  \begin{solution}
    % TODO
  \end{solution}
\end{exercise}

\begin{exercise}{Sentiment analysis}
  Discuss sentiment detection using lexicon based vs. supervised learning approaches.

  \begin{solution}
    % TODO
  \end{solution}
\end{exercise}

\begin{exercise}{Sentiment lexicon extension}
  How could you analyse sentiment with lexicon extension using distributed thesauruses.

  \begin{solution}
    % TODO
  \end{solution}
\end{exercise}

\begin{exercise}{Sentiment analysis}
  Why is lexicon based sentiment analysis not enough?

  \begin{solution}
    % TODO
  \end{solution}
\end{exercise}

\begin{exercise}{Naive Bayes and SVM}
  Which of the following 4 are correct
  \begin{enumerate}
    \item Features for Naive Bayes classifiers do not need to be uncorrelated
    \item Naive Bayes is slow to train
    \item SVM itself is a multi-label classifier
    \item SVM handles sparse data well
  \end{enumerate}

  \begin{solution}
    % TODO
  \end{solution}
\end{exercise}

\begin{exercise}{Language detection}
  Explain approaches for language detection

  \begin{solution}
    % TODO
  \end{solution}
\end{exercise}

\begin{exercise}{Crowdsourced annotation}
  Explain the crowdsourced annotation workflow

  \begin{solution}
    % TODO
  \end{solution}
\end{exercise}

\begin{exercise}{Tokenization ambiguities}
  What are the ambiguities for tokenization?

  \begin{solution}
    % TODO
  \end{solution}
\end{exercise}

\begin{exercise}{QA vs IR}
  Differences between QA and IR

  \begin{solution}
    % TODO
  \end{solution}
\end{exercise}

\begin{exercise}{Cross-language QA}
  What are the challenges in cross-language QA?

  \begin{solution}
    % TODO
  \end{solution}
\end{exercise}

\begin{exercise}{Stemming vs lemmatization}
  What is the difference between stemming and lemmatization?

  \begin{solution}
    % TODO
  \end{solution}
\end{exercise}

\begin{exercise}{Web corpus}
  Why are duplicates on the web and how to detect

  \begin{solution}
    % TODO
  \end{solution}
\end{exercise}

\begin{exercise}{Web corpus}
  What are the 4 challenges when using the web as a corpus

  \begin{solution}
    % TODO
  \end{solution}
\end{exercise}

\begin{exercise}{TREC}
  What are the TREC question types

  \begin{solution}
    % TODO
  \end{solution}
\end{exercise}

\begin{exercise}{QALD}
  What is QALD and what are the challenges

  \begin{solution}
    % TODO
  \end{solution}
\end{exercise}


\setcounter{section}{2020}
\sheet{1st exam}

\begin{exercise}{Comprehensiveness}
  Explain comprehensiveness.

  \begin{solution}
    % TODO
  \end{solution}
\end{exercise}

\begin{exercise}{Complete the picture}
  Fill in the white boxes in the image.

  % TODO: add image

  \begin{solution}
    % TODO
  \end{solution}
\end{exercise}

\begin{exercise}{NLTK}
  Explain two parameters in NLTK that start with a c.

  \begin{solution}
    % TODO
  \end{solution}
\end{exercise}

\begin{exercise}{SVM}
  Explain Support Vector Machines when you have more than two classes.

  \begin{solution}
    % TODO
  \end{solution}
\end{exercise}

\begin{exercise}{Feature differences}
  Explain the differences in features between classical ML and deep network based ML.

  \begin{solution}
    % TODO
  \end{solution}
\end{exercise}

\begin{exercise}{Cohen's Kappa}
  Calculate Cohen's Kappa for a given table.

  \begin{solution}
    % TODO
  \end{solution}
\end{exercise}

\begin{exercise}{Evaluation metrics}
  Calculate Precision, Recall, and F1-Score.

  \begin{solution}
    % TODO
  \end{solution}
\end{exercise}

\begin{exercise}{IR and classification}
  Where is classification used in Information Retrieval?

  \begin{solution}
    % TODO
  \end{solution}
\end{exercise}

\begin{exercise}{Translate datasets}
  List 3 reasons for and against translating an existing annotated dataset from German to another language.

  \begin{solution}
    % TODO
  \end{solution}
\end{exercise}

\begin{exercise}{Web corpus}
  List 3 reasons for and against using the web as a corpus.

  \begin{solution}
    % TODO
  \end{solution}
\end{exercise}

\begin{exercise}{Pipeline}
  Given the pipeline from the first slide, assign three words to the individual steps.

  \begin{solution}
    % TODO
  \end{solution}
\end{exercise}

\begin{exercise}{MRR and accuracy}
  Calculate MRR and accuracy for a Question/Answering system.

  \begin{solution}
    % TODO
  \end{solution}
\end{exercise}

\begin{exercise}{Language detection}
  Explain language detection.

  \begin{solution}
    % TODO
  \end{solution}
\end{exercise}

\begin{exercise}{Plagiarism detection}
  Explain how to find plagiarism.

  \begin{solution}
    % TODO
  \end{solution}
\end{exercise}

\begin{exercise}{Crowdsourcing}
  List the pros and cons of crowdsourcing.

  \begin{solution}
    % TODO
  \end{solution}
\end{exercise}

\begin{exercise}{Recall improvement}
  Explain how to improve recall when using a search engine without changing the algorithm.

  \begin{solution}
    % TODO
  \end{solution}
\end{exercise}

\begin{exercise}{Spell checker}
  Explain how a spell checker works with non-words.

  \begin{solution}
    % TODO
  \end{solution}
\end{exercise}

\begin{exercise}{IR cycle}
  Explain the Information Retrieval cycle.

  \begin{solution}
    % TODO
  \end{solution}
\end{exercise}

\begin{exercise}{Vector space}
  Explain Vector Space in Information Retrieval.

  \begin{solution}
    % TODO
  \end{solution}
\end{exercise}

\begin{exercise}{Sentiment analysis}
  Explain the difference between factual sentiment analysis and opinionated analysis.

  \begin{solution}
    % TODO
  \end{solution}
\end{exercise}

\begin{exercise}{Sentiment analysis}
  Explain the areas where sentiment analysis is used.

  \begin{solution}
    % TODO
  \end{solution}
\end{exercise}

\begin{exercise}{Teacher/Student}
  Explain Teacher/Student and why it is better than just audio.

  \begin{solution}
    % TODO
  \end{solution}
\end{exercise}

\begin{exercise}{Representation and fusion}
  Explain representation and fusion and the differences.

  \begin{solution}
    % TODO
  \end{solution}
\end{exercise}

\end{document}