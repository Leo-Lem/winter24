\documentclass{article}

\usepackage[]{../.template/xrcise}

\subject{Natural Language Processsing}
\semester{Winter 2024}
\author{Leopold Lemmermann}

\begin{document}\createtitle

\sheet[2024]{1st exam}
% 1 Question: Foundations (30 Points)
% This section has various tasks to assess your understanding of fundamental concepts, principles and
% terminology related to requirements engineering and software architecture.
% 1.1
% (3 Points) Name one activity in requirements engineering and provide a brief explanation.
% 1.2
% You have learned about three types of qualtity scenarios including use-case, growth and exploratory
% scenarios. Explain each of them by an example: Use-case scenarios, Growth scenarios, Exploratory scenarios.

% 1.3 Give three reasons why software architecture is important. Explain each in one sentence.

% 1.4 Define the term architectural tactic in the context of software architecture. In addition
% provide an example of an achtitecture tactic and explain how it can be used to address a specific
% quality attribute in a software system.

% 1.5
% (3 Points) Which of the following pairs of qualities are typically in conflict with each other (i.e. are
% trade-offs to each other)? Explain each decision in one sentence.
% Modularity - Time behavior
% ⃝ typically in conflict
% ⃝ typically not in conflict
% Explanation:
% Usability - Security
% ⃝ typically in conflict
% ⃝ typically not in conflict
% Explanation:
% 3
% 1.6
% (3 Points) What is the purpose of conformance/compliance checking? In addition, briefly explain
% strucural and behavioral conformance checking.
% 1.7
% (2 Points) Explain the term "lift & shift" in the context of cloud development.
% 1.8
% (4 Points) Decide for each of the following statements wether it is true or false. Justify your decision
% in one sentence.
% If your system consists of a set of microservices, there is no need for a central architecture document
% since each service is fre to choose its technologies.
% ⃝ True
% ⃝ False
% Explanation:
% Celebrating release parties is a sign of great architecture.
% ⃝ True
% ⃝ False



% 2 Question: Reqirements Engineering (15 Points)
% The GreenHome Energy Management System is an innovative solution designed to empower (1) home-
% ownerstoharnessandoptimzegreenenergysourceswithintheirresidences, (2)energyserviceproviders
% to improve service customization, e.g. flexible pricing based on user behavior and offer energy-saving
% solutions, and (3) renewable energy solution providers to invest in research and development of renew-
% able energy technologies. To promote sustainability, reduce carbon footprint, and maximize energy
% efficiency, this comprehensive software system integrates seamlessly with (1) smrt home devices such
% as smart thermostats, lighting systems, and appliances and (2) renewable energy sources, including
% solar panels, and energy storage systems.
% 2.1
% Solve the following tasks regarding the stakeholder analysis.
% i. (4 Points) Write the name of each axis after each = sign. In addition, classify three stakeholders
% derived from the text according to the following matrix.
% Figure 1: Stakeholder matrix
% ii. (2 Points) Classify two more stakeholders that are not explicitly mentioned in the text. Add
% them to the matrix in Figure 1. Show which ones are the two additional stakeholders by drawing
% a circle around the two names.
% iii. (3 Points) Briefly explain the reason for your classification for the two additional stakeholders.
% 5
% 2.2
% Consider the following scenario, which is related to the GreenHome Energy Management System
% requirements:
% • "The Cost analysis in the GreenHome Energy Management System addresses the need for home-
% owners to observe their energy-rekated expenses, which involve providing users witha straight-
% forward breakdown of their energy costs, including expenses associated with renewable energy
% production."
% • "The system provides advanced analytics to forecast future energy costs based on usage patterns
% and external factors."
% • "It would be cool to have a gamified elemet eithin the Cost calculation. The idea is to award
% users GreenPoints for implementing energy saving measures, achieving specific sustainability
% milestones."
% Regarding the Kano model, solve the following tasks.
% i. (2 Points) Derive one basic requirement from the given scenario.
% ii. (2 Points) Derive one performance requirement from the given scenario.
% iii. (2 Points) Derive one excitement requirement from the given scenario.



% 3 Question: Architecture Modeling (23 Points)
% Figure 2 is part of an architectural description of an image hosting application. The application has
% two keey functionalities: (1) the ability to upload (write) an image to the serer, and (2) the ability to
% query for an image. Make sure to read and understand Figure 2.
% Figure 2: Architectural description of an image hosting application
% The team responsible for the image hosting application project decided to use 4+1, which is a
% template to structure the documentation of a software architecture. Answer the following questions
% and help the team to create the documentation of the architecture.
% 3.1
% Qestions regarding Figure 2:
% i. (1 Point) What Architectural Description Language (ADL) is used by the team?
% ii. (2 Points) What view does the architectural description show accordingto the 4+1 template?
% Briefly explain why.
% 7
% 3.2
% (10 Points) Process view:
% Consider the following scenario. "A user calls getImages through the IDownloadImage interface of
% WebUI component to download images. WebUI calls a method with the same name getImages from
% the IRequestHandler interface. Furthermore, the RequestHandler component calls the zip method from
% the ReEncoder component to get a zip package of all images. The ReEncoder component calls the
% imagesList method of the ImageAccess component. Finally, the ImageAccess component returns a
% result if it is already cashed. Otherwise, it returns NULL."
% Specify the intercomponent behavior through a UML sequence diagram for the given scenario.
% 8
% 3.3
% (10 Points)Deployment view:
% Use the UML notation to specify that (1) the WebUI component is deployed on aFrontEnd server,
% (2) the RequestHandler, ReEncoder and IMage Acces components are deployed to the Application-
% Server, (3) the Database and FileStorage components are deployed to the DatabaseServer and File-
% Server respectively. All the physical servers ar interconnected via LAN.



% 4 Question: Architecture Evaluation (22 Points)
% Jitsi Meet is a video-conferencing solution. The main service that Jitsi offers is creating video confer-
% ences and allowing participants across the Internet to join the conference. Three components are the
% core of the Jitsi system:
% • JitsiWeb provides a web-based user interface to access the Jitsi functionality.
% • JiCofo provides functionality for managing conferences.
% • VideoBridge is the critical component that is responsible for routing the video stream between
% multiple participants of a conference.
% Figure 3 shows how users attend different conferences using the Jitsi system. User1 and User2
% (depicted with a black box around user icon) joined Conference 1 (depicted with a black box around
% the text), and User 3 and User 4 joined Conference 2. The JiCofo component communicates with
% the VideoBridge component through XMPP, which is a communication protocol. Each server can
% have redundancies to support more conferences simultaneously. It is important to note that when a
% conference fails, e.g. Conference 1, a new conference has to be created. There is no redundancy and
% failover for a particular conference.
% Figure 3: Jitsi video conference system
% 10
% 4.1
% Consider the Architecture Tradeoff Analysis Method (ATAM).
% i. Provide a scenario for the Performance quality attribute such that the system maintains its
% performance when the number of users accessing the system increases.
% a) (1 Point) Source
% b) (1 Point) Stimulus
% c) (1 Point) Artifact
% d) (1 Point) Environment
% e) (1 Point) Response
% f) (1 Point) Response measure
% ii. Provide a scenario for the Modifiablity quality attribute.
% a) (1 Point) Source
% b) (1 Point) Stimulus
% c) (1 Point) Artifact
% d) (1 Point) Environment
% e) (1 Point) Response
% f) (1 Point) Response measure
% 11
% 4.2
% ATradeoff point isasystempropertythataffectsmorethanonequalityattribute. Forexample, adding
% more bridge servers affects more than one quality attribute in the system. Therefore, the number of
% bridge servers is a tradeoff point.
% i. (1 Point) Name at least two quality attributes of the system that the number of bridge servers
% affects.
% ii. (2 Points) Explain in one statement how each quality is affected by a change in the number of
% bridge servers.
% 4.3
% The current deployment strategy of Jitsi Meet system is visible in Figure 3. The engineering team
% deploys JitsiWeb and JiCofo service instances into one physical server. The team argues efficient
% resource usage is the main reason for their decision.
% i. (1 Point) Name one drawback for the current deployment strategy.
% ii. (2 Points) Suggest a new deployment strategy and explain the reason behind your decision.
% iii. (1 Point) Name one drawback of the new deployment strategy.
% 12
% 4.4
% (3 Points) Performance engineers of the JitsiMeet system claim that their system is efficient in terms
% of CPU usage. To prove their calim, they experimented on a server machine, which runs all the services
% needed by Jitsi Meet including the JitsiWeb, JiCofo and VideoBridge.
% The engineering team’s big challenge was to send enough traffic to put the system under load, ehich
% requires a considerable number of machines. What would you suggest to the engineering team as an
% alternative approach for quantitative evaluation of the Jitsi Meet? Briefly explain what your approach
% is.

\sheet[2024]{2nd exam}

% 1 Foundation (30 P.)
% 1. What does vertical traceability of requirements mean?
% 2. Name the layers of cloud computing and explain vendor lock-in.
% 3. What is a Domain Model? Explain using an example.
% 4. Consider architectural styles and patterns.
% a) Name two similarities between architectural styles and patterns.
% b) Name two differences between architectural styles and patterns.
% c) How do styles and patterns relate to each other?
% 5. Give two advantages and two disadvantages of semi-/formal vs. informal architec-
% ture documentation. (4 P.)
% 1
% 2 Requirements Engineering (12 P.)
% Here was a text given describing the rough proposed functionality of a music streaming
% service (Melody).
% 1. Classify four stakeholders according to the following matrix. Write the name of
% each axis after the = sign. Now there was the empty stakeholder matrix given, the
% axes named “x =” and “y =”. (5 P.)
% 2. Briefly explain your classification of two of the stakeholders. (1 P.)
% 3. Consider the Kano model.
% a) Describe one basic feature of the system and say, why it is one.
% b) Describe one performance feature of the system and say, why it is one.
% c) Describe one excitement feature of the system and say, why it is one.
% 3 Architecture Modeling (ca. 24 P.)
% Here was a text given describing the rough functionality of a new IDE (Power IDE ).
% 1. Which architectural style was used for the system? Give a reason why.
% 2. Which quality attribute does this style match and why?
% 3. Single choice question (yes/no): Is there usually a trade-off between the quality
% attributes modifiability and performance? Explain your answer briefly.
% 4. Single choice question (yes/no): Is there usually a trade-off between reusability
% and interoperability? Explain your answer briefly.
% 5. Use UML to model the presented architecture. (10 P.)
% 6. (Here a text was given describing a certain starting procedure.) Give a UML
% sequence diagram for the presented process. (8 P.)
% 2
% 4 Evaluation (ca. 24 P.)
% Here a text and a UML deployment diagram were given describing the rough functional-
% ity of an online order service (Food2Go) consisting of the three components OrderService,
% RestaurantService and AccountingService where OrderService had a connection to
% both of the other services but these had no connection to each other.
% 1. Consider the Architecture Tradeoff Analysis Method (ATAM).
% a) Provide a scenario for the Availability quality attribute: Source, Stimulus,
% Artifact, Environment, Response, Response measure (one text field each).
% b) Provide a scenario for the Modifiability quality attribute such that deploy-
% ment has finished after a maximum of one day: Source, Stimulus, Artifact,
% Environment, Response, Response measure (one text field each).
% 2. A design alternative (design 2) is to add a cache to OrderService.
% a) How does this alternative affect availability?
% b) Describe a scenario in which a client could experience data inconsistency.
% 3. Another design alternative (design 3) is to add third party messaging service.
% a) Would that solve the problem of data inconsistency?
% b) Does design 3 improve availability?
% c) Name a disadvantage of design 3 compared to design 2.

\end{document}