\documentclass{article}

\usepackage[]{../.template/xrcise}

\subject{Natural Language Processsing}
\semester{Winter 2024}
\author{Leopold Lemmermann}

\begin{document}\createtitle

\sheet{2023}{1st exam}
% The exam was 2 hours long and in exactly the same format with basically the same questions as the memory logs from the last years, again with about 23 questions. The following are some (but not all of them):

% Give at least 3 examples of ambiguity in tokenization.
% Given the sentence "John shot the man with the gun" and a grammar: Draw all existing parse trees. Which types of syntactic ambiguity occur here and where?
% Difference between stemming and lemmatization.
% List 4 issues you have to deal with when using the web as a corpus for NLP purposes. Explain each.
% How does cross-language QA work? Issues?
% What are the main concepts in a machine learning pipeline?
% What are the problems with precision, recall, and f-measure in IR? How to improve and why is as an example cohen kappa better?
% Calculate P@5, P@10, MAP for the given IR ranked result list (one for IR system A and one for B. Both go up to ten entries and either have a checkmark or an x in each rank).
% Which of the following 4 are correct
%     Features for Naive Bayes classifiers do not need to be uncorrelated
%     Naive Bayes is slow to train
%     SVM itself is a multi-label classifier
%     SVM handles sparse data well
% How to construct a corpus for linguistic purposes from web data and how to make it usable?
% Why are pragmatics and discourse analysis difficult?
% Define hate speech and how to construct a hate speech classifier
% What is counter hate speech and how to construct such a system
% Crowdsourced annotation workflow
% Why are duplicates on the web and how to detect
% Differences in QA vs IR
% Explain applications of multi-language QA and difficulties?
% Explain language detection
% Differences between factual QA and opinionated QA, and which is easier to evaluate?
% What's RAG and how does LangChain help?
% Discuss: Sentiment analysis using supervised vs. lexicon based approaches.
% Calculate the Minimum Edit Distance for "TAC" vs. "CAT" (given is also a table just like on the slides, just without the numbers filled in). List all possible word outcomes (in this case 4).

\sheet{2022}{1st exam}
% 1. How to construct a corpus for linguistic purposes and how to make it usable?
% 2. Why are pragmatics and discourse analysis difficult?
% 3. Calculate SO with given probabilities
% 4. Minimum edit distance
% 5. P@k and map
% 6. Why is precision,recall, f1 not so good for ir. Possible solutions?
% 7. Parse tree and which ambiguity appears
% 8. What is sequence tagging and what is crf
% 9. Define hate speech and how to construct a hate speech classifier
% 10. What is counter hate speech and how to construct such a system
% 11. Discuss sentiment detection: lexicon based vs supervised learning
% 12. Sentiment: lexicon extension with distributed thesauraus
% 13. Why is lexicon based sentiment not enough
% 14. 4 single choice questions about naive bayes and svm
% 15. Approaches for language detection
% 16. Crowdsourced annotation workflow
% 17. Ambiguities for tokenization
% 18. Differences QA vs IR
% 19. Cross language QA and challenges
% 20. Difference stemming and lemmatization
% 21. Why are duplicates in the web and how to detect
% 22. 4 challenges when using the web as corpus
% 23. TREC question types
% 24. What is QALD and challenges

\sheet{2021}{1st exam}
%  Comprehensiveness erklären
% TODO: add picture
%  Weiße Kästen im Bild ausfüllen
%  Zwei Parameter aus NLTK erklären (beide mit c am Anfang)
%  Support Vector Machines erklären wenn man mehrere Klassen hat
%  Unterschied von Features bei Classical ML und Deep Network Based ML
%  Coherenscappa ausrechnen bei einer Tabelle
% o Annotation score
%  Precision, Recall, F1-Score ausrechnen
%  Wo wird in Information Retrieval Classification genutzt
%  Nenne je drei Gründe Pro/Contra ein bestehendes annotiertes Dataset in deutsch in eine
% andere Sprache zu übersetzen
%  Drei Pro/Contra Web als Corpus nutzen
%  Pipeline aus der ersten Folie kennen und drei gegebene Wörter den einzelnen Schritten
% zuordnen
%  MRR ausrechnen von einem Question/Answering System. Accuracy ausrechnen
%  Language Detection erklären
%  Erklären wie man Plagiate finden kann
%  Crowd Sourcing Pro/Contra
%  Erklären wie man den Recall erhöhen kann wenn man eine Suchmaschine benutzt ohne den
% Algorithmus zu ändern
%  Spell Checker erklären bei non-words
%  Information Retrieval Cycle erklären
%  Vector Space bei Information Retrieval erklären
%  Unterschied zwischen Factual sentiment analysis und oppiniated analysis erklären
% o Zwei approaches für oppiniated
%  Bereiche in den Sentiment Analysis verwendet wird nennen
%  Teacher / Student erklären und warum es besser ist als nur Audio.
%  Representation und Fusion erklären und die Unterschiede erklären

\end{document}