%!TEX encoding = UTF-8 Unicode
% ================================================================================
\documentclass[
	fontsize=12pt,
	headings=small,
	parskip=half,           % Ersetzt manuelles Setzen von parskip/parindent.
	bibliography=totoc,
	numbers=noenddot,       % Entfernt den letzten Punkt der Kapitelnummern.
	open=any,               % Kapitel kann auf jeder Seite beginnen.
%	final                   % Entfernt alle todonotes und den Entwurfstempel.
]{scrreprt}
% ================================================================================
% Hinweis: Das Übersetzen dieser Datei funktioniert auch Online mit 
% https://www.overleaf.com. Hierzu müssen neben dieser Datei im gleichen 
% Verzeichnis die Dateien hinweiseabschlussarbeit.bib und stylesvs.tex liegen.
% ================================================================================
%!TEX encoding = UTF-8 Unicode
%!TEX root = hinweiseabschlussarbeit.tex

% Kodierung, Sprache, Patches {{{
\usepackage[T1]{fontenc}    % Ausgabekodierung; ermöglicht Akzente und Umlaute
                            %  sowie korrekte Silbentrennung.
\usepackage[utf8]{inputenc} % Erlaubt die direkte Eingabe spezieller Zeichen;
                            %  utf8 muss die Eingabekodierung des Editors sein.
\usepackage[ngerman]{babel} % Deutsche Sprachanpassungen (z.B. Überschriften).
\usepackage{microtype}      % Optimale Randausrichtung und Skalierung.
\usepackage[
    autostyle,
    ]{csquotes}             % Korrekte Anführungszeichen in der Literaturliste.
%\usepackage{fixltx2e}      % Patches fuer LaTeX2e - seit 2015 nicht mehr nötig
\usepackage{scrhack}        % Verhindert Warnungen mit älteren Paketen.
\usepackage[
  newcommands
]{ragged2e}                 % Verbesserte \ragged...Befehle
\PassOptionsToPackage{
  hyphens
}{url}                      % Sorgt für URL-Umbrüche in Fußzeilen u. Literatur
% }}}

% Schriftarten {{{
\usepackage{mathptmx}       % Times; modifies the default serif and math fonts
\usepackage[scaled=.92]{helvet}% modifies the sans serif font
\usepackage{courier}        % modifies the monospace font
% }}}

% Biblatex {{{
\usepackage[
    style=alphabetic,
    backend=biber,
    %backref=true
    ]{biblatex}             % Biblatex mit alphabetischem Style und biber.
\usepackage[realmainfile]{currfile} % Overleaf unterstuetzt \jobname nicht richtig.
\bibliography{\currfilebase.bib} % Dateiname der bib-Datei (funktioniert auch mit Overleaf)
%\bibliography{\jobname.bib} % Dateiname der bib-Datei (funktioniert nicht mit Overleaf)
\DeclareFieldFormat*{title}{
    \mkbibemph{#1}}         % Make titles italics
% }}}

% Dokument- und Texteinstellungen {{{
\usepackage[
    a4paper,
    margin=2.54cm,
    marginparwidth=2.0cm,
    footskip=1.0cm
    ]{geometry}             % Ersetzt 'a4wide'.
\clubpenalty=10000          % Keine Einzelzeile am Beginn eines Absatzes
                            %  (Schusterjungen).
\widowpenalty=10000         % Keine Einzelzeile am Ende eines Absatzes
\displaywidowpenalty=10000  %  (Hurenkinder).
\usepackage{floatrow}       % Zentriert alle Floats
\usepackage{ifdraft}        % Ermöglicht \ifoptionfinal{true}{false}
\pagestyle{plain}           % keine Kopfzeilen
% \sloppy                    % großzügige Formatierungsweise
\deffootnote{1em}{1em}{
  \thefootnotemark.\ }      % Verbessert Layout mehrzeiliger Fußnoten
% Linien zwischen Kapitelnummer und Titel.
\usepackage[explicit]{titlesec}
\titlespacing*{\chapter}{0pt}{1.05cm}{0.7cm}
\titleformat{\chapter}[hang]{\Large\sffamily\bfseries}{\color{black}\thechapter}{8pt}{\begin{tabular}[t]{@{\color{gray!50}\vrule width 2pt\hspace{8pt}}p{0.85\textwidth}}\raggedright #1\end{tabular}}
\titleformat{name=\chapter,numberless}[hang]{\Large\sffamily\bfseries}{}{0pt}{\raggedright #1\markboth{#1}{#1}}
\titleformat{\section}[hang]{\large\sffamily\bfseries}{{\thechapter.\arabic{section}}}{1ex}{#1}{}
%\setkomafont{disposition}{\normalcolor\bfseries} % Aus der KOMA-Skript-Anleitung: „Mit dieser Änderung verzichten Sie darauf, für alle Gliederungsebenen serifenlose Schrift voreinzustellen“

\makeatletter
\AtBeginDocument{%
    \hypersetup{%
        pdftitle = {\@title},
        pdfauthor  = \@author,
    }
}
\makeatother
% }}}

% Weitere Pakete {{{
\usepackage{graphicx}       % Einfügen von Graphiken.
\usepackage{tabu}           % Einfügen von Tabellen.
\usepackage{multirow}       % Tabellenzeilen zusammenfassen.
\usepackage{multicol}       % Tabellenspalten zusammenfassen.
\usepackage{booktabs}       % Schönere Tabellen (\toprule\midrule\bottomrule).
\usepackage[nocut]{thmbox}  % Theorembox bspw. für Angreifermodell.
\usepackage{amsmath}        % Erweiterte Handhabung mathematischer Formeln.
\usepackage{amssymb}        % Erweiterte mathematische Symbole.
\usepackage{rotating}
\usepackage[
    printonlyused
    ]{acronym}              % Abkürzungsverzeichnis
\usepackage[
    colorinlistoftodos,
    textsize=tiny,          % Notizen und TODOs - mit der todonotes.sty von
    \ifoptionfinal{disable}{}%  Benjamin Kellermann ist das Package "changebar"
    ]{todonotes}            %  bereits integriert.
\usepackage[
    breaklinks,
    hidelinks,
    pdfdisplaydoctitle,
    pdfpagemode = {UseOutlines},
    pdfpagelabels,
    ]{hyperref}             % Sprungmarken im PDF. Lädt das URL-Paket.
    \urlstyle{rm}           % Entfernt die Formattierung von URLs.
%\usepackage{breakurl}
%\def\UrlBreaks{\do\/\do-}
\usepackage{listings}       % Spezielle Umgebung für Quelltextformatierung.
    \lstset{                
        language=C,
        breaklines=true,
        breakatwhitespace=true,
        frame=l,            % Linie links: l, doppelt: L
		framerule=2pt,      % Dicke der Linie (angeglichen an den sekrechten grauen Balken in Kapitelüberschriften)
		rulecolor=\color{gray!50},% Farbe der Linie (angeglichen an den sekrechten grauen Balken in Kapitelüberschriften)
        captionpos=b,
        xleftmargin=6ex,
        tabsize=4,
        numbers=left,
        numberstyle=\ttfamily\footnotesize,
        basicstyle=\ttfamily\footnotesize,
        keywordstyle=\bfseries\color{green!50!black},
        commentstyle=\itshape\color{magenta!90!black},
        identifierstyle=\ttfamily,
        stringstyle=\color{orange!90!black},
        showstringspaces=false,
        }
%\usepackage{filecontents}  % Direktes Einfügen von Dateiinhalt. Wird hier für
                            %  die Verwendung einer .bib-Datei in dieser .tex-
                            %  Datei benötigt.
% }}}

\newcommand{\uhhlogo}{\pdfliteral{%(fold)
q                        % Save graphic state
1090 0 0 353 0 0 cm      % Translation
1 0 0 1 0 0 cm           % Rotation
0.174 0 0 0.174 0 0 cm   % Scaling
BI                       % Begin Image
/W 1090                  % Width
/H 353                   % Height
/CS /RGB                 % Color Space 
/BPC 8                   % Bits Per Component
/F [ /AHx /Fl ]          % Filter ASCIIHexDecode FlateDecode
ID                       % Image Data
789CEDDB4DCE6339AE20D0596FA0870FB5FF85F542B2070F0814AA1CB624FEC9D7E780C3CF124951
0E1989FCE79F7FFEDFFFF9BF4288BFC53F0000DC67FC9528C4CD317D4101007861FC9528C4CD317D
4101007861FC9528C4CD317D4101007861FC9528C4CD317D4101007861FC9528C4CD317D41010078
61FC9528C4CD317D4101007861FC9528C4CD317D4101007861FC9528C4CD317D4101007861FC9528
C4CD317D4101007861FC9528C4CD317D4101007861FC9528C4CD317D4101007861FC9528C4CD317D
4101007861FC9528C4CD317D4101007861FC9528C4CD317D4101007861FC9528C4CD317D41010078
61FC9528C4CD317D4101007861FC9528C4CD317D4101007861FC9528C4CD317D4101007861FC9528
C4CD317D41010078A1E18D37FBC2B4BBA8EE330000CD1ADE78B32F4CBB8BEA3E0300D0ACE18D37FB
C2B4BBA8EE330000CD1ADE78B32F4CBB8BEA3E0300D0ACE18D37FBC2B4BBA8EE330000CD1ADE78B3
2F4CBB8BEA3E0300D0ACE18D37FBC2B4BBA8EE330000CD1ADE78B32F4CBB8BEA3E0300D0ACE18D37
FBC2B4BBA8EE330000CD1ADE78B32F4CBB8BEA3E0300D0ACE18D37FBC2B4BBA8EE330000CD1ADE78
B32F4CBB8BEA3E0300D0ACE18D37FBC2B4BBA8EE330000CD1ADE78B32F4CBB8BEA3E0300D0ACE18D
37FBC2B4BBA8EE330000CD1ADE78B32F4CBB8BEA3E0300D0ACE18D37FBC2B4BBA8EE330000CD1ADE
78B32F4CBB8BEA3E0300D0ACE18D37FBC2B4BBA8EE330000CD1ADE78B32F4CBB8BEA3E0300D0ACE1
8D37FBC2B4BBA8EE330000CD1ADE78B32F4CBB8BEA3E0300D0ACE18D37FBC2B4BBA8EE330000CD1A
DE78B32F4CBB8BEA3E0300D0ACE18D37FBC2B4BBA8EE330000CD1ADE78B32F4CBB8BEA3E0300D0AC
E18D37FBC2B4BBA8EE330000CD1ADE785E9853C67F023C20A6CF100080171ADE785E9853C67F023C
20A6CF100080171ADE785E9853C67F023C20A6CF100080171ADE785E9853C67F023C20A6CF100080
171ADE785E9853C67F023C20A6CF100080171ADE785E9853C67F023C20A6CF100080171ADE785E98
53C67F023C20A6CF100080171ADE785E9853C67F023C20A6CF100080171ADE785E9853C67F023C20
A6CF100080171ADE785E9853C67F023C20A6CF100080171ADE78B32F4CBB8BEA3E0300D0ACE18D37
FBC2B4BBA8EE330000CD1ADE78B32F4CBB8BEA3E0300D0ACE18D37FBC2B4BBA8EE330000CD1ADE78
B32F4CBB8BEA3E0300D0ACE18D37FBC2B4BBA8EE330000CD1ADE78B32F4CBB8BEA3E0300D0ACE18D
37FBC2B4BBA8EE330000CD1ADE78B32F4CBB8BEA3E0300D0ACE18D37FBC2B4BBA8EE330000CD1ADE
78B32F4CBB8BEA3E0300D0ACE78D37F8BC9C7DDF3E75F7DF89C5310300A053CF1B6FEA7939FEBE7D
EAEEBF132B7D0600A059CF336FEA7939FEBE7DEAEEBF132B7D0600A059DB336FE46D39FEBE7DEAEE
BF138B93060040A7B667DEC8DB72FC7DFBD4DD7F2716270D00804E9DCFBCE687E50DEFDBA7EEFE3B
B1386C0000746A7EE97DD75EF14C9EBAFBEFC4FABC0100D0A6FFA5F7151B6525F3D4DD7F27B6460E
00801E232FBD865D2E79DF3E75F7DF89ADA90300A0C7E063EF92C54B4B7EEAEEBF1307B3070040B5
AB1E7B594BDDF3BE7DEAEEBF13B1490400A0C4F3DE7B57BD6F9FBAFBEF446C18010028F1BC27DF55
EFDBA7EEFE3B111B4600004A3CECD577DBFBF6A9BBFF4EC4E6110080124F7AF85DF8BE7DEAEEBF13
B1910400A0C433DE7ED7BE6F9FBAFBEF446C30010028F180E7DFCDEFDBA7EEFE3B119B4D00004A7C
F523F0FEF7ED5377FF9D884D28000025BEF43538FEB85DACE8A9BBFF4EC4E6140080125FF42C1C7F
D08A1F8CC4BB06004096F157A21037C7F4050500E085F157A21037C7F4050500E085F157A21037C7
F4050500E085F157A21037C7F4050500E085F157A21037C7F4050500E085F157A21037C7F4050500
E085F157A21037C7F4050500E085F157A21037C7F4050500E085F157A21037C7F4050500E085F157
A21037C7F4050500E085F157A21037C7F4050500E085F157A21037C7F4050500E085F157A21037C7
F4050500E085F157A21037C7F4050500E085F157A21037C7F4050500E085F157A21037C7F4050500
E085F157A21037C7F4050500E085F157A21037C7F4050500E085F157A21037C7F4050500E085F157
A21037C7F4050500E085F157A21037C7F4050500E085F157A21037C7F4050500E085F157A21037C7
F4050500E085F157A21037C7F4050500E085F157A21037C7F4050500E085F157A21037C7F4050500
E085F157A21037C7F4050500E085F157A21037C7F4050500E085F157A21037C7F4050500E085F157
A21037C7F4050500E085F157A21037C7F4050500E085F157A21037C7F4050500E085F157A21037C7
F4050500E085F157A21037C7F4050500E085F157A21037C7F4050500E085F157A21037C7F4050500
E085F157A21037C7F4050500E085F157A21037C7F4050500E085F157A21037C7F4050500E085F157
A21037C7F4050500E085F157A21037C7F4050500E085F157A21037C7F4050500E085F157A21037C7
F4050500E085F157A21037C7F4050500E085F157A21037C7F4050500E085F157A21037C7F4050500
E085F157A21037C7F4050500E085F157A278523C6FBA466E250000EF8DBF12C593E279D335722B01
00786FFC95289E14CF9BAE915B0900C07BE3AF44F1A478DE748DDC4A0000DE1B7F258A27C5F3A66B
E4560200F0DEF82B517C7B3C7BD24A6F1F000067C65F89E2DBE3D993567AFB00003833FE4A14DF1E
CF9EB4D2DB0700C099F157A2F8F678F6A495DE3E0000CE8CBF12C593E279D335722B0100786FFC95
289E14CF9BAE915B0900C07BE3AF44F1A478DE748DDC4A0000DE1B7F258AA9A838FDEAE9EA9FD881
3B0900C027CD6F42714F540C40E9748D4C6CEB6D040060CDC8FB598C47EE00F4CCD8C8D0565D3C00
0002C69FD3A23F7267A067D2A68636E79A0100906AFC452D7AA2681E4A87ED8601DE2A1000801EE3
AF6BD116E9C3503D6F370CF0418D0000541B7F5A8BB6C81D869E911B9FDEE3320100A833FEB4169D
9135099D23373BBD914A01002832FEAE166D91380C6D53373EBDC14A0100A830FEB4166D91380C3D
5377C300072B0500A0C2F8D35AF444E23CB44D5DE252D57D0300A0D3F8EB5AF447700646A66E6A6E
E3C50200906EFC452D4622320023533735B4F16201004837FE9C1623B13B00370CE1C8C496160E00
C099F1E7B4988AADD3BF6408FB27B6B4700000CE8CBFA5C557C4CF0E6169E100009C197F258AAF88
9F1DC2D2C201003833FE4A145F113F3B84A58503007066FC95284AA368304AA7AE68D9E606020050
67FC992D4AA368304AA7AE68D9E60602005067FC992D4AA368304AA7AE68D9E60602005067FC992D
4AA368304AA7AE68D9E60602005067FC992D4AA368304AA7AE68D9E60602005067FC992D4AA36830
4AA7AE68D9E60602005067FC992D4AA368304AA7AE68D9E60602005067FC992D4AA368304AA7AE68
D9E60602005067FC992D4AA368304AA7AE68D9E60602005067FC992D4AA368304AA7AE68D9E60602
005067FC992D4AA368304AA7AE68D9E60602005067FC992D4AA368304AA7AE68D9E60602005067FC
992D4AA368304AA7AE68D9E60602005067FC992D4AA368304AA7AE68D9E60602005067FC992D4AA3
68304AA7AE68D9E60602005067FC992D4AA368304AA7AE68D9E60602005067FC992D4AA368304AA7
AE68D9E60602005067FC992D4AA368304AA7AE68D9E60602005067FC992D4AA368304AA7AE68D9E6
0602005067FC992D4AA368304AA7AE68D9E60602005067FC992D4AA368304AA7AE68D9E606020050
67FC992D4AA368304AA7AE68D9E60602005067FC992D4AA368304AA7AE68D9E60602005067FC992D
4AA368304AA7AE68D9E60602005067FC992D4AA368304AA7AE68D9E60602005067FC992D4AA36830
4AA7AE68D9E60602005067FC992D4AA368304AA7AE68D9E60602005067FC992D4AA368304AA7AE68
D9E60602005067FC992D4AA368304AA7AE68D9E60602005067FC992D4AA368304AA7AE68D9E60602
005067FC992D12637A9AAEA3A500008F34FEF01689313D4DD7D1520080471A7F788BC4989EA6EB68
2900C0238D3FBC45624C4FD375B41400E091C61FDE2231A6A7E93A5A0A00F048E30F6F9118D3D374
1D2D050078A4F187B71037C7F4050500E085F157A21037C7F4050500E085F157A21037C7F4050500
E085F157A21037C7F4050500E085F157A2288DE9F96AA58100003F62FC992D4A637ABE5A692000C0
8F187F668BBA981EAE6E7A0800F023C65FDAA22EA687AB9B1E0200FC88F197B6A88BE9E11AA08700
00BF60FCA52DEA627AB8063CB887FFFAD7FFFC2DBE6E1780446FBEB812BFBB7A7621C819FD94F197
B6A88BE9E11AF0E01EFA1503BF63EBD2B9A47EC5F08733FA29E32F6D5117D3C335E0C13DF42B26E2
49B5B0EB8B8679F7EA3DEFAA9EF12B863F9CD14F197F698BBA981EAE010FEEA15F31679E57115BBE
E8F47767D56CFFE1570C7F38A39F32FED21675313D5C031EDC43BF620E7875FCB8EF1A80AD6CBFAB
B46A7EC5F08733FA29E32F6D5117D3C335E0C13DF42BE6C0FB57C79716C5BAD201489FA5DB7EC57C
D14DF12B863F9CD14F197F698BA2989EAC190F6EA35F3107FC8AF9651F4FFF6C00EA66E9865F315F
7A53FC8AE10F67F453C61FDBA228A6276BCC53DBE857CC01BF627E59FAAF98EA591AFF15F3BD37C5
AF18FE70463F65FCB12D8A627AB2C63CB58D7EC51CF02BE697F915939840B8B85A7EC5F08733FA29
E38F6D5114D39335E6A96DF42BE6805F31BFEC49BF6252FE3E3D817BF815C31FCEE8A78C3FB64551
4C4FD698A7B6D1AF98037EC5FCB8DCD36F98A5945F31E9BBDF7F53FC8AE10F67F453C61FDBA228A6
276BCC53DBE857CC99E755C496EFFA15F3DF1BE5FEF1FA52DF7559FC8AE10F67F453C61FDBA222EE
99A54BD2F8AE4EBEE457CCB18795C3AEACD37FFC23F67BABF32B863F9CD14F197F6F8B8AB86D8A6E
CBE7DA4EBEE1570CCC7AFC23F67BABF32B863F9CD14F197F6F8B8AB8707EDA52DACAEAAAB4DFF32B
06663DFE11FBBDD5F915C31FCEE8A78CBFB745455C3B3CD726369BF00ABF6260D6E31FB1DF5B9D5F
31FCE18C7ECAF87B5B54C4B593D390D8716E8309AFF02B06663DFE11FBBDD5F915C31FCEE8A78CBF
B7457A5C3E3937E736D5CC158FFF15E35F9C3357FD639D95C65545BDCFAA21BDB61DBFF7F9F7C85F
31379F45565615D59D9DD19D7DBE444567B2667BFCC92DD22365C04AC7E6FE0C3B535DD4F3AF67CA
2E5B8BBC7F18C46B3C58B32293DC35179BD6D6BAD29989ACDCD9F6DDD5CE7629DDBDA1BA22371C4D
E9E275A56D2D9E92527569EB6B3634396BE5DC0C83FDD9DD6EB18AB3F91F7F728BF4381EB0758FCF
B033D545A5FF90E5EEB2B8C8C1F7554F451599F4FC5BD696F3D6DF545494B2787A569184DB361AA9
AE5A4F7A15BB8C1FC1E29A29F9F4D4D559512499AD3573730BF6276BBB78FCE357CC136377C076DD
9F614A926DA92E4AFF76ADDBE5E32291AFAC868A9AD3585F30FE859F95734A510DB51C2F126CF56E
C29D7BF557D7A027BDF45D6E38858F0B66E5D356575B45C164B616EC69D1CA5EBBDB1D777BF144C6
9FDC223D76E779D7EF24D990E7BADCAFD6D25DDE2F12FFD66AA8A82D87C50513BFF353D28E17D553
C8F122590DEFE9C6E5DB35E8492F7D971B0EE2FD6A59F97416D553513C99ADD57AB25AD92B31ED78
FCE357CCE36277980F7C459E5FD4CF4589DF60D5BB347C715557D496C3CA5253DD7BBF4864A3B62A
0633DC4A387DBBABAA6BD0935EFA2ECDE7DE96C37F24D35C546945B9C9F4AFF37EB58F1B25E69C75
16E3AF6E911BBBC37C409E2312BFC1AA77F9967F44225564E5B0B2D454038B76B9A7847B7A9EBEDD
55D535A8686043136E4863BC6FE9757D5132FDEBC45BD4B0C55632E3AF6E911BBBC37C409E2312BF
C1AA77F9967F442255647523F8F1FF582AB77B4567D43900C78BA42499356C771EC1C171D4296A60
75136EC864BC6FE9457D5132FDEBC45B94BE7E249F7FFC8A795CEC0EF301798E48FC06ABDE25F2C5
D8F98F48B08A946E14E59092FCF1BF2C6FB6881F6E4FF383B5671D7430BD4BAA6B905E514F139ACF
3DAB75596DDF5AE7B68A5292C92A6A7D9DDD320FF68A2C7290CCF8AB5B24C6EE249FF99654BF25CF
4589DF60D5BBC4BF12B3FE0509569492436491A9CFAEAC50D1D2DC61FEB8DDEC6A6F564EC927B7C0
DDA59A1DCC6A7A04D34E2CF63887B395537A9558544F4529C96415B5BECE7A81451395BEC2F8C35B
24C6D6A41DFBA9547BF25C91F80D56BD4BCA7763C3977FB090E0EE1F5798DDFDE3C72B66637D9D91
EDEA923F5EA1ADC083DC3AADCF6A5D1CA79D5E6FE2528BCB362C725B4529C96415B5BECE6275F1A9
6EEBDBF8C35B24C6C1BC1DF8A9547BF25C91FB3D53BA4BCAD763C3977FBC90E00A7599C7775F29FF
60FC7AC678B184D9D5B2F414589179A2C5592D8D7B4ACE5A677DCDAC16A574B82799C57CB2662677
F6EAE6B9BFDEF187B7C88AB3913BF045D97E4B9E2BD2BF6DEA76C9FA86AC4E265E4BE4B3EF3FDEF0
6F41B0F6B3D9CB5DAD79BBE6E4E3590D2ED56C6556ABE39E92B3D6C9FDAE88AFD359D1CA3AF14516
2B4A5C67B1B4E379EE49F2DFFF6CFCED2DB2E26CE40EFC54B66D797ED4F3AF67C3F76DCA3F466D15
450A99FAECE23AC1DA2BFEF5DC5DAD79BBE6E4E3590D2ED56C6556ABE39E92B3D6595F2DAB3F291D
4E3CA94B2A4A5C6765B5C830F724F9EF7F36FEF61659713C75BB7E2ADBB63C3FEAF9D7B3E1FB36E5
1FA3CE8A8E6B893421EBB88B72381EBCBA7F3D0FB69B5D2D4B4F81159927FA38570D714FC959EB6C
ADF62D8B6CF527BECE3D99ACAF1619E69E24FFFDCFC6DFDE222B8EA76ED74F65DB96E7473DFF7A5E
F57D7BC92267E5049BF0F15F999408E6B0D8C0DDBA0E963DDB6E76B52C3D0556649EA8E768AE1A80
AC4CB28A4A59E79E45B296BAAABD45AB2DAE5CB4D4F8DB5BA44464EA76FD54C29D79BED7F3AFE755
DFB7972CF2BE9CBFAD1369C2C71DB3A2A2F0C4D20ED6DFDA6E76B52C3D0556649EA8E768AE1A80AC
4CB28A4A59E79E45B296BAAABD45AB2DAE5CB4D4F8F35BA44464EA76FD54C29D79BED7F3AFE755DF
B75FB1C8DFD68934E1FD671323BDEA15B949769650976770EB86022B8A4AD47334FD03D070EE594B
552FB2BECE55CD69686FEE4D0F4E724F92FFFE67E3CF6F911291A9DBF5530977E6F95ECFBF9EF7FC
73765532BBDFF9C17F23DE7F3C3172AB5E54946D4309450D39DBAEA2639DD5E5EA399A1B76493FF7
ABD649A9ABA7398B4B6525F32D37BDBFDEF1E7B7884770EA767D5DCEDF92E77B0DFF7A66EDF22DFF
2CA62CF272A9E037F9C7EDB22292C67AF7720B4CDC2837EDDDD58E372AEAD5FB7D13ABABD07334D5
BB8C9CFB55EBA4D4957B46C1D5B24EEA5B6E7A7FBDE32F70118FE0D4EDFAA99C9BF37CA3F45FCF8F
5B647DFF64E5735532FFBD54E93F7C899158F2AEBAB4EB4A286DC8CA16758DFAB87B4A75751A8EA6
7497C173BF6A9D94BA72CF28B85AD6497DCB4D4F19CEAD15C65FE0221EC1A9DBF5533937E7F946DD
D7CECA1659DFB759F95C95CC7F2C55FD1D9E1589259F294DBEA2849B1B525D60BCBA52D54753BACB
ECB95FB54E4A5D3D376271B5AC93FAA29B1E19D183CF8EBFC04530E223B7EB1BD3FE963CDF28FDDA
F9B845D6F76D563EFDC9BC5FEADF57BBE71FBEA09E3456FED9BAA493A50D39EE434F81C1EAAADD30
AB156B369CFB55EBA4D4D573231657CB3AA92FBAE907B31A99EDF147B808467CE476FD54DAFD79FE
4DEE9758E916DFF2CFE25632EF97FAD7DAAF98DC8DAA75A6B1F50FDF7A0EB925D4352452784F8191
EA1AF4CC6AFA2E379CFB55EBA474F8AA2B9F3533B9B3577D597607FBEC0AFCAFF147B80846CAC86D
F9A9B4FBF3FC9BDC2FB1D22DBEE59FC5AD64DE2FF567B5941E369CF5B56904FF452B2DA1A821C14A
7B0A3CAEAE47CFAC768E53DBB95FB54E4A87AFBAF25933D3367BBB4B1D647B1C2FF71A7F848B48A4
CCDBAE9FCA7C24CF9772BFC44AB7F8967F16B792F9B8DAFF2E98D2C386B3BE3C8DC83F6A75251435
A4AEC6C44C8E2AEBD333AB89BB94CE765626FDEBA474F8AA2B9F3533B9133E7E590EE26F1B8DBFC3
4524B2E66DCB4F653E92E74BB95F62A5EB5FB54E6ED33E76E9AADA83AA472E98C04A0EB925543424
BE664F81FB95B5EA99D5C45DEE39F7ABD649E9F05557BEBAA29B6FFAC7B417E3FD2EE3EF70216E8E
C41B1D94FE3DB6B57ED652FDEBE476ACEEAB786BA383CC0FDC9FC3C734724BA8684870C1B602F72B
6BD533AB3D17FCBB32C95D27A5AE868A5292C95AE4DA9BFE31ED95F8B8CBF82B51889B23F14607A5
7F8FD52D7ED53AE91D2BFA2ADEDAE52CF36FCC2198C6FD87125CB0ADC0FDCA5AF5CC6AD62EF17512
EBBD6A9D94BA1A2A4A4926E5ACAFBDE92B69A7A431FE4A14E2E6C8BAD129D2BFCA16574E5CAD7F9D
F476557F272FEE72967C620E0D097C4C23F2D9F443692EED633E8337B75FCFAC66ED72D5B95FB54E
4A5D0D99A4249375DC17DEF49EFBF8BFC65F8942DC1CB9D72DA8E2DB6C65D9C405FBD7A9F82E5D39
88BA1E56FC437090434302F11C3A0FE5B6EA066F6EBF9E59CDDA25B848EEB95FB54E4A5DA52B3457
F431933B6F7AC365FC63FC9528C4CD917EE322D2BFD02EFF86ACAE71B3ACA5351337AA5B362581B3
DD770573681BECB38644166CBEB96D277EA627F3AC5DAE3AF7ABD649A92BB8426E87234BAD64927E
D37797DA4D3BBEFE7F187F250A7173A4DFB8A0F5AFB5375F17298B9C25D9BF4EC5D7E956038F374A
5F7CF753CDFF18ED26B092436E09E9077DB660DDE0154D72839EB42B6EE2F8B95FB54E4A5D91152E
B956EB679D9ED2EE5229F5468CBF1285B83972AF5BDCEE975B30D293EC5FA7E8BBB4A18D5BBBECAE
10DF3DABA88AAAEB4AC8BD4ABB1F4CD9B4ADBA4E3DE965ED72D5B95FB54E4A5D072D2A1DF5F59577
0FFA38ABDC02CFEA4DD9EB1FBF6284781B29B72CD7F1175DC377E3C70CFBD729FA0AEDF9A29E3DD9
A2BABEBA84F46CEB8EF8A0C69E5D2AF4A47743AB3B0FBD7F9D94BAEEE96D433E6759E516D853EFDF
B61B7F250A7173C46F7485866FC5C8B759D69A29EB147D5D37B47177A3F4C3ED1F8FF4AEA697909E
70D1F99ED5D8B34B859EF46E6875E7A1F7AF9352D7258D6DC8E738B1F41ADBEAFDEFEDC65F8942DC
1CF11B5DE4B66FC5C5DCFAD7A9FBBAAE6EE3D64615E75B545AE77CA697909E704A2B126B2C3D8E3A
3DE925EE72CFB9375494D5969445220D3F76CF717FCC27526662BDEB6731FE4A14E2E648B9D1457A
BE227213EB5F67A4CCDC7F1116F74A3FE2A2D23AE7B3A284F49CE3AD48ACB1FA448AF4A497BBCB25
E79E55517532EBEB441A5B344229BBF7CC5EB0D2947AB74E64FC9528C4CD9172A3EBDCF65DFD31AB
FE754A8BEDE9E7E25EE9075D545AE77C569450917970CDDC1A1BCE255D4F7AE9BBDC70EE59155527
B3BECE59576787A7EDB8D3974AAF77B72DE3AF44216E8EAC1B5DEAAAEFEAF7F9F4AF535A725B4B17
77CC3DEBA2EA3AE7F39E12D2D75CFC6C4F8167BB24EA49AF7F441BCE3DABA2EA64D6D7D96D69DB54
0773484C75FCB2DC7940C0AC4BBE07B2B64B59A7B4F6A9EFD89EEFFCA2EA3A47B4F4802AAA385B6A
FC988E77C9D293DED4A0969E7B5645D5C99C75786B8C7B86F9388DC49CABCB3F68FB8527058C73F1
01801EF15F1F7EC80000009DB27E74F8150300003448FFD1E1570C000050A7E217875F310000409D
A29F1B7EC500000045FC8A010000BE8B5F310000C077A9F8B9E1FF8B010000EA34FFDFFDB9C90300
003F28FD578CFF10030000947AFFA363EBA747D63A0000006FACFC8AF9F81B24BE020000C0BAF51F
329198AE120000780E3F61000080AFE3270C0000F075FC84010000BE8EDF2F0000C0D7F1FB050000
F8527EBC0000005FCDCF160000000000000000000000000000000000000000000000000000000000
00000000000000000000000000000000000000000000000000000000000000000000000000000000
0000000000000000000000FEC3BFFEF53FFF1ED3E9001FB8B3C0DFFCC7F7C3C7E8D9E5CD76D5A9DE
69E450AECDFCFDE229691374EDDDBC24936BFBF3DF7EEA9BF672BF79103794F9459DBFF0C25E98D2
B1C714729548573F4E57CA79256E519DEAAED2EE1595D9D0C01B565E9F9FE0EE297FBC58E6D61F07
276D31D5C534126720E29EAD8B5A94B8C56DA719DFAEA139C14C12D36E6E489DA9AD7B9AD3309323
E758DABDAD0F0E9ED1D646C115E23D59DFF1AAC98F9C4E70E4CE16EF19A77575C97CC571C4175F59
F338EDC5458239A4FCF162B15B7F1C1CB6C5545712C89A81B87BB6BEE402EE5E96C1038DEFD5D09C
6026B9697FCCE7B6235E49726ADFA22E35CCE4C86996B66EEB23F19267272192E4594523A91E3733
7E3A91920F566E98A52D1599549FC8552B37373938155B1F8FECB5D5AE7893FB8FE6ACFF71F76C5D
D4ABEACB3278ACF15D1A9A13CC2437ED8F29DD73B8EB494EED5BD4A886991C39D0D2D66DFD7DBCDE
9449F8B869F068E23DE94935BD8DB3A773B06C75AABBD2D3683891A2958B8E2337D5E048C41B12D9
E86F1F8FF7B9FF680E9A9F6270F794767D4CBBFAB20C1E6E7C8B86E60433C94DFB6356979CEC7B53
5BC7DBBE9270C34C8E9C6969EBEAFEB8AE968FFB06CF25DE93F8D09E7D2AD8C3D9D33958B334CF03
B9695C7E1C15E752D4DEAD3CDFE770FCF1C85E15639355D7D6A6C1014837B87B56C70EA62EF1B20C
1E6E7C8B86E60433C94DBB735AEA4C6D1D6FFB4ACE0D333972ACA5AD8B0FF9482D0779AEA79AB2D7
E2BE891F893770F6740E16AC4BF2CC55A5551F4730F9977F56D1DB372B074722DEE7C8465B45C55B
7D905ED631D519DC3DB169BB5377B6E06D871BDFA2A139C14C72D3AE5B79B1CC14535BA7B4FD63DA
0D333972ACA57D8B7778A496B349882419A9E8A0210D0D9C2A3C3E45BB4B1D9F4E513EC74BBD59B6
FA38B23239EBC65981D5A318BF2F918DB68AAA4B3577AFC5866419DC3DBDC9F15D52123E1B83BA7C
4A5778BF54F3055FFCB39E69A93395DE0F5ED8B37DD793D99DF094A6157D6F1CD49272CD53CA5C49
F2E0E391ED163F1B2C7C259FF58FBCDF656BF7DD3CCFB60EE673B0CECA82A5C7B1B86C45AB8F1B72
B06CD1C7237BC50FB42ED5ACA349BF8CEB9B567C15EC6EBD9240E9ED4E4CB8E2AA46F2A95BE1FD52
CD17BCE88F174B6833959E0B1BB1BB7E5DD38ABE37DE2F923509EBA9A674BB22D5B32DFEB654B0F0
C5ACB27659DF7A37CFB3ADE3F9ECE6397E1C8BCB56F43965E617972DFA7864AFADDADF0FC0F10A07
7B2D56376830E1DB6E776EC20D23115FBFBA39CD17BCE88F174B6833959E0B1BB1BB7EDDDC167D6F
7C5C246512D6534DE976A4FFBB7F9C356CC72BC74FA721E1D2D61DE4B39B677080538E6365D9F43E
D78D7D70CD78EB221BFDEDE36FDA953E127547D36330E7E3A6D5DDEEC4841BA622BE7E75739A2F78
D11F2F96D0662A3D17366277FDBAB92DFADE5859243E09EBA9A6743B3DD5AD65CF1C2F1E2FB921E1
EAEEEDE6539767DD71ACFC71FAA0D64D7E459F2BF6DA2AFF4DBB7247A2EE5CDA0CA6DD330FD5D730
9EE1B1F8FA0DDF51151F6FFEE3C512DA4CA577DC9CADEFC9A75ED8DDF5EBE6B6E87B23380CB9BBFC
EDE3BB59A5A7BAB5EC99F4A6AD17D2907075F776F3D94AB2EEA2ADEFB5B8E6EE1548AC2565E5A28F
F75CAE377FBC7E2891C949BF53A506D3AE9887E3A34C4FB86130E2EB5737A7F98217FDF162096DA6
D2EB39DFA75ED8DDF5EBE6B6E87B637191BF7DF6BF3F5EF19572905562AAEB0B46A437EDE5C713AB
08CE796EF7B6F2D95AE120CF8AE388CF6A624352CE2EB8ECD6C77B2ED79B3F5EEF646472D2EF54A9
C1CC2BE6E1F82873132EBDB307F9D4ADF07EA9E60B5EF4C78B25B4994AEF072FECD9BEC12D0EBE3D
EAFE38584BA4BAE0A1C47BF23E0E523D38D903E94DDBAD65B7CCC5BFECE9DE7A3EBB2B1CE4193C8E
C882EFFF78B7A8D2833B1BC5B3A22255A45C8AF5B483A7BC58D40D06330FB66EB1F989631C9C8AF4
F6C6B7A86E4E3CB6D2DECAAABA2175A6B68E14BEDEBAEA3989E416F1718BC898D5FDF1592D291FAF
D86537A5C5B3E8A9A8A2F0C50CD76B3918E0F77F79BCECB178698979468E63BDA2781F52AA387376
5E67E544AA8837FF63B1BBA91E1FE85506330FB66EB1F989637C363675ED8D6F51DD9C786CA5BD9E
4C4343EA4C6D1D2C7CB175D573B2B5D1419776CB4F99B1F890A7D492FBF18A5D76FF78F1447A2AAA
287CF1E3EBB51CCCF0D95245DD4B292D31D5D2E3F89858F53A070D8927995E4830CFDD3F5ECC3C32
368B155D6230F960F716FB9F38C645F7E2587C977B9AB3D5B4BA83B8E770D7939CDA7777F7C5D67D
C59C1C28CD64EB53F162831D0B4E426497DD3F5E3C979E8A2A0A5FFCF87A2D07937CB65445EBB24A
4BCCB6F4383E66557DC4070D4949F2B827912AB23AB6987C646C162B4A5921EE9EAD8BBA9738C615
972222BED125CDD9ED5BC59AE90DA933B575B0F0C5D65D3227E76DDA2CBF6E6E53FE78AB96DC8F57
EC72F0C72B87D253D16251155BACD77230C3674BED0E6DBC03EBC924265C7A1C8B29C5D72C3DC1E3
DA0FDA12A9626BAFF77FBC927C646C162B4A5921EE9EAD8BBA9738C6E9372228BED70DCD39685DDD
41DC76C42B494EEDBBBBFB62EBBE624E0ED45D87BF2D9EF2C75BB5E47EBC6297833F5E39919E8A16
8BAA389AF55A0E06386B9D2CF1D212D3AE3B8ED29E9C5571E6ECBCCE9A13A9626BAF8F7FFC31FF9E
73293DD9B31C06B72EEA5EE21827DE8514F11D679B73DCC0C4A5EA1A52676AEB60E18BAD9B9D93F3
EE9C965F57E9562629B5E47EBC6297B33FFE78223D152D16D570341F7B129FDE78AF8EC54BCB4A7E
BD0F6D3D9C6D4830A5788B22556CEDF5F18F3FE61F1C9B8AA28ADCB3F56E028BDD4B1CE3945B9028
BEEF5473826D4C59A4BA2175A6B62EEAEDE29F55CFC9795F92CA4FAF742B93DC5A523E5EB1CBF11F
A78C657CF68E7789ACFFF2E3899728D8D574E317E4609DE61EEE6E517A940D772A65AFAD2644BABA
7EEEF109293DD9B31C06B74E99BDC5634A4C383E06C7E29B3634A7E2E37507317E1F574CA5D773BE
4FBDB08913BEF8F75B99D4D572FCF18A5D8EFF7877962A2A5A2CAAE16882BB2C6EBA7505128D5F90
8375461AB8BE4569260D772A65AFAD0E04BBBAB857FC5C4A07EC2C87C1AD535AB7784C89096FCD4C
AEF88E0DCDA9F878DD418CDFC71553E9F59CEFE0853DDB2598CC41F28B7FBCB572692D677575DEE5
483E6D1545F23C5E3C65E4CE8ADACA2751DD05A9BB68533D5C5CBF348D863B95B2D7560716FFF84D
632393535454917BB62E6A5D6287FFB654E90D3DC8A77385F74BF55FF0E0598CDFC71553E9B9B011
073B1635ADE87B2372C5D68B0A8E5CE48FDF8CCD62AAF1C13BDEE578F194913B2E2ADEB1DC7C222B
ECAE93721C5BF3796665F1D21C2AFA5CB1D7560782479FF2F1F4A28ADCB375BC6F5BB73B37E1D24B
7A904FDB0AEF97EABFE0BB577865E5F526F4984ACF858D38D8AEA86945DF1BC7F76BABA8E0C86DFD
FD56FE2B2BC4A7EE7897B395DF7C36F1EEBC596AABBD5952F60AA69D781CD50D6C1BFEAD048A3E5E
71A6C13F7E7F47821FCF2DAAC83D5B175DF0C40EBF592A3809E9F9B4ADF07EA9F5D5122F78FA955C
2CA1CD547AB9DF725B879B9E70FF853DD86BFD235B2B177D6FA40F43E474763F1B4C295E54A4AEB3
5AD6EB8A742C5ED4416E41291B551CF4F17194367071E5BA730C2EB8F5F1F4334DD9E87D6F831F3F
28AAE8622EE630B8F56202B9B73B31E1BA4B7A964FCF0AEF975A5F2DE5D27DFC83B32BB958429BA9
F412BFE22EBFB067DB1D24F366AFADF4E22B97D67250D1F12E679F8D671539A9F5D2563EB5BBF259
4A8917E7E352918E55E473BCC8716383C7119CBAB332B31AF2EF2B1CD47E5642BCDEC50FA6FCFDC7
DE063F9E5857857BB63E6ED7C10115251CB9A115F934ACF07EA9F5D5522EDDC73F389BB1F526F498
4A2FF7C266ED1249B8F3C2A6F4E16F1F69EE70622D07D7FC63B6D517FF6339BB27152CEDAC96E312
0E3A162C2AD2B18A7C22EBEC1E5CD6711C2C1BBC08070D59F9D441EDEFC5FB1CF9D4EE47CE124BF9
784A9E15EED9FAA051C7A75397F04186C7E2BB5437677DB5F48B737610E3F771C5547ACFBBB06719
2626F3729783AC9A3B9C55CBD97D3CA82EF7E2AF1415BC1D5BCD3C4EAFE86852EADADAEE4D9E07B2
B6483C88B3E358CF273DFFBA869CD51E6C5D7A15C7BB04B36A286AB7FF71F76CDD3C0C076BAEFC59
E7C9C6EBAA6ECE7AC95B1F5FFCE38383A898C074535B1F3767AB5DD533B9BE5D66EFF2BAB7BE786E
2615B57C6C72E246EFF74AF9FBC86125D675F9D11C37FF38D55D895BD49DF241CE29E712CCB6A20F
C1F3DAFD78CF99267EA4BFAEADFEC7DDB37551BB12978DFC59457BE375553767BDE4AD8F2FFEF1C1
D8140D61AEA9AD8F9BB39570F54C6EED98D3B8585D2BF9C4FF786BEB9E49E8DC6BF72315E795555A
CFD124AE1CF9B3F584D7E5AE3F781CBBC964655EDA90E3DA83AD4B2C21B24530B186BA76FB1F77CF
D645BD1AB92C3D871BAFABBA39EBF56E7D3C9ED5D9DFF71CEB47535B1F37672BDBEA99DCDD31A171
B1BA5632D9FAFBF8EE3D93D0B9D7C1A72A8E2CA5BA9EA3495C7C3D8794893A2B2D7DC186E3D84D26
25EDEA86446A0FB62EAB84C8FAC1ACAAEB3AE87FDC3D5BDF765F16973AD834D4B570699DCD592F76
EBE3297FBC5545CA286699DAFAB8395BA956CFE4C1A6A1AE85EB5AC961EB23F1047A26A173BBC44F
054F2D5E5DCFB9246EB19549E25C6D6D51B166E289AC7F767DBB47CE4FE2C74BCFF4F8B375E35434
0971F76CFDB0CBD270C4F1BAAA9B13BF35757FFCF2230D031937B5F57173B692AC9EC9837D4F9A95
57D74A0259952EE6D03309293B46D68F67555ADADF76E93994C413D94AA962BA22F9C4574EA9A5A2
87D523B4BBC5DF7609E613F9784AFE5B2B07535A4FE0A0BAC824440C26D0D3A2B35D5E6EB49B4FF5
41C7EBAA6ECE7AA55B1FDFDD6BA42175A6B6EE694BF54C9EED7B50484A5D91C5EB3299BA23753B1E
7FBC3481A2F158CF8A22A5D72457F5A5FEB851E216A5BEE84C77B5CD404A7AD3E95C970FDCE69E3B
724F268CEB1C869BFF498D7B767500897C49C27771670100BC88E0BBB8B300005E44F05DDC59E06F
FE3F3356AE6C
>
EI % End Image
Q  % Graphic state restoration
}}%(end)

% ===================================Dokument===================================
\title{Hinweise für das Erscheinungsbild von Seminar-, Studien- und Bachelor-, Master- und Diplomarbeiten}
\author{Hannes Federrath}
% \date{01.01.2015} % Falls ein bestimmtes Datum eingesetzt werden soll, einfach diese Zeile aktivieren.

\begin{document}

\begin{titlepage}
% \includegraphics[width=6.8cm]{../pic/up-uhh-logo-u-2010-u-farbe-u-rgb.pdf}
\mbox{\parbox[t][1.75cm][b]{2.2cm}{\uhhlogo}}
\begin{center}\Large
	% Universität Hamburg \par
	% Fachbereich Informatik
	\vfill
	\makeatletter
	{\Large\textsf{\textbf{\@title}}\par}
	\makeatother
	\bigskip
	am Arbeitsbereich Sicherheit in Verteilten Systemen (SVS) \par
	\bigskip
	\makeatletter
	{\@author} \par
	\makeatother
	\bigskip
	\makeatletter
	{\@date}
	\makeatother
	\vfill
	\vfill
	%(Muster für das Deckblatt: siehe Anhang dieser Hinweise)
\end{center}
\end{titlepage}

\chapter*{Aufgabenstellung}

Nur Studien-, Bachelor-, Master- und Diplomarbeiten: Soweit eine ausformulierte Aufgabenstellung mit der Betreuerin bzw. dem Betreuer vereinbart wurde, diese bitte hier einfügen.

\chapter*{Zusammenfassung}

Für die eilige Leserin bzw. den eiligen Leser sollen auf etwa einer halben, maximal einer Seite die wichtigsten Inhalte, Erkenntnisse, Neuerungen bzw. Ergebnisse der Arbeit beschrieben werden.

Durch eine solche Zusammenfassung (im Engl. auch Abstract genannt) am Anfang der Arbeit wird die Arbeit deutlich aufgewertet. Hier sollte vermittelt werden, warum man die Arbeit lesen sollte.

\tableofcontents

\chapter{Vorbemerkung}

Um auf die wiederholten Fragen von Studierenden nach dem Umfang, formalen Aufbau und Erscheinungsbild, das bei Seminar-, Studien-, Bachelor-, Master- und Diplomarbeiten erwartet wird, einheitlich zu antworten, wird dieses Dokument bereitgestellt.

Diese Hinweise haben Empfehlungscharakter. Bei Unklarheiten geben die Mitarbeiterinnen und Mitarbeiter der Arbeitsgruppe gerne weitere Auskünfte. Als Muster steht auch eine große Anzahl abgeschlossener Arbeiten zur Ansicht zur Verfügung.

\chapter{Inhalt}
\label{sec.inhalt}

Eine Seminar-, Studien-, Bachelor-, Master- und Diplomarbeit ist eine längere wissenschaftliche Abhandlung, mit der die Studierenden zeigen sollen, dass sie in einem vorgegebenen Zeitrahmen in der Lage sind, wissenschaftlich zu arbeiten. Gelegentlich werden im Rahmen des Studiums auch synonyme Begriffe wie Abschlussarbeit, Hausarbeit oder Projektbericht verwendet. Die hier niedergeschriebenen Empfehlungen gelten für alle genannten Dokumentarten gleichermaßen.

\section{Anforderungen an eine Arbeit}
\label{sec.anforderungen}

Eine Seminar-, Studien-, Bachelor-, Master- und Diplomarbeit trägt inhaltlich normalerweise zu einem aktuell am Arbeitsbereich bearbeiteten Forschungsthema bzw. -projekt bei oder leistet einen Beitrag in der Lehre (z.\,B. Erstellung eines Lehrmittels).

Normalerweise besteht eine Arbeit aus einem darstellenden und einem konstruktiven Teil. Im darstellenden Teil wird gezeigt, dass mit wissenschaftlicher Literatur umgegangen, Wichtiges von Unwichtigem getrennt werden kann und die relevanten Aspekte etwaiger Vorarbeiten erfasst und dargestellt werden können. Im konstruktiven Teil werden dann eigene Lösungen erarbeitet und bewertet.

Um den inhaltlichen und sprachlichen Stil einer wissenschaftlichen Arbeit zu treffen, ist es sehr empfehlenswert, beim Lesen wissenschaftlicher Publikationen auf deren „Klang“ \cite{Tolk2003} zu achten. Die Ich-Form bzw. die Wir-Form sollte im Übrigen vermieden werden.

\section{Aufbau der Arbeit}
\label{sec.aufbau}

Eine wissenschaftliche Arbeit sollte -- wie nahezu jede schriftliche Arbeit -- aus einer Einleitung, einem Hauptteil und einem Schluss bestehen. Der Einleitung ist ein Deckblatt, eine Zusammenfassung und ein Inhaltsverzeichnis voranzustellen. Tabellen- und Abbildungsverzeichnisse sind optional.

Als Muster kann dieses Dokument herangezogen werden.

In der Einleitung wird die Problemstellung und deren Relevanz geschildert. Außerdem werden die Methoden genannt, mit der die Problemstellung bearbeitet wird.

Der Hauptteil sollte mit einem Kapitel zum Stand der Wissenschaft bzgl. des zu bearbeitenden Problems beginnen und das eigene Problem einordnen. Soweit erforderlich, können in einem weiteren Kapitel Grundlagen gelegt werden, z.\,B. Grundverfahren beschrieben werden, die in den folgenden Kapiteln benutzt, ausgebaut oder verändert werden.

Der Schluss fasst die Ergebnisse noch einmal zusammen, bewertet die eigenen Ergebnisse kritisch und benennt die offenen Fragen. Es ist völlig normal, dass im Verlauf der Bearbeitung neue Problemstellungen und Forschungsfragen entstehen, die dann wieder der Ausgangspunkt für weitere Arbeiten sein können.

Ein Literaturverzeichnis am Ende ist obligatorisch. Man sollte sich stets auf die besten Quellen stützen. In abnehmender Qualität:

\begin{enumerate}
	\item Fachbücher, Standards,
	\item Wiss. Zeitschriftenartikel, Survey-Artikel,
	\item Konferenzbeiträge,
	\item Technical Reports, graue Literatur,
	\item Online-Material, Arbeitspapiere, Firmenmaterial, Ausarbeitungen.
\end{enumerate}

Im Internet können zur Feststellung der Qualität und Recherche von Publikationen z.\,B.

\begin{itemize}
	\item Google Scholar (\url{http://scholar.google.com}),
	% \item Microsoft Academic (\url{https://academic.microsoft.com/}),
		% alt: Microsoft Academic Search (\href{http://academic.research.microsoft.com/?SearchDomain=2&SubDomain=2&entitytype=2}{http://academic.research.microsoft.com}) $\to$ computer science $\to$ security \& privacy,
	\item Computer Science Bibliography (\url{http://dblp.uni-trier.de/}) und die
	\item CiteSeerX
		%ehemals Scientific Literature Digital Library 
		(\url{https://citeseerx.ist.psu.edu/})
\end{itemize}

herangezogen werden.

Bei Bedarf kann auch ein Index und Abkürzungsverzeichnis beigefügt werden. Bei Seminar-, Studien-, Bachelor-, Master- und Diplomarbeiten ist dies jedoch normalerweise wegen des überschaubaren Umfangs nicht sinnvoll.

Bei umfangreichen Quelltexten (mehr als 2 Seiten) sollten diese nicht im Fließtext wiedergegeben werden, sondern im Anhang oder (mit Verweis in der schriftlichen Arbeit) auf dem beigelegten Datenträger erscheinen. Dies gilt auch für andere den Lesefluss störende Informationen größeren Umfangs.

Für Prüfungsarbeiten wie Bachelor-, Master- und Diplomarbeiten ist wichtig: Eigenhändig unterschriebene Selbständigkeitserklärung am Anfang oder Ende des Textes nicht vergessen (siehe Muster am Ende dieser Hinweise). Bei Seminararbeiten kann diese entfallen.

\chapter{Form}
\label{sec.form}

Abgesehen von inhaltlichen Anforderungen kann eine klare, gut strukturierte Form den Gesamteindruck der schriftlichen Ausarbeitung steigern. Obwohl die Form keine neuen Inhalte einführt, kann sie sich doch stark auf das Leseverständnis und die Informationsaufnahme der Lesenden auswirken. 

\section{Umfang der schriftlichen Ausarbeitung}
\label{sec.umfang}

Generell gilt: Je weniger Text, umso besser. Auf klare Formulierungen ist in jedem Fall zu achten. Für Studien-, Bachelor-, Master- und Diplomarbeiten ist der Richtwert 40--50 Seiten. 20~Seiten sind zu wenig, 100 sind zu viel. Bei Seminararbeiten genügen 10--15 Seiten (maximal 10 Seiten Text zzgl. Deckblatt, Literaturverzeichnis und ggf. Anhang, insgesamt nicht mehr als 15 Seiten).

Insbesondere für Bachelor-, Master- und Diplomarbeiten gilt: Wo immer möglich, sollte auf andere relevante Veröffentlichungen verwiesen werden, anstatt deren Inhalt noch einmal wiederzugeben. Für alle Aussagen und Darstellungen, die aus Veröffentlichungen stammen, sind Quellenangaben zu machen. 

Bei Inhalten aus fremden Quellen, die paraphrasiert oder wörtlich übernommen werden, ist die Quellenangabe an der Textstelle zu machen. Es genügt nicht, die Quelle ins Literaturverzeichnis aufzunehmen. Wörtliche Übernahmen von längeren Wortgruppen und ganzen Sätzen sind in Anführungszeichen zu setzen.

Die sprachliche Leistungsfähigkeit von generativen KI-Programmen wie etwa ChatGPT kann dazu verleiten, Zusammenfassungen fremder Texte einer KI zu überlassen. Der Einsatz solcher Hilfsmittel zur Erstellung eigener wissenschaftlicher Texte muss klar gekennzeichnet sein. Auch hier gilt: Die Quellenangabe ist an der Textstelle zu machen.

Viele Studierende haben zu Beginn der Bearbeitung Sorge, dass sie womöglich nicht auf die „übliche“ Seitenzahl kommen. Diese Sorge ist meist unbegründet. Es sollte möglichst früh mit dem Schreiben begonnen werden: Dokumentieren Sie, was Sie gelesen und neu erarbeitet haben.

\section{Gestaltung}

Wissenschaftliche Arbeiten, die am Arbeitsbereich betreut werden, sollen mit LaTeX gesetzt sein. Ausnahmen von dieser Regel (etwa die Verwendung von Open Office oder Word) können in Absprache mit der Betreuerin bzw. dem Betreuer getroffen werden.

Als Hauptschriftart sollte eine mit Serifen verwendet werden, z.\,B. Times (wie dieser Text). In LaTeX kann auch Latin Modern (\textbackslash usepackage\{lmodern\}) verwendet werden oder besser, falls möglich, die Postscript-Schrift Times (\textbackslash usepackage\{mathptmx\}). Bitte verwenden Sie keine \textsf{Helvetica} oder \textsf{Arial}, da diese Schriften bei längeren Texten schwerer lesbar sind. In Überschriften ist eine serifenlose Schrift jedoch in \textsf{\textbf{Bold}} erlaubt, wie in diesem Beispiel.

Die Schriftgröße sollte 12pt (wie dieser Text), zur Not auch 11pt sein. Eine Größe von 10pt ist zu klein! Als Zeilenabstand sollten möglichst 15pt oder 14pt verwendet werden. 1,5-zeilig entspricht etwa 18pt und ist zu viel. Bei LaTeX sind keine benutzerdefinierten Abstände nötig. Der Text ist im Blocksatz zu setzen. Ränder bei A4-Papier: ca. 2,5--3cm rundherum. In LaTeX erzeugt beispielsweise \textbackslash usepackage[a4paper, margin=2.54cm, marginparwidth=2.0cm, footskip=1.0cm]\{geometry\} einen geeigneten Satzspiegel (wie dieses Dokument).

Es sollten möglichst nicht mehr als drei Gliederungsebenen verwendet werden. Die Nummerierung einer Überschrift erfolgt stets \emph{ohne} nachgestellten Punkt. Weiterhin sollte eine Überschrift nie allein für sich stehen, sondern von einem Text begleitet werden. Folgt auf eine Hauptüberschrift direkt eine Unterüberschrift, könnte zwischen den beiden Überschriften beispielsweise eine Einführung zum jeweiligen thematischen Abschnitt oder ein Überblick über die darauffolgenden Unterabschnitte gegeben werden. In diesem Dokument sind dies beispielsweise zwischen den Überschriften zu Kapitel~\ref{sec.form} und Abschnitt~\ref{sec.umfang} zwei Sätze, die das Kapitel einleiten.

Innerhalb eines Kapitels sollte eine Untergliederung immer dann verwendet werden, wenn es deutlich mehr als zwei Seiten Text umfasst. Bei der Untergliederung ist zu beachten, dass stets wenigstens zwei Unterabschnitte vorhanden sein sollten. Das Kapitel~\ref{sec.inhalt} dieses Textes hat beispielsweise die Unterabschnitte~\ref{sec.anforderungen} und \ref{sec.aufbau}, und es wäre eigenartig, wenn ein Kapitel~x nur den Unterabschnitt~x.1 hätte und kein weiterer folgen würde. Dann ist es besser, auf eine Untergliederung zu verzichten.  

Eine Kopfzeile kann verwendet werden, muss aber nicht. Hier wird oft unnötig Zeit verschwendet. 

Bitte benutzen Sie nur einen Absatztyp (wie in diesem Dokument; wird in LaTeX durch mindestens eine Leerzeile zwischen den Absätzen erzeugt). Es ist weit verbreitet, Gedanken, die irgendwie zusammenhängen, aber aus Sicht des Autors noch keinen neuen Absatz rechtfertigen, auf einer neuen Zeile zu beginnen -- in LaTeX meist durch {\textbackslash\textbackslash} erzeugt.\\ Dies ist zu vermeiden, weil es das Textbild uneinheitlich und unruhig macht. Man soll zwar keine Negativbeispiele bringen, aber der Zeilenwechsel vor dem vorherigen Satz ist eines.

Dieses Dokument wurde mit LaTeX erstellt und steht übrigens auch im Quelltext (.tex-File) zur Verfügung und kann für eigene Zwecke weiterverwendet werden. Die Befehlsfolge zum Erzeugen eines PDF aus diesem Dokument lautet etwa:

\begin{lstlisting}[numbers=none,xleftmargin=6pt]
pdflatex hinweiseabschlussarbeit.tex
biber hinweiseabschlussarbeit.bcf
pdflatex hinweiseabschlussarbeit.tex
pdflatex hinweiseabschlussarbeit.tex
\end{lstlisting}

Weiterführende Literatur zum Schreiben wissenschaftlicher Arbeiten mit LaTeX findet sich beispielsweise in \cite{Schl2013}.

Für detaillierte Informationen zu typographischen Regeln sowie Beispiele der korrekten Umsetzung dieser Regeln sei auf die kompakte und sehr hilfreiche Arbeit „typokurz -- Einige wichtige typographische Regeln“ von Christoph Bier verwiesen \cite{Bier2009}.

Häufig werden die Regeln zum Setzen von Text in Anführungszeichen missachtet. Im Deutschen sollten nur die „Gänsefüßchen“ (links nach unten und rechts nach oben geschwungen) verwendet werden. Genaues Hinschauen ermöglicht hier die korrekte Verwendung: Weder „dies” noch {\verb#"#}das{\verb#"#} noch “jenes” ist korrekt.

Bei der Kommasetzung bietet es sich an, noch einmal die Regeln unter \url{https://www.duden.de/sprachwissen/rechtschreibregeln/komma} anzuschauen. Sehr häufig werden notwendige Kommas weggelassen, etwa beim erweiterten Infinitiv mit zu (genauer: bei satzwertigen Infinitivgruppen, vgl. Duden, Regel D117). Dagegen werden Kommas leider immer wieder dort gesetzt, wo sie nicht hingehören, z.\,B. werden keine Kommas vor „etc.“ und „sowie“ gesetzt.

Beim Verwenden von zusammengesetzten Substantiven und anderen Aneinanderreihungen, wie sie in informatischen Texten sehr häufig vorkommen, gilt im Deutschen die Regel, dass diese üblicherweise mit einem Bindestrich verbunden werden. Man schreibt etwa DES-Verschlüsselung, IP-Adresse und Public-Key-Verfahren. Näheres zum Nachlesen findet sich in den deutschen Rechtschreibregeln. Eine E-Mail ist eine elektronisch übermittelte Nachricht, während Email ein Keramiküberzug ist. Man schreibt im Deutschen selten „Netzwerk“, wenn man ein Kommunikationsnetz oder Rechnernetz meint. Als Kurzform für das Englische \emph{computer network} verwendet man den Begriff Netz.

Wenn es möglich ist, ein zusammengesetztes Wort ohne Bindestrich zu schreiben, dann sollte davon Gebrauch gemacht werden. So sollte beispielsweise besser „IT-Sicherheitsmanagement“ anstelle von „IT-Sicherheits-Management“ geschrieben werden.

Zwar mögen solche Formfragen aus Sicht des wissenschaftlichen Gehalts eines Textes eher nachrangig sein, allerdings verhilft eine saubere, fehlerfreie und konsistente Form zu einem positiven Gesamteindruck. Dagegen kann eine hohe Rate an Rechtschreib- und Interpunktionsfehlern einen inhaltlich guten Text eigentlich nur schwächen. 

Weniger kann übrigens manchmal mehr und Besseres bewirken. Spiegel Online berichtete in \cite{textwahrnehmung} beispielsweise, dass einfache, klare Sprache und eine gut lesbare Standardschrift die Textwahrnehmung verbessern kann: „Schreib so einfach und deutlich wie möglich, dann hält man dich eher für intelligent.“

Eine interessante Regel für klares Schreiben ist die „Daumenregel“, die durchaus wörtlich zu nehmen ist: Wenn ein Satz durch Weglassen (mit dem Daumen Verdecken) eines Wortes oder einer Wortgruppe noch immer den gewünschten Sinn ergibt, dann sollte dieses Wort bzw. die Wortgruppe gestrichen werden. Diese Regel kann auch auf ganze Sätze oder gar Abschnitte erweitert werden.

Darüber hinaus sollten beim Schreiben unbestimmte Aussagen vermieden werden. Wenn etwa in einem Text die Rede davon ist, dass es „verschiedene Verfahren zur Angriffserkennung gibt“, dann ist es für die positive Wahrnehmung der entsprechenden Textstelle hilfreich, entweder eine Quellenangabe zu machen, bei der diese verschiedenen Verfahren zur Angriffserkennung genannt und näher erläutert werden, oder kurz und unmittelbar die konkreten Verfahren zu nennen, um die Leserinnen und Leser nicht im Unklaren zu lassen. Wenn öfter von „unterschiedlichen“, „verschiedenen“ oder „mehreren“ Dingen geschrieben wird, sollte ernsthaft die Anwendung der vorangegangenen Daumenregel in Betracht gezogen werden. 

Tipps und Textbausteine sowie die Grundlagen des wissenschaftlichen Schreibens sind in \cite{Küht2021} sehr ansprechend und mit zahlreichen Beispielen und Gegenbeispielen zu finden.

\section{Abbildungen, Tabellen und Listings}

Abbildungen sollten möglichst schlicht, in schwarzweiß und als Strichzeichnungen gestaltet sein. Wenn schon Farben verwendet werden, dann bitte in \emph{allen} Abbildungen das gleiche Farbschema verwenden. Farben sind nur dann sinnvoll, wenn sie einen Sachverhalt deutlich unterstreichen oder veranschaulichen. Es ist zu beachten, dass die Aussagekraft auch in einem Schwarzweiß-Ausdruck erhalten bleiben muss.

Die Auflösung muss ausreichend groß gewählt werden, damit im fertigen Dokument weder Pixel noch Treppen oder Unschärfe erkennbar sind. Deshalb möglichst Vektorgrafiken verwenden.

Gleitobjekte (sog. Floats) wie Abbildungen und Tabellen müssen eine Unterschrift erhalten. Auf diese muss zudem im Text eindeutig verwiesen werden, da durch das automatische Setzen unter Umständen nicht ersichtlich ist, zu welchem Textabschnitt eine Abbildung gehört. Wie das aussehen kann, ist anhand von Abbildung~\ref{fig:bsp} ersichtlich.

\begin{figure}[ht]
\sffamily\footnotesize
%\includegraphics[width=0.6\textwidth]{abb.pdf}
\unitlength=0.75mm
\special{em:linewidth 0.4pt}
\linethickness{0.4pt}
\begin{picture}(111,73)(0,0)
	\put(21,59){\makebox(0,0)[cc]{}}
	\put(1,55){\line(1,0){40}}
	\put(1,55){\line(0,1){8}}
	\put(41,55){\line(0,1){8}}
	\put(1,63){\line(1,0){40}}
	\put(91,59){\makebox(0,0)[cc]{}}
	\put(71,55){\line(1,0){40}}
	\put(71,55){\line(0,1){8}}
	\put(111,55){\line(0,1){8}}
	\put(71,63){\line(1,0){40}}
	\put(21,19){\makebox(0,0)[cc]{}}
	\put(1,15){\line(1,0){40}}
	\put(1,15){\line(0,1){8}}
	\put(41,15){\line(0,1){8}}
	\put(1,23){\line(1,0){40}}
	\put(91,19){\makebox(0,0)[cc]{}}
	\put(71,15){\line(1,0){40}}
	\put(71,15){\line(0,1){8}}
	\put(111,15){\line(0,1){8}}
	\put(71,23){\line(1,0){40}}
	\put(31,43){\makebox(0,0)[cc]{F}}
	\put(24,38){\line(1,0){14}}
	\put(24,38){\line(0,1){10}}
	\put(38,38){\line(0,1){10}}
	\put(24,48){\line(1,0){14}}
	\put(81,43){\makebox(0,0)[cc]{F}}
	\put(74,38){\line(1,0){14}}
	\put(74,38){\line(0,1){10}}
	\put(88,38){\line(0,1){10}}
	\put(74,48){\line(1,0){14}}
	\put(31,30){\circle{6}}
	\put(81,30){\circle{6}}
	\put(79,30){\line(1,0){4}}
	\put(81,28){\line(0,1){4}}
	\put(29,30){\line(1,0){4}}
	\put(31,28){\line(0,1){4}}
	\put(21,63){\line(0,1){7}}
	\put(21,63){\vector(0,-1){0.12}}
	\put(31,48){\line(0,1){7}}
	\put(31,48){\vector(0,-1){0.12}}
	\put(31,33){\line(0,1){5}}
	\put(31,33){\vector(0,-1){0.12}}
	\put(31,23){\line(0,1){4}}
	\put(31,23){\vector(0,-1){0.12}}
	\put(21,8){\line(0,1){7}}
	\put(21,8){\vector(0,-1){0.12}}
	\put(11,30){\line(1,0){17}}
	\put(28,30){\vector(1,0){0.12}}
	\put(91,63){\line(0,1){7}}
	\put(91,63){\vector(0,-1){0.12}}
	\put(81,48){\line(0,1){7}}
	\put(81,48){\vector(0,-1){0.12}}
	\put(81,33){\line(0,1){5}}
	\put(81,33){\vector(0,-1){0.12}}
	\put(81,23){\line(0,1){4}}
	\put(81,23){\vector(0,-1){0.12}}
	\put(91,8){\line(0,1){7}}
	\put(91,8){\vector(0,-1){0.12}}
	\put(84,30){\line(1,0){17}}
	\put(84,30){\vector(-1,0){0.12}}
	\put(11,30){\line(0,1){25}}
	\put(101,30){\line(0,1){25}}
	\put(21,73){\makebox(0,0)[cc]{Klartext M}}
	\put(91,73){\makebox(0,0)[cc]{Chiffretext C}}
	\put(21,3){\makebox(0,0)[cc]{Chiffretext C}}
	\put(91,3){\makebox(0,0)[cc]{Klartext M}}
	\put(56,43){\makebox(0,0)[cc]{K}}
	\put(56,48){\makebox(0,0)[cc]{Schlüssel}}
	\put(38,43){\line(1,0){13}}
	\put(38,43){\vector(-1,0){0.12}}
	\put(61,43){\line(1,0){13}}
	\put(74,43){\vector(1,0){0.12}}
	\put(21,55){\line(0,1){8}}
	\put(21,15){\line(0,1){8}}
	\put(91,55){\line(0,1){8}}
	\put(91,15){\line(0,1){8}}
	\put(11,59){\makebox(0,0)[cc]{L}}
	\put(31,59){\makebox(0,0)[cc]{R}}
	\put(81,59){\makebox(0,0)[cc]{R}}
	\put(11,19){\makebox(0,0)[cc]{R}}
	\put(81,19){\makebox(0,0)[cc]{L}}
	\put(101,19){\makebox(0,0)[cc]{R}}
	\put(13,52){\line(1,0){18}}
	\put(81,52){\line(1,0){18}}
	\put(99,23){\line(0,1){29}}
	\put(99,23){\vector(0,-1){0.12}}
	\put(13,23){\line(0,1){29}}
	\put(13,23){\vector(0,-1){0.12}}
	\put(31,19){\makebox(0,0)[cc]{L'}}
	\put(101,59){\makebox(0,0)[cc]{L'}}
\end{picture}
\caption{Beispiel für eine Abbildung}
\label{fig:bsp}
\end{figure}

Ein Abbildungsverzeichnis ist nicht unbedingt erforderlich, kann aber bei einer Vielzahl von verwendeten Abbildungen für Übersichtlichkeit sorgen.

Längere Listings sollten wie Abbildungen in einer Float-Umgebung untergebracht werden, d.h. eine Über- bzw. Unterschrift haben. Ein Beispiel zeigt Listing~\ref{lst:ggt}.

\begin{lstlisting}[float,caption={Berechnung des größten gemeinsamen Teilers zweier ganzer Zahlen a und b},label={lst:ggt}]
int getGGTOf(int a, int b) {
    // requires ((a > 0) && (b > 0)); ensures return > 0;
    int h;
    while (b != 0) {
        h = b;
        b = a % b; // % is the modulo operator. This line is long enough to show how line breaks in lstlisting are handled.
        a = h;
    }
    return a;
}
\end{lstlisting}

\section{Literaturverzeichnis}
\label{sec:literaturhowto}

Nachfolgend werden einige Hinweise für die Angaben im Literatur- bzw. Quellenverzeichnis gegeben. Es ist wichtig, dass alle für den jeweiligen Quellentyp relevanten Informationen angegeben werden. Zudem sollte darauf geachtet werden, dass die Quellenangaben stets einheitlich erfolgen, also beispielsweise die Autoren konsequent zuerst mit Vorname und dann mit Nachname genannt werden. In den folgenden Syntaxbeschreibungen sind optionale Angaben in eckigen Klammern angegeben.

\begin{description}

	\item[Zitierweise für Fachbücher:] \mbox{}\\[1ex]
	Syntax: Vorname Nachname. Buchtitel. {[}Auflage,{]} Erscheinungsort: Verlag, Jahr. \\[1ex]
	Beispiele: \cite{Beut2009,ScWe2007,Pfit90}

	\item[Zitierweise für Zeitschriften:] \mbox{}\\[1ex]
	Syntax: Vorname Nachname. Artikeltitel. Zeitschrift Jahrgang/Volume (Jahr), Seiten. \\[1ex]
	Beispiele: \cite{Kili2006,Lamp1981,ThKZ2002,Chau81,Chau88}

	\item[Zitierweise für Konferenzbeiträge:] \mbox{}\\[1ex]
	Syntax: Vorname Nachname. Beitragstitel. {[}Herausgeber/Editoren.{]}
	Konferenzband. {[}Volume. Buchserie.{]} Ort{[: Verlag]}, Datum, Seiten. \\[1ex]
	Beispiele: \cite{InBr2009,WWPK2010,HSFN2009,GoRS99,WaMS2008}

	\item[Zitierweise für Online-Quellen:] \mbox{}\\[1ex]
	Syntax: Vorname Nachname. Titel. {[}Quelle.{]} Datum. URL (Zugriffszeitpunkt). \\[1ex]
	Beispiele: \cite{CCC2009,Heise2011,textwahrnehmung}

\end{description}

Die Literatur sollte im Text durch alphanumerische Kürzel mit Erscheinungsjahr in eckigen Klammern angegeben werden.
% ----- Der nachfolgende Text ist obsolet geworden durch die konsequente Verwendung von Bibtex:
% Bei einem Autor werden meist die ersten drei Buchstaben des Nachnamens % % verwendet (Beispiel: \cite{Beut2009}), bei zwei oder drei Autoren werden die % Anfangsbuchstaben der Nachnamen verwendet (Beispiel: \cite{InBr2009}). Bei % vier oder mehr Autoren werden die ersten drei Buchstaben des Nachnamens des % ersten Autors gefolgt von einem Pluszeichen verwendet (Beispiel: \cite{HSFN2009}).
% ----- Ende des obsoleten Textes
Bei Online-Quellen ohne referenzierbare Autoren und Titel (z.B. Webseiten von Firmen und Organisationen) kann anstelle eines Eintrags im Literaturverzeichnis auch vereinfachend eine Fußnote mit der URL gesetzt werden. Hinter der URL ist der Zugriffszeitpunkt zu vermerken. Einige Fußnoten in diesem Text dienen als Beispiele.

In diesem Dokument wurde für die Erzeugung des Literaturverzeichnisses BibLaTeX verwendet. Im Literaturverzeichnis auf Seite~\pageref{sec:literaturverzeichnis} können Beispiele angeschaut werden. Solange die Angaben zu einer Quelle vollständig sind, erzeugt BibTeX automatisch eine korrekte Quellenangabe, die allerdings je nach verwendetem Bibstyle (hier: alphabetic) von den o.a. Hinweisen abweichen kann, was in Ordnung ist, solange die Angaben vollständig und einheitlich sind.

Teilweise wird in den Seminaren die Erstellung einer Literaturliste mit den fünf wichtigsten und besten Quellen gefordert. Im Anhang findet sich auf Seite~\pageref{sec:literaturliste} ein Beispiel für eine solche Literaturliste.

\section{Wikipedia und generative KI als Quellenangabe}

Grundsätzlich sollte bei der Literaturarbeit darauf geachtet werden, dass die Originalquelle
%für eine Information oder einen Sachverhalt
referenziert wird. Referenzen auf Wikipedia und Sekundärliteratur sollten daher möglichst vermieden werden. 

Im wissenschaftlichen Kontext kann aber etwa aus didaktischen Gründen eine Referenz auf Inhalte aus Wikipedia, Lehrbücher und Sekundärliteratur trotzdem sinnvoll sein. Bei der Referenz auf einen Eintrag der Wikipedia sollte, wie auch bei anderen Online-Quellen, der Zugriffszeitpunkt angegeben werden. Der Verweis auf eine bestimmte Version eines Eintrags wird innerhalb von Wikipedia mittels der sog. \emph{oldid} realisiert. Ein Beispiel hierfür ist \cite{Wiki}.

Werden generative KI-Programme wie etwa ChatGPT\footnote{\url{https://openai.com/chatgpt}, Abruf am 14.07.2023} oder GitHub Copilot\footnote{\url{https://github.com/features/copilot/}, Abruf am 14.07.2023} innerhalb einer wissenschaftlichen Arbeit dazu verwendet, z.\,B. Programmschnippsel oder Text zu erzeugen, sind diese mit einer Quellenangabe klar als solche zu kennzeichnen. Es wird empfohlen, sowohl die Anfragen an als auch die Ausgaben  von generativer KI in der eigenen Materialsammlung zu sichern, um später ggf. noch genau nachvollziehen zu können, was die Eigenleistung im Zusammenhang mit solchen Hilfsmitteln war.

\section{Angabe von Literaturreferenzen im Fließtext}
\label{sec:literaturfliesstext}

Beispiele für die korrekte Angabe von Referenzen im Fließtext finden sich an vielen Stellen in diesem Dokument. Im Fließtext sollen Quellenangaben immer möglichst dicht an der jeweiligen Paraphrasierung genannt werden. Die nachfolgenden Satzbeispiele zeigen, wie so etwas gehen kann:

Das Mix-Netz \cite{Chau81} und das DC-Netz \cite{Chau88} bilden die Grundlage vieler moderner Verfahren zum Schutz vor Beobachtung im Internet. In \cite[S.~13]{Beut2009} wird ein Algorithmus zur statistischen Analyse der Cäsar-Chriffre angegeben. Soll bei einer Quellenangabe auch eine konkrete Seitenangabe gemacht werden, gelingt dies mit \textbackslash cite[S.\textasciitilde 13]\{XYZ\}, siehe das Beispiel zuvor.

Manchmal möchte man, dass der Namen der Autorin bzw. des Autors einer zitierten Quelle im Fließtext erscheint, wie das nachfolgende Textbeispiel zeigt. So stellte etwa \textcite{Chau81} das Mix-Netz vor und hat mit dieser und weiteren bahnbrechenden Arbeiten das neue Teilgebiet der Privacy Enhancing Technologies (PET) innerhalb der IT-Sicherheitsforschung etabliert.

Gelegentlich sieht man auch die Quellenangabe am Ende eines Satzes. Wenn diese Form der Quellenangabe genutzt wird, dann bitte die Quellenangabe \emph{vor} das Satzzeichen setzen, wie in diesem Satz \cite{XYZ}. Danach kann es im Absatz mit weiteren Aussagen weitergehen, die sich auf andere Quellen beziehen können.

Die Übernahme längerer paraphrasierter Passagen aus bereits veröffentlichten Texten ist eher unerwünscht. Gelegentlich sieht man die Quellenangabe bei solchen aus mehreren Sätzen bestehenden paraphrasierten Passagen am Ende des Absatzes, wie hier in diesem Absatz demonstriert. Dann sollten sich innerhalb dieses Absatzes nur Aussagen finden, die sich auf diese eine Quelle beziehen, die dann \emph{nach} das letzte Satzzeichen des Absatzes gestellt wird. \cite{XYZ}

Bei Übernahme längerer paraphrasierter Inhalte sollte anstelle der nach dem letzten Satzzeichen gestellten Quellenangabe besser der jeweilige Absatz bzw. der jeweilige Abschnitt zu Beginn im Fließtext explizit darauf hinweisen, dass sich die nachfolgenden Bemerkungen auf die Quelle \cite{XYZ} beziehen. Eine wörtliche Übernahme aus Fremdtexten ist aber auch dann nicht erlaubt, abgesehen von kurzen Zitaten, die stets in Gänsefüßchen zu setzen sind.

\section{Vor der Abgabe}

Vor der Abgabe sollten die Funktionen zur Rechtschreibprüfung und Silbentrennung genutzt werden. In LaTeX können dazu spezielle Entwicklungsumgebungen verwendet werden, die für jedes gängige Betriebssystem verfügbar sind. TeXworks\footnote{\url{http://www.tug.org/texworks/}, Abruf am 14.07.2023} ist beispielsweise ein kostenloser plattformunabhängiger LaTeX-Editor.

Zusätzlich lohnt es sich, den Text vor Abgabe von jemandem lesen zu lassen (Freund, Freundin, Bekannte, Haustier), damit er auch sprachlich noch einmal überprüft wurde. Zudem ist auf die korrekte Kommasetzung zu achten.

Bachelor-, Master-, Seminar- und Studienarbeiten können inzwischen zumeist rein elektronisch abgegeben werden. Sofern Arbeiten in gedruckter Form abgegeben werden müssen, sollten sie möglichst gebunden sein. Eine einfache Heissleim-, Kaltleim oder Klemmbindung (für ca. 3~EUR pro Exemplar aus dem Copyshop) genügt. Es sollte möglichst doppelseitig gedruckt werden, um Papier zu sparen. Die Umwelt dankt es. Schwarzweiß-Druck genügt in den meisten Fällen völlig.

Zusätzlich muss die Arbeit noch einmal als PDF-Datei per Mail an die Betreuerin oder den Betreuer geschickt werden. Falls Quellcodes und Programme erstellt wurden, sollte vor Abgabe mit der Betreuerin oder dem Betreuer besprochen werden, in welcher Weise diese abzugeben sind.

\chapter{Betreuung und Bewertung der Arbeit}

Für die Betreuung der Arbeit steht die bzw. der mit der Ausgabe der Aufgabenstellung genannte Betreuerin oder Betreuer zur Verfügung. Bitte nutzen Sie die Kontaktmöglichkeiten im Rahmen der Sprechstunden und nach Vereinbarung für regelmäßige Gespräche (mindestens etwa alle 2--3 Wochen). Sinnvollerweise sollte Sie jeweils darauf vorbereitet sein, einen kurzen mündlichen Bericht über den Stand der Bearbeitung zu geben. Während der Vorlesungszeit finden möglicherweise regelmäßige Treffen aller Bearbeiterinnen und Bearbeiter von Abschlussarbeiten statt, zu denen ggf. kurzfristig eingeladen wird. Jedes Gesprächsangebot sollte wahrgenommen werden!

\section{Schriftlicher Teil}

Folgende \textbf{Meilensteine} sollten bereits zu Beginn der Bearbeitung des Themas im Kalender vermerkt werden:

Bei \textbf{Arbeiten mit etwa 3-monatiger Bearbeitungszeit} soll \textbf{nach 1,5 Monaten} die \textbf{Abgabe eines ersten Textentwurfs} bei der Betreuerin bzw. beim Betreuer erfolgen. Wenn mit der Zweitbetreuerin bzw. dem Zweitbetreuer (soweit vorhanden) nichts anderes vereinbart ist, sollte ihr bzw. ihm zu diesem Zeitpunkt ein Zwischenbericht geliefert werden und ggf. nachgefragt werden werden, ob der aktuelle Textentwurf zur Kommentierung überlassen werden soll.

Bei \textbf{Arbeiten mit etwa 6-monatiger Bearbeitungszeit} soll \textbf{nach 2 Monaten} ein etwa 12-seitiger Textentwurf inkl. Gliederungsentwurf vorliegen und \textbf{nach weiteren 2 Monaten} ein erster vollständiger Textentwurf.

Die Textentwürfe werden von uns gelesen, kommentiert und zurückgegeben. Die Meilensteine dienen der Fortschrittskontrolle und sind für die endgültige Bewertung der Arbeit bedeutungslos; Fehler dürfen sorgenfrei gemacht werden.

\section{Bewertung}

Typische Kontrollfragen zur Beurteilung einer Arbeit sind:

\begin{itemize}
	\item Wurde die Fragestellung auf hohem Niveau bearbeitet?
	\item Handelt es sich um eine kreative Herangehensweise bzw. Lösung?
	\item Sind die Annahmen und getroffenen Voraussetzungen realistisch, oder wurden unzulässige Vereinfachungen vorgenommen?
	\item Sind alle Aussagen klar und verständlich formuliert?
	\item Wurde die Literatur zur Kenntnis genommen?
	\item Falls Programme entwickelt wurden: Sind die Quellcodes dokumentiert, die Module und Schnittstellen beschrieben? Gibt es eine Programmbeschreibung?
	\item Wie ist die äußere Form (Layout, Rechtschreibung, Grammatik)?
	\item Ist der Umfang angemessen?
\end{itemize}

George H. Heilmeier hat in einen nach ihm benannten Heilmeier-Katechismus\footnote{\url{https://www.darpa.mil/work-with-us/heilmeier-catechism}, Abruf am 03.11.2023} einige Fragen formuliert, die Orientierung sowohl bei der Erstellung einer Arbeit als auch bei ihrer Bewertung geben können. Sie werden hier in einer leicht angepassten deutschen Übersetzung\footnote{siehe auch \url{https://de.wikipedia.org/wiki/George_H._Heilmeier}, Abruf am 03.11.2023} wiedergegeben:

\begin{itemize}
	\item Was hast Du vor? Beschreibe Dein Vorhaben, möglichst ohne viele Fachbegriffe zu verwenden.
	\item Wie wird es bislang gemacht und was sind die Grenzen der derzeitigen Verfahren oder Methoden?
	\item Was ist neu an Deinem Ansatz und warum denkst Du, dass er erfolgreich sein wird?
	\item Wen kümmert's? Wenn Du erfolgreich bist, was für einen Unterschied wird Deine Lösung machen?
	\item Was sind die Risiken und was die Vorteile Deiner Lösung?
	\item Wie viel wird sie kosten und wie lange wird die Umsetzung dauern?
	\item Woran kann man im Verlauf des Projekts und zu dessen Ende den Erfolg messen?
\end{itemize}

Bei der Bewertung der schriftlichen Ausarbeitung wird von uns zumeist ein Punkteschema verwendet, das sich an \cite{faui2} orientiert, welches am Lehrstuhl für Informatik 2 (Programmiersysteme) der Friedrich-Alexander Universität Erlangen-Nürnberg entwickelt wurde. Eine gekürzte und angepasste Übernahme des Punkteschemas ist im Anhang enthalten.

\section{Referat}

Oft müssen die Ergebnisse der Arbeit in einem Referat vorgestellt werden. Generell gilt: Ein Referat soll die Zuhörerschaft gezielt informieren. Bei der Vorbereitung des Referats sollte deshalb Klarheit darüber bestehen, wieviele Zuhörerinnen und Zuhörer voraussichtlich teilnehmen werden, welches Vorwissen sie haben und mit welchen Erwartungen sie zu dem Referat gekommen sind.

Übersichtliche Folien sind für die bzw. den Vortragenden und die Zuhörerinnen und Zuhörer eine große Unterstützung. Die Folien sollten nummeriert sein, nicht mehr als 4--8 Stichpunkte enthalten, keinen Fließtext und aussagekräftige Abbildungen. Bei Farbfolien sollte man sich auf drei bis vier Farben beschränken, die durchgehend durch die Präsentation verwendet werden. Ansonsten wirken die Folien bunt und unruhig. Schriften ohne oder mit unauffälligen Serifen (z.\,B. Helvetica, Calibri oder Verdana) in 18--20pt eignen sich sehr gut für Vortragsfolien. Es existieren am Arbeitsbereich Templates für Folien, die möglichst verwendet werden sollten.

Als Daumenregel gilt: Folienanzahl $\approx$ Vortragszeit$\,/\,$3 Minuten.

Während des gesamten Vortrags sollte man ins Publikum schauen und nicht zur Wand oder in den Laptop.

Auch das Referat wird nach festgelegten Kriterien beurteilt, die dem Formular im Anhang entnommen werden können.

Ein Kolloquium zur Abschlussarbeit kann auch vor Abgabe der schriftlichen Ausarbeitung stattfinden. Der Vorteil ist, dass ggf. noch Tipps gegeben werden, die in die schriftliche Ausarbeitung einfließen können. Bitte sprechen Sie Ihre Betreuerin bzw. Ihren Betreuer an, wenn Sie Interesse an einem vorgezogenen Kolloquiumstermin haben.

\chapter{Schlussbemerkungen}

Im Internet sind zahlreiche Erfahrungsberichte von (renommierten) Wissenschaftlerinnen und Wissenschaftlern zu finden, die auch bei der Bearbeitung einer Seminar- oder Abschlussarbeit hilfreich sein können. Hier einige wenige Empfehlungen:

\begin{itemize}
	\item Randy Pausch Lecture: Time Management. \\ \url{http://www.youtube.com/watch?v=oTugjssqOT0}
	\item Richard Hamming: You and Your Research. \\ \url{http://www.cs.virginia.edu/~robins/YouAndYourResearch.html}
	\item Nick Feamster: Writing Tips for Academics. \\ \url{http://greatresearch.org/2013/10/11/storytelling-101-writing-tips-for-academics/}
\end{itemize}

Eine besondere Empfehlung ist der Duden-Ratgeber „Wie schreibt man wissenschaftliche Arbeiten?“ von Ulrike Pospiech \cite{Posp2012}, der Informationen und Beispiele zu allen wichtigen Themen bezüglich wissenschaftlicher Texte enthält.

Wissenschaftliches Arbeiten und Schreiben will gelernt sein. Dafür dienen während des Studiums u.a. die Seminararbeiten. Die Abschlussarbeit soll dann zeigen, welche methodischen und fachlichen Fähigkeiten während des Studiums erworben wurden. 

Das Bearbeiten von wissenschaftlichen Fragestellungen während des Studiums schult zudem auch die Entschlussfähigkeit. Wenn Sie sich etwa zwischen zwei Darstellungsvarianten eines Problems entscheiden sollen, grübeln bitte Sie nicht zu lange, sonst landen Sie in einem Deadlock. Dieses Phänomen ist als das Buridansche Paradoxon (auch: Buridans Esel, Grasbüschelproblem) \cite{BuridansAss} bekannt.

Neben einem guten Zeitmanagement, Disziplin und Bereitschaft zur Literaturrecherche ist die Kommunikation mit der Betreuerin bzw. dem Betreuer ein Schlüssel zur erfolgreichen Bearbeitung des Themas.


% =============================Literaturverzeichnis=============================
\begin{raggedright}         % Schaltet Blocksatz ab, erzeugt ein stimmigeres
                            %  Schriftbild im Literaturverzeichnis.
  \printbibliography        % Falls Biblatex verwendet wird.
  \label{sec:literaturverzeichnis}
\end{raggedright}


% ===================================Anhang=====================================
\appendix
\setcounter{figure}{0}
\renewcommand\thetable{A.\arabic{figure}}
\setcounter{table}{0}
\renewcommand\thetable{A.\arabic{table}}

\chapter*{Punktesystem zur Beurteilung}

Bei der Bewertung der schriftlichen Ausarbeitung wird ein Punkteschema verwendet, das am Lehrstuhl für Informatik 2 (Programmiersysteme) der Friedrich-Alexander Universität Erlangen-Nürnberg entwickelt wurde. Die folgende Übersicht ist eine gekürzte und angepasste Übernahme von \cite{faui2}.

\section*{Allgemeine Hinweise}

Die Arbeit wird unter fünf Aspekten einzeln bewertet, die jedoch nicht gleichgewichtig sind. Das verschiedene Gewicht wird dadurch berücksichtigt, dass für die einzelnen Aspekte verschieden hohe Punktzahlen zur Verfügung stehen (siehe Tabelle~\ref{tab:punktzahl}).

\begin{table}[!h]%Tabelle soll hier (!h) erscheinen!
\begin{tabu}{lcrl}
	\toprule
	\multicolumn{3}{l}{Punktzahl} & Aspekt\\
	\midrule
	0 & -- & 6  & Schwierigkeitsgrad\\
	0 & -- & 8  & Schöpferische Originalität\\
	0 & -- & 10 & wissenschaftliche Arbeitstechnik\\
	0 & -- & 4  & Stil\\
	0 & -- & 3  & Äußere Form\\
	\midrule
	0 & -- & 31 & Summe\\
	%\bottomrule
\end{tabu}
\caption{Maximale Punktzahlen pro Aspekt}
\label{tab:punktzahl}
\end{table}

\section*{Notenstufen}

Die Note wird in folgender Weise festgesetzt:
\begin{enumerate}
	\item Arbeiten, bei denen für wissenschaftliche Arbeitstechnik weniger als 4~Punkte oder für die wissenschaftliche Arbeitstechnik, den Stil und die Form zusammen weniger als 8~Punkte vergeben wurden, erhalten die Note~5 (nicht ausreichend, nicht bestanden).
	\item Alle anderen Arbeiten werden nach Tabelle~\ref{tab:noten} benotet.
\end{enumerate}

\begin{table}
\begin{tabu}{lp{6cm}}
	\toprule
	Punktzahl  & Note \\
	\midrule
	31--29 &   1,0 \quad  sehr gut\\
	28--27 &   1,3\\
	\midrule
	26--25 &   1,7\\
	24--23 &   2,0 \quad  gut\\
	22--21 &   2,3\\
	\midrule
	20--19 &   2,7 \\
	18--17 &   3,0 \quad  befriedigend\\
	16--15 &   3,3\\
	\midrule
	14--13 &   3,7\\
	12--11 &   4,0 \quad  ausreichend\\
	\bottomrule
\end{tabu}
\caption{Punkte- und Notenverteilung}
\label{tab:noten}
\end{table}

\section*{1. Schwierigkeitsgrad (0--6)}

Bei der Beurteilung des Schwierigkeitsgrades ist davon auszugehen, ob die Problemstellung mit der durchschnittlichen Ausgangsqualifikation der Bearbeitungsgruppe gelöst werden kann (4~Punkte). Die Beurteilung des Schwierigkeitsgrades einer Arbeit kann erst nach Abschluss erfolgen und umfasst die Prüfung, ob die vorgelegte Fassung die genannten Merkmale auch tatsächlich enthält.

\section*{2. Schöpferische Originalität (0--8)}

Bei der Beurteilung der schöpferischen Originalität ist nicht nur davon auszugehen, inwieweit die Bearbeiterin bzw. der Bearbeiter der Anleitung und Führung durch die Betreuerin bzw. den Betreuer bedarf. Es ist vielmehr selbstverständlich, dass die Bearbeiterin bzw. der Bearbeiter Initiative entwickelt, d.h. aus eigenem Antrieb Schwierigkeiten aufgreift und mit der Betreuerin bzw. dem Betreuer diskutiert (4~Punkte).

\section*{3. Wissenschaftliche Arbeitstechnik (0--10)}

Bei der Beurteilung der wissenschaftlichen Arbeitstechnik ist nicht nur vom Grad der Fehlerfreiheit (formale Richtigkeit der Aussagen und eventueller Programme) auszugehen, die vielmehr als selbstverständlich vorausgesetzt werden muss. Daneben fällt sehr stark das Ausmaß der Selbstkontrolle ins Gewicht, das sich bei formalen Aussagen in der Beweisgründlichkeit, bei Programmen in ausführlichen Tests zeigt. Bezüglich der Programmrichtigkeit darf davon ausgegangen werden, dass bei hinreichend modularem Programmaufbau eine durchdachte (Begründung!) Menge von Testprogrammen genügt (5~Punkte).

\section*{4. Stil (0--4)}

Bei der Beurteilung des Stils ist von der sprachlichen Ausdrucksfähigkeit auszugehen, die sich der Leserin bzw. dem Leser in der vorgelegten Arbeit bietet. Diese zeigt sich insbesondere in der Klarheit und Kürze des Ausdrucks: Auch schwierige Probleme müssen verständlich dargelegt, triviale Zusammenhänge nicht hinter einem formalen Apparat verborgen sein. Die Gedankenführung muss eindeutig sein (2~Punkte).

\section*{5. Äußere Form (0--3)}

Bei der Beurteilung der äußeren Form fällt neben der Sorgfalt der Ausführung, insbesondere der Zeichnungen und Tabellen, die Klarheit der Gliederung und des
Inhaltsverzeichnisses ins Gewicht (2~Punkte).


% ===========================Präsentationsbewertung============================
\chapter*{Formular Präsentationsbewertung}
\begin{sideways}
	\begin{tabu} to 1.2\textwidth {|l|X|X|X|X|X|X|X|X|}
	\multicolumn{6}{l}{\textbf{Präsentationsbewertung}} \\
	\multicolumn{6}{l}{} \\
	\hline
	Datum                                    & & & & & & & \\ \hline
	Name                                     & & & & & & & \\ \hline
	Thema                                    & & & & & & & \\ \hline
	\multicolumn{6}{l}{} \\
	\multicolumn{6}{l}{\textbf{Stil}} \\
	\hline
	Sicheres Auftreten                       & & & & & & & \\ \hline
	Kontakt zum Zuhörer                      & & & & & & & \\ \hline
	Deutliche Sprechweise                    & & & & & & & \\ \hline
	Angemessenes Tempo                       & & & & & & & \\ \hline
	Freies Sprechen                          & & & & & & & \\ \hline
	Einhalten der Zeit                       & & & & & & & \\ \hline
	\multicolumn{6}{l}{} \\
	\multicolumn{6}{l}{\textbf{Inhalt}} \\
	\hline
	Verständlichkeit des Inhalts             & & & & & & & \\ \hline
	Prägnanz                                 & & & & & & & \\ \hline
	Konzept/Gliederung                       & & & & & & & \\ \hline
	Beispiele                                & & & & & & & \\ \hline
	Angemessene Detailtiefe                  & & & & & & & \\ \hline
	\multicolumn{6}{l}{} \\
	\multicolumn{6}{l}{\textbf{Folien/Demo}} \\
	\hline
	Vertrautheit mit Folien/Demo             & & & & & & & \\ \hline
	Verständlichkeit der Folien/Demo         & & & & & & & \\ \hline
	Qualität von Abbildungen                 & & & & & & & \\ \hline
	Nachvollziehbarkeit der Demo             & & & & & & & \\ \hline
	\multicolumn{6}{l}{} \\
	\hline
	Subjektiver Gesamteindruck               & & & & & & & \\ \hline
	\end{tabu}
\end{sideways}


% ===========================Formular Rump Session======================
% \chapter*{Formular „Rump Session“}
\newpage
\thispagestyle{empty}
\vspace*{-3.2cm}
\begin{center}
	Formular „Rump Session“
\end{center}
\vspace*{-0.2cm}

\newsavebox{\rumpsessBox}
\savebox{\rumpsessBox}{%
	%\begin{sffamily}
	\begin{tiny}
	\begin{tabu} to 10cm {lcccc} %\hline
	\multicolumn{5}{l}{} \\[3mm]
	Vortragsnummer/Titel: \dotfill                                               & \tiny ja & \multicolumn{2}{l}{\dotfill} & \tiny nein \\[3pt]
	\textbf{Struktur:} Ich konnte einen „roten Faden“ erkennen.                         & $\Box$ & $\Box$ & $\Box$ & $\Box$ \\[1pt]
	\textbf{Auftreten:} Die/der Vortragende ist selbstsicher aufgetreten.               & $\Box$ & $\Box$ & $\Box$ & $\Box$ \\[1pt]
	\textbf{Kompetenz:} Die/der Vortragende kennt sich mit dem Thema aus.               & $\Box$ & $\Box$ & $\Box$ & $\Box$ \\[1pt]
	\textbf{Didaktik:} Ich habe alles verstanden, was gesagt wurde.                     & $\Box$ & $\Box$ & $\Box$ & $\Box$ \\[1pt]
	\textbf{Niveau:} Ich kann die Kernideen des Vortrag mit eigenen Worten wiedergeben. & $\Box$ & $\Box$ & $\Box$ & $\Box$ \\[1pt]
	\multicolumn{5}{l}{\textbf{Anmerkungen:}} \\[24mm] %\hline
	\end{tabu}
	\end{tiny}
	%\end{sffamily}
}

\hspace*{-2cm}\rule{3mm}{0.5pt}\rule{95mm}{0pt}\rule{0.5pt}{3mm}\rule{95mm}{0pt}\rule{3mm}{0.5pt}\\
\hspace*{-2cm}\usebox{\rumpsessBox}\usebox{\rumpsessBox}\\
\hspace*{-2cm}\rule{3mm}{0.5pt}\rule{95mm}{0pt}\rule{0.5pt}{3mm}\rule{95mm}{0pt}\rule{3mm}{0.5pt}\\
\hspace*{-2cm}\usebox{\rumpsessBox}\usebox{\rumpsessBox}\\
\hspace*{-2cm}\rule{3mm}{0.5pt}\rule{95mm}{0pt}\rule{0.5pt}{3mm}\rule{95mm}{0pt}\rule{3mm}{0.5pt}\\
\hspace*{-2cm}\usebox{\rumpsessBox}\usebox{\rumpsessBox}\\
\hspace*{-2cm}\rule{3mm}{0.5pt}\rule{95mm}{0pt}\rule{0.5pt}{3mm}\rule{95mm}{0pt}\rule{3mm}{0.5pt}\\
\hspace*{-2cm}\usebox{\rumpsessBox}\usebox{\rumpsessBox}\\
\hspace*{-2cm}\rule{3mm}{0.5pt}\rule{95mm}{0pt}\rule{0.5pt}{3mm}\rule{95mm}{0pt}\rule{3mm}{0.5pt}\\
\hspace*{-2cm}\usebox{\rumpsessBox}\usebox{\rumpsessBox}\\
\hspace*{-2cm}\rule{3mm}{0.5pt}\rule{95mm}{0pt}\rule{0.5pt}{3mm}\rule{95mm}{0pt}\rule{3mm}{0.5pt}\\

% ===========================Selbstständigkeitserklärung======================
\chapter*{Eidesstattliche Versicherung} % war: Selbständigkeitserklärung
\vspace{1cm}

\todo[noline]{Bitte verwenden Sie hier in jedem Fall die offizielle von der Prüfungsbehörde vorgegebene Formulierung der Selbständigkeitserklärung.}
%
Hiermit versichere ich an Eides statt, dass ich die vorliegende Arbeit selbstständig verfasst und keine anderen als die angegebenen Hilfsmittel – insbesondere keine im Quellenverzeichnis nicht benannten Internet-Quellen – benutzt habe. Alle Stellen, die wörtlich oder sinngemäß aus Veröffentlichungen entnommen wurden, sind als solche kenntlich gemacht. Ich versichere weiterhin, dass ich die Arbeit vorher nicht in einem anderen Prüfungsverfahren eingereicht habe und die eingereichte schriftliche Fassung der auf dem elektronischen Speichermedium entspricht.

Ggf. streichen: Ich bin damit einverstanden, dass meine Abschlussarbeit in den Bestand der Fachbereichsbibliothek eingestellt wird.

\makeatletter
Hamburg, den {\@date}
\makeatother

\vspace{2cm}
\rule{6cm}{0.25pt}\\
\makeatletter
{\@author} \par
\makeatother

% ================================Deckblatt-Muster Abschlussarbeit==============================
\newpage
\thispagestyle{empty}
% \addcontentsline{toc}{chapter}{Muster des Deckblatts}
%
% In der Ausarbeitung bitte den Bereich ab \begin{titlepage} bis \end{titlepage}
%  vorne durch den nachfolgenden Codeabschnitt ersetzen:
% ====> Von hier ...
\begin{titlepage}% {{{
% \includegraphics[width=6.8cm]{../pic/up-uhh-logo-u-2010-u-farbe-u-rgb.pdf}
\mbox{\parbox[t][1.75cm][b]{2.2cm}{\uhhlogo}}
\begin{center}\Large
	\vfill
	Masterarbeit
	\vfill
	\makeatletter
	{\Large\textsf{\textbf{\@title}}\par}
	\makeatother
	\vfill
	vorgelegt von
	\par\bigskip
	\makeatletter
	{\@author} \par
	\makeatother
	Matrikelnummer 1234567 \par
	Studiengang Informatik
	\vfill
	MIN-Fakultät \par
	Fachbereich Informatik
	\vfill
	\makeatletter
	eingereicht am {\@date}
	\makeatother
	\vfill
	Betreuerin: Erika Musterfrau, M.\,Sc. \todo{Todos im Text und Fragen an den Betreuer sind in dieser Form dargestellt}\par
	Erstgutachter: Prof. Dr.-Ing. Hannes Federrath \par
	Zweitgutachter: N.N.
\end{center}
\ifoptionfinal{}{
\begin{tikzpicture}[remember picture, overlay]
	\node[draw, red, font=\ttfamily\bfseries\Large, xshift=30mm, yshift=238mm, rotate=340, text centered, text width=6cm, very thick, rounded corners=4mm] at (current page.south) {Entwurf vom \today};
\end{tikzpicture}
% ====> Delete me
\begin{tikzpicture}[overlay]
	\node[draw, blue, font=\sffamily\Large, xshift=0mm, yshift=242mm, rotate=0, text centered, rounded corners=1mm] at (current page.south) {Muster des Deckblatts für Abschlussarbeiten};
\end{tikzpicture}
% <==== /Delete me
}
\end{titlepage}% }}}
% <==== ... bis hierher.

% ================================Deckblatt-Muster Seminararbeit==============================
\newpage
\thispagestyle{empty}
% \addcontentsline{toc}{chapter}{Muster des Deckblatts}
%
% In der Ausarbeitung bitte den Bereich ab \begin{titlepage} bis \end{titlepage}
%  vorne durch den nachfolgenden Codeabschnitt ersetzen:
% ====> Von hier ...
\begin{titlepage}% {{{
% \includegraphics[width=6.8cm]{../pic/up-uhh-logo-u-2010-u-farbe-u-rgb.pdf}
\mbox{\parbox[t][1.75cm][b]{2.2cm}{\uhhlogo}}
\begin{center}\Large
	\vfill
	Seminararbeit
	\vfill
	\makeatletter
	{\Large\textsf{\textbf{\@title}}\par}
	\makeatother
	\vfill
	vorgelegt von
	\par\bigskip
	\begin{tabu}{p{0.5\textwidth}p{0.5\textwidth}}
	\centering Martina Mustermann     & \centering Max Mustermann \\[1ex]
	\centering Matrikelnummer 1234567 & \centering Matrikelnummer 7654321 \\
	\centering Studiengang Informatik & \centering Studiengang Informatik \\
	\end{tabu}
	\vfill
	MIN-Fakultät \par
	Fachbereich Informatik
	\vfill
	\makeatletter
	eingereicht am {\@date}
	\makeatother
	\vfill
	Betreuer: Heinz Mustermann, M.\,Sc.
\end{center}
\ifoptionfinal{}{
\begin{tikzpicture}[remember picture, overlay]
	\node[draw, red, font=\ttfamily\bfseries\Large, xshift=30mm, yshift=238mm, rotate=340, text centered, text width=6cm, very thick, rounded corners=4mm] at (current page.south) {Entwurf vom \today};
\end{tikzpicture}
% ====> Delete me
\begin{tikzpicture}[overlay]
	\node[draw, blue, font=\sffamily\Large, xshift=0mm, yshift=242mm, rotate=0, text centered, rounded corners=1mm] at (current page.south) {Muster des Deckblatts für Seminararbeiten};
\end{tikzpicture}
% <==== /Delete me
}
\end{titlepage}% }}}
% <==== ... bis hierher.


% ================================Literaturliste-Muster==============================
\newpage
\thispagestyle{empty}
\label{sec:literaturliste}
\par\textbf{\textsf{Thema:}} Privacy Enhancing Technologies zum Schutz von Kommunikationsbeziehungen
\par\textbf{\textsf{Bearbeiter:}} Eva Musterfrau, Heinz Mustermann
\par\textbf{\textsf{Datum:}} \today
\bigskip
% ====> Delete me
\begin{tikzpicture}[overlay]
	\node[draw, blue, font=\sffamily\Large, xshift=70mm, yshift=0mm, rounded corners=1mm]{Muster der Literaturliste};
\end{tikzpicture}
% <==== /Delete me
\par\textbf{\Large\textsf{Literaturliste}}

% ----- Nachfolgend eine händisch gesetzte Literaturliste, die sich exakt an die Syntax im Abschnitt \ref{sec:literaturhowto} hält. Wir nutzen diese aber hier nicht, sondern lassen BibLaTeX die Einträge formatieren.
\iffalse
David Chaum: Untraceable Electronic Mail, Return Addresses, and Digital Pseudonyms. Communications of the ACM 24/2 (1981) 84--88.

David Chaum: The Dining Cryptographers Problem: Unconditional Sender and Recipient Untraceability. Journal of Cryptology 1/1 (1988) 65--75.

David Goldschlag, Michael Reed, Paul Syverson: Onion Routing for Anonymous and Private Internet Connections. Communications of the ACM 42/2 (1999) 39--41.

Andreas Pfitzmann: Diensteintegrierende Kommunikationsnetze mit teilnehmerüberprüfbarem Datenschutz. IFB 234, Springer-Verlag, Berlin 1990.

Wei Wang, Mehul Motani, Vikram Srinivasan: Dependent link padding algorithms for low latency anonymity systems. Proc. 15th ACM conference on Computer and communications security. ACM, 2008, 323--332.
\fi

% ----- Nachfolgend die Ausgabe unter Verwendung von BibLaTeX. Die Formatierung übernimmt BibLaTeX. Dadurch wird es zu Abweichungen von der vorgegebenen Syntax kommen. Dies ist tolerabel, da es i.W. auf Einheitlichkeit ankommt, nicht auf eine dogmatische Einhaltung der Syntax.
\fullcite{Chau81}

\fullcite{Chau88}

\fullcite{GoRS99}

\fullcite{Pfit90}

\fullcite{WaMS2008}

% ================================Todo list==============================
\listoftodos
% \todototoc

\end{document}
