\documentclass{article}

\usepackage{../.template/summary}

\subject{Privacy by Design}
\semester{Winter 2024}
\author{Leopold Lemmermann}

\usepackage{pdfpages}

\begin{document}\createtitle

\includepdf[pages={23-24, 27-39, 41-57, 59-124}, nup=4x9]{slides.pdf}
\part{Privacy Enhancing Technology}
\section{In Theory}

\subsection{Privacy Goals and Protection Models}
\begin{itemize}
    \item \textbf{Confidentiality:} Prevent unauthorized access to sensitive information.
    \item \textbf{Integrity:} Protect against unauthorized modification of information.
    \item \textbf{Availability:} Ensure authorized access to information when needed.
\end{itemize}

\subsection{Confidentiality Mechanisms}
\begin{itemize}
    \item \textbf{Encryption:}
    \begin{itemize}
        \item Symmetric (e.g., AES): Uses a shared key for both encryption and decryption.
        \item Asymmetric (e.g., RSA, ECC): Employs public-private key pairs for secure communication.
    \end{itemize}
    \item \textbf{Anonymization:}
    \begin{itemize}
        \item Techniques such as k-anonymity, differential privacy, and pseudonymization to obscure user data.
    \end{itemize}
    \item \textbf{Decentralized Networks:}
    \begin{itemize}
        \item Leverage architectures like DC networks (Dining Cryptographers) and mix networks to ensure anonymity in communication.
    \end{itemize}
\end{itemize}

\subsection{DC Networks (Dining Cryptographers)}
\begin{itemize}
    \item \textbf{Principle:}
    \begin{itemize}
        \item Participants (cryptographers) communicate in such a way that a message's sender remains anonymous unless they choose to reveal themselves.
        \item Each pair of participants shares a secret bit, and all participants cooperate to hide the identity of the message sender.
    \end{itemize}
    \item \textbf{Use Cases:}
    \begin{itemize}
        \item Anonymous broadcasting, electronic voting.
    \end{itemize}
    \item \textbf{Advantages:}
    \begin{itemize}
        \item Provable anonymity.
    \end{itemize}
    \item \textbf{Limitations:}
    \begin{itemize}
        \item Scalability issues due to increasing communication overhead with more participants.
    \end{itemize}
\end{itemize}

\subsection{Blind Messaging Services}
\begin{itemize}
    \item \textbf{Principle:}
    \begin{itemize}
        \item Users send messages through a series of anonymizing layers, where each layer hides specific metadata (e.g., sender's IP address).
        \item Common implementations include Tor (The Onion Router), which routes traffic through multiple relays to anonymize users.
    \end{itemize}
    \item \textbf{Use Cases:}
    \begin{itemize}
        \item Anonymous browsing, secure whistleblowing, and bypassing censorship.
    \end{itemize}
    \item \textbf{Advantages:}
    \begin{itemize}
        \item Scalable and widely adopted.
        \item Resilient against traffic analysis.
    \end{itemize}
    \item \textbf{Limitations:}
    \begin{itemize}
        \item Vulnerable to global adversaries monitoring the entire network.
        \item Latency introduced due to multi-hop routing.
    \end{itemize}
\end{itemize}

\subsection{Mix Networks}
\begin{itemize}
    \item \textbf{Principle:}
    \begin{itemize}
        \item Messages are shuffled and encrypted at each node (mix) in a network, ensuring unlinkability between sender and recipient.
        \item Each mix node reorders, decrypts, and forwards messages in batches.
    \end{itemize}
    \item \textbf{Use Cases:}
    \begin{itemize}
        \item Email anonymization (e.g., Mixminion, Mixmaster), anonymous payments.
    \end{itemize}
    \item \textbf{Advantages:}
    \begin{itemize}
        \item Strong resistance to traffic analysis.
        \item Can provide anonymity for both sender and recipient.
    \end{itemize}
    \item \textbf{Limitations:}
    \begin{itemize}
        \item Increased latency due to batching and processing.
        \item Vulnerable to active attacks if a mix node is compromised.
    \end{itemize}
\end{itemize}

\subsection{Integrity Mechanisms}
\begin{itemize}
    \item \textbf{Hash Functions:}
    \begin{itemize}
        \item Fixed-size digest for verifying data integrity (e.g., SHA-256).
    \end{itemize}
    \item \textbf{Digital Signatures:}
    \begin{itemize}
        \item Combines hashing with asymmetric encryption for authenticity and non-repudiation.
    \end{itemize}
\end{itemize}

\subsection{Availability Mechanisms}
\begin{itemize}
    \item \textbf{Redundancy:}
    \begin{itemize}
        \item Use of backup servers and data replication.
    \end{itemize}
    \item \textbf{Load Balancing:}
    \begin{itemize}
        \item Distributes network traffic to prevent overload.
    \end{itemize}
    \item \textbf{DDoS Mitigation:}
    \begin{itemize}
        \item Techniques like rate limiting and anomaly detection to block malicious traffic.
    \end{itemize}
\end{itemize}



\includepdf[pages={126-161}, nup=4x9]{slides.pdf}
\section{In Practice}

\subsection{Privacy Enhancing Technologies (PETs)}
\begin{itemize}
    \item \textbf{Objective:} Protect user anonymity and prevent unauthorized tracking in real-world applications.
    \item \textbf{Examples of PETs:}
    \begin{itemize}
        \item \textbf{Anonymity Networks:} Tor, I2P for secure and anonymous communication.
        \item \textbf{Encryption:} End-to-end encryption in messaging apps (e.g., Signal, WhatsApp).
        \item \textbf{Pseudonymization:} Use of alternate identifiers to reduce linkage to real identities.
    \end{itemize}
\end{itemize}

\subsection{Traffic Pseudonymization}
\begin{itemize}
    \item \textbf{IPv6 Privacy Extensions:} Randomize IPv6 addresses to prevent tracking over time.
    \item \textbf{Pseudonym Rotation:} Periodically change pseudonyms to reduce linkability between user actions.
    \item \textbf{Limitations:}
    \begin{itemize}
        \item Limited effectiveness in preventing advanced traffic analysis.
        \item Increased complexity in managing pseudonyms.
    \end{itemize}
\end{itemize}

\subsection{Anonymity Networks in Practice}
\begin{itemize}
    \item \textbf{Tor:}
    \begin{itemize}
        \item Relays user traffic through multiple nodes to obscure origin and destination.
        \item Vulnerable to global adversaries and end-node monitoring.
    \end{itemize}
    \item \textbf{I2P:}
    \begin{itemize}
        \item Optimized for anonymous communication within the network.
        \item Use cases: Peer-to-peer file sharing, messaging, and hosting services.
    \end{itemize}
    \item \textbf{Mix Networks:}
    \begin{itemize}
        \item Provide stronger anonymity by shuffling and re-encrypting messages at each node.
        \item Used in email anonymization and anonymous transactions.
    \end{itemize}
\end{itemize}

\subsection{Censorship Resistance}
\begin{itemize}
    \item \textbf{Goals:} Ensure access to information and communication in censored environments.
    \item \textbf{Techniques:}
    \begin{itemize}
        \item Decentralized networks (e.g., Freenet, IPFS).
        \item Bridge nodes in Tor to bypass censorship firewalls.
        \item Steganography to disguise traffic as normal communication.
    \end{itemize}
    \item \textbf{Challenges:}
    \begin{itemize}
        \item Sophisticated censorship tools (e.g., deep packet inspection).
        \item Balancing usability with security under surveillance.
    \end{itemize}
\end{itemize}

\subsection{Criminal Misuse and Data Retention}
\begin{itemize}
    \item \textbf{Misuse of Anonymity:}
    \begin{itemize}
        \item Use of anonymity networks for illicit activities (e.g., illegal marketplaces, malware distribution).
        \item Challenges for law enforcement in tracking activities without violating privacy.
    \end{itemize}
    \item \textbf{Data Retention Policies:}
    \begin{itemize}
        \item Mandated storage of communication metadata for investigative purposes.
        \item Conflict with privacy-focused systems and anonymization techniques.
    \end{itemize}
    \item \textbf{Legal Balance:}
    \begin{itemize}
        \item Protecting civil liberties while enabling effective law enforcement.
    \end{itemize}
\end{itemize}

\subsection{Traffic Analysis and Fingerprinting}
\begin{itemize}
    \item \textbf{Techniques:}
    \begin{itemize}
        \item Identifying patterns in encrypted traffic (e.g., packet size, timing).
        \item Inferring user behavior or accessed resources from metadata.
    \end{itemize}
    \item \textbf{Defenses:}
    \begin{itemize}
        \item Padding and random delays to obscure traffic patterns.
        \item Use of obfuscation proxies to disguise protocol characteristics.
    \end{itemize}
    \item \textbf{Limitations:}
    \begin{itemize}
        \item Increased latency and overhead.
        \item Limited effectiveness against global adversaries.
    \end{itemize}
\end{itemize}



\includepdf[pages={161-216}, nup=4x9]{slides.pdf}
\part{Location Privacy}
\section{Location Privacy \& Mobile Networks}

\subsection{Overview}
\begin{itemize}
    \item \textbf{Objective:} Protect users' location information from being tracked or monitored by adversaries or unauthorized entities.
    \item \textbf{Challenges:}
    \begin{itemize}
        \item Increasing reliance on location-based services.
        \item Potential abuse by service providers, adversaries, or malicious third parties.
    \end{itemize}
\end{itemize}

\subsection{Systematic Protection Methods}
\begin{enumerate}
    \item \textbf{Broadcasting:} Randomized transmission of location data.
    \item \textbf{Group Pseudonyms:} Shared identifiers among a group of users to obscure individual location.
    \item \textbf{Temporal Pseudonyms:} Changing pseudonyms periodically to reduce traceability.
    \item \textbf{Mixing Networks:} Use of cascaded nodes to anonymize the source and destination of location data.
\end{enumerate}

\subsection{Broadcasting Techniques}
\begin{itemize}
    \item \textbf{Implicit Addressing:}
    \begin{itemize}
        \item Targets devices indirectly by transmitting generalized location updates.
        \item Performance: Reduces bandwidth usage but may decrease precision.
    \end{itemize}
    \item \textbf{Variable Broadcasts:}
    \begin{itemize}
        \item Adapts broadcast size dynamically to obscure exact location.
        \item Useful in areas with sparse populations.
    \end{itemize}
\end{itemize}

\subsection{Group and Temporal Pseudonyms}
\begin{itemize}
    \item \textbf{Group Pseudonyms:}
    \begin{itemize}
        \item Users within a region share a pseudonym to make identification harder.
        \item Reduces linkability across updates.
    \end{itemize}
    \item \textbf{Temporal Pseudonyms:}
    \begin{itemize}
        \item Pseudonyms change over time to prevent long-term tracking.
        \item Requires synchronization with network systems for functionality.
    \end{itemize}
\end{itemize}

\subsection{Mobile Communication-Mixing}
\begin{itemize}
    \item \textbf{Centralized Approach:}
    \begin{itemize}
        \item A trusted third party manages pseudonym and location updates.
        \item High scalability but susceptible to single points of failure.
    \end{itemize}
    \item \textbf{Decentralized Approach:}
    \begin{itemize}
        \item Users and mix cascades collaborate to anonymize location updates.
        \item Improves privacy but increases coordination complexity.
    \end{itemize}
\end{itemize}

\subsection{Anonymity Techniques in Mobile Networks}
\begin{itemize}
    \item \textbf{Mix Networks:}
    \begin{itemize}
        \item Obscure the link between senders and receivers using layered encryption.
        \item Suitable for large-scale deployments but introduces latency.
    \end{itemize}
    \item \textbf{Blind Signatures:}
    \begin{itemize}
        \item Allows authentication without revealing identities.
        \item Useful in anonymous mutual authentication protocols.
    \end{itemize}
\end{itemize}

\subsection{Mobile Internet Protocol (IP)}
\begin{itemize}
    \item \textbf{Location Hiding:}
    \begin{itemize}
        \item Use of proxies or agents to hide the user's exact location.
        \item Common in applications with stringent privacy requirements.
    \end{itemize}
    \item \textbf{Challenges:}
    \begin{itemize}
        \item Higher latency due to intermediate nodes.
        \item Trade-off between performance and privacy.
    \end{itemize}
\end{itemize}

\subsection{Performance Analysis of Privacy Methods}
\begin{itemize}
    \item \textbf{Trust Levels:}
    \begin{itemize}
        \item Varying levels of trust required in centralized vs. decentralized systems.
        \item Broadcast and mixing networks minimize reliance on trusted third parties.
    \end{itemize}
    \item \textbf{Efficiency Loss:}
    \begin{itemize}
        \item Privacy-preserving methods incur bandwidth and processing overhead (1-10\%).
    \end{itemize}
    \item \textbf{Protection Metrics:}
    \begin{itemize}
        \item Degree of protection against tracking and profiling varies by method.
        \item Methods like mobile communication-mixing outperform traditional GSM in anonymity.
    \end{itemize}
\end{itemize}

\end{document}