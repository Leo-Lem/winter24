\documentclass{article}

\usepackage[]{../.template/xrcise}

\subject{Privacy by Design}
\semester{Winter 2024}
\author{Leopold Lemmermann}

\begin{document}\createtitle

\section{Catalogue}

% TODO



\section[2024]{Mock exam}

% TODO



\section[2020]{1st exam}
% 1. Broadcast
% • Wer wird geschützt und welches Schutzziel wird verfolgt?
% • Nenne zwei Anwendungsbeispiele (abgesehen von Zeitung/TV/Radio)
% • Professorin möchte über eine Zeitschrift auf Feedback von Studierenden Antworten und die Identität dieser
% schützen. Anweisungen ähnlich wie: Neben dem Feedback soll eine Zufallszahl zwischen 0 und 100 angegeben
% werden. Antworten werden im Bezug auf dieses Pseudonym gegeben. Welche der beiden Arten der impliziten
% Adressierung wurde hier genutzt? Was sind mögliche Probleme in Bezug auf speziell diese Auswahl an
% Pseudonymen?
% • Der Professorin gefällt nicht, dass die Antworten auf das Feedback von jedem gelesen werden können (und
% nicht nur von der entsprechenden Person). Wir könnte das Problem gelöst werden? Schreibe einen Text mit
% Anweisungen an die Studierenden und Begründe die Notwendigkeit der einzelnen Komponenten.
% 2. IPv6-Präfixe
% Zur Einleitung wurden Präfix-Sharing, -Hopping und -Boquets beschrieben
% • Warum genügt es nicht verschiedene Adressen unter einem Präfix zu verwenden?
% • Was sind die Schwächen von Präfix-Sharing bei einem gleich bleibendem Präfix und einer kleinen Nutzergruppe?
% • Was sind die Schwächen von Präfix-Boquets, wenn unterschiedliche Präfixe für verschiedene Ziele aber nicht
% pro Verbindung genutzt werden?
% • Was ist das Problem beim Präfix-Hopping, wenn sich während einer bestehenden Verbindung die IP-Adresse
% ändert?
% • Wie kann man dieses Problem beim Präfix-Hopping lösen?
% • Wie kann man Benutzer abgesehen von der Adressierung tracken? Nenne 3 Beispiele.
% 3. Mixe
% • Angreifermodell angeben
% • Wie kann ein Mix-Netz von einem starken Angreifer (z.B. Geheimdienst) angegriffen werden? Nenne 2
% Beispiele.
% • Es wird ein Mix-Netz angeboten von nur einem Betreiber, welcher damit wirbt, dass 20 Mixe hinter einander
% verwendet werden und für höhere Performance nur 10 Teilnehmer pro Mix zugeordnet werden. Bewerte die
% Sicherheit des vorgestellten Produkts mit Begründung.
% 4. DC-Netze
% Gegeben waren 4 lokale Summen (A, B, C, D), ein Schlüsselgraph kAB , kBC , kCD und der Schlüssel kBC . Wir sind
% Teilnehmer B und haben eine Leernachricht gesendet.
% • Berechne die gesendete Nachricht N . Skizziere deinen Rechenweg.
% • Grenze ein wer die Nachricht gesendet haben kann. Skizziere deinen Rechenweg.
% • Was muss passieren, damit B weniger Aussagen treffen kann?

\end{document}