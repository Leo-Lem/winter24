\documentclass{article}

\usepackage[]{../.template/xrcise}

\subject{Privacy by Design}
\semester{Winter 2024}
\author{Leopold Lemmermann}

\begin{document}\createtitle

\section{Catalogue}

\setcounter{subsection}{34}
\begin{exercise}{Informationsgewinn aufgrund von Kontextinformationen}
  Unmittelbar vor der Durchführung eines militärischen Kampfeinsatzes steigt die Anzahl der Essenslieferungen (Pizza, Burger, Croques, Sushi etc.), die an die Adresse des Verteidigungsministeriums geliefert werden, massiv an. Diskutieren Sie, ob das System durch einen passiven oder einen aktiven Angreifer verwundbar ist und welche Schutzziele dadurch bedroht werden. Für welche Angriffsart bzw. Angriffsarten ist das System anfällig? Was wären mögliche Gegenmaßnahmen?

  \begin{solution}
    % TODO
  \end{solution}
\end{exercise}

\begin{exercise}{Angriff über WLAN-Namen}
  Viele öffentliche WLAN-Access-Points (APs) teilen allen Stationen (Clients) in Reichweite ihren WLAN-Namen (SSID) mit. Der Nutzer einer Station wählt aus der Liste aller Stationen anhand der SSID das Netz, mit dem er sich verbinden will. Damit der Nutzer diese Auswahl nicht jedes Mal wieder treffen muss, verbinden sich die Stationen automatisch mit Access Points, mit denen sie bereits verbunden waren. Hierzu wird eine Liste der bereits bekannten SSIDs auf der Station aufbewahrt. Stehen an einem Ort mehrere APs mit derselben SSID zur Auswahl, verbindet sich eine Station i. d. R. mit dem AP mit der größten Signalstärke.
  \begin{enumerate}
    \item Diskutieren Sie, ob das System durch einen passiven oder einen aktiven Angreifer verwundbar ist.
    \item Welche Schutzziele werden dadurch bedroht?
    \item Was wären mögliche Gegenmaßnahmen?
  \end{enumerate}

  \begin{solution}
    \begin{enumerate}
      \item The system is vulnerable to a passive attacker.
      \item The protection goals that are threatened are confidentiality and integrity.
      \item Possible countermeasures are to use a VPN connection and to disable automatic connection to unknown networks.
    \end{enumerate}
  \end{solution}
\end{exercise}

\setcounter{subsection}{167}
\begin{exercise}{Verschlüsselte PIN-Übermittlung}
  Der Nutzer eines Laptops soll sich gegenüber einem Server im Heimnetz durch eine PIN authentisieren. Um die vom Nutzer am Laptop eingegebene PIN nicht mitlesen zu können, wird sie vor der Übermittlung mit dem öffentlichen RSA-Schlüssel des Servers deterministisch verschlüsselt.
  \begin{enumerate}
    \item Analysieren Sie die (Un-)Sicherheit eines solchen Protokolls gegenüber einem Angreifer auf der Kommunikationsstrecke.
    \item Welche Schutzmaßnahmen können Nutzer und Server umsetzen, um die Sicherheit des Protokolls zu erhöhen?
  \end{enumerate}

  \begin{solution}
    % TODO
  \end{solution}
\end{exercise}

\setcounter{subsection}{188}
\begin{exercise}{Datenschutzgerechte Pseudonyme}
  Es ist denkbar, dass eine Zertifizierungsstelle Z den Public Key als Pseudonym einer ihr bekannten Person P zertifiziert. Allerdings kann die Zertifizierungsstelle dann nicht nur im Streitfall das Pseudonym aufdecken, sondern auch vollständig nachvollziehen, welche Geschäfte P gemacht hat (bzw. welche Signaturen P geleistet hat), da sie den Public Key mit der Identität von P verketten kann.

  Denken Sie sich eine Variante aus, bei der eine einzige Zertifizierungsstelle hierzu nicht mehr in der Lage ist, im Streitfall aber trotzdem eine Aufdeckung des Pseudonyms möglich ist.

  \begin{solution}
    % TODO
  \end{solution}
\end{exercise}

\setcounter{subsection}{2220}
\begin{exercise}{Speicherung des Aufenthaltsortes}
  \begin{enumerate}
    \item Welchen Zweck erfüllt die Speicherung des Aufenthaltsortes im GSM? Warum werden beim WLAN keine Aufenthaltsorte gespeichert?
    \item Warum werden im GSM zur Speicherung des Aufenthaltsortes zwei getrennte Datenbanken (HLR, VLR) verwendet?
    \item Bitte diskutieren Sie den erreichbaren Schutz des Aufenthaltsortes gegenüber Insider-Angriffen (z.B. durch Mitarbeiter des Netzbetreibers) und Outsidern.
    \item Inwieweit trägt die TMSI zum Schutz des Aufenthaltsortes bei?
  \end{enumerate}

  \begin{solution}
    % TODO
  \end{solution}
\end{exercise}

\setcounter{subsection}{2223}
\begin{exercise}{Kritik an GSM}
  Trotz der vorhandenen Sicherheitsfunktionen bietet GSM nur beschränkte Sicherheit hinsichtlich Vertraulichkeit. Warum?

  \begin{solution}
    % TODO
  \end{solution}
\end{exercise}

\setcounter{subsection}{2229}
\begin{exercise}{Broadcast}
  Durch Verteilung (Broadcast) von Verbindungswünschen im gesamten Versorgungsgebiet eines Kommunikationsnetzes (z.B. Mobilfunknetzes) kann auf das Speichern von Aufenthaltsdaten sowie Routinginformation verzichtet werden.
  \begin{enumerate}
    \item Erläutern Sie den Unterschied von offenen impliziten und verdeckten impliziten Adressen.
    \item Könnte man auch die explizite Adressierung verwenden, um einem mobilen Teilnehmer eine nicht notwendigerweise vertrauliche Nachricht (z.B. eine SMS) zu senden, ohne ihn lokalisieren zu können? Begründen Sie Ihre Antwort.
    \item Ist bei der variablen impliziten Adressierung ein Satellit zur Verteilung einsetzbar? Begründung!
  \end{enumerate}

  \begin{solution}
    % TODO
  \end{solution}
\end{exercise}

\begin{exercise}{Kollisionsabstand offener impliziter Adressen}
  Die Länge einer impliziten Adresse sei 32 Bit. Es sollen 100.000 Teilnehmer versorgt werden. Jedem Teilnehmer sind zu jedem Zeitpunkt 50 Adressen zugeordnet. Angenommen, jeder Teilnehmer erhält im Durchschnitt 1 Nachricht pro Stunde.
  \begin{enumerate}
    \item Wie groß ist der Kollisionsabstand, d.h. wie lange dauert es im Mittel, bis ein ganz bestimmter Teilnehmer einen Fehlalarm (Nachricht eigentlich für einen anderen Teilnehmer bestimmt) erhält?
    \item Der mittlere Kollisionsabstand soll mindestens 3 Jahre betragen. Wie lang muss die offene implizite Adresse mindestens sein?
  \end{enumerate}

  \begin{solution}
    % TODO
  \end{solution}
\end{exercise}

\setcounter{subsection}{231}
\begin{exercise}{Blind-Messsage-Service}
  \begin{enumerate}
    \item Welche Schutzziele erfüllt der Blind-Message-Service?
    \item Warum müssen beim Blind-Message-Service die Anfragevektoren und die Server-Antworten verschlüsselt übertragen werden?
    \item Warum genügt beim modifizierten Blind-Message-Service die Verschlüsselung der Anfragevektoren?
    \item Welche Verfahren können Sie alternativ zum Blind-Message-Service einsetzen, wenn Bandbreite keine Rolle spielt?
  \end{enumerate}

  \begin{solution}
    % TODO
  \end{solution}
\end{exercise}

\setcounter{subsection}{235}
\begin{exercise}{Modifikation von Chaum'schen Mixen}
  Chaum'sche Mixe nutzen eine Datenbank, in der alle eingehenden Nachrichten gespeichert werden.
  \begin{enumerate}
    \item Bitte erläutern Sie den Angriff, der durch die Datenbank verhindert wird.
    \item Unter welcher Bedingung kann die Datenbank gelöscht werden?
    \item Wie lässt sich das Verfahren umbauen, sodass Sie unter Beibehaltung des Angreifermodells die Speichereffizienz steigern können?
  \end{enumerate}

  \begin{solution}
    % TODO
  \end{solution}
\end{exercise}

\begin{exercise}{DC-Netz}
  \begin{enumerate}
    \item Der (entartete) Schlüsselgraph eines DC-Netzes für die Teilnehmer A, B, C und D sei (A-B-C-D). B greift nun A, C und D an. Beschreiben Sie den erreichbaren Schutz für A, C und D. Wie sind Sie bei Ihrer Analyse vorgegangen?
  \end{enumerate}

  \begin{solution}
    % TODO
  \end{solution}
\end{exercise}

\setcounter{subsection}{238}
\begin{exercise}{Überwachung im Internet}
  \begin{enumerate}
    \item Erläutern Sie eine mögliche Vorgehensweise, mit der ein starker Angreifer (z.B. ein Nachrichtendienst) die IP-Adressen der Nutzer herausfinden kann, die eine bestimmte Webseite über ein mixbasiertes Anonymisierungssystem (z.B. Tor) abrufen.
    \item Erläutern Sie eine mögliche Vorgehensweise, mit der ein starker Angreifer (z.B. ein Nachrichtendienst) ermitteln kann, welche Webseiten ein bestimmter Nutzer über ein mixbasiertes Anonymisierungssystem abgerufen hat. Hinweis: Beschreiben Sie eine andere Vorgehensweise als in Teilaufgabe a), auch wenn Ihre für Teilaufgabe a) gewählte Vorgehensweise für Teilaufgabe b) geeignet ist.
    \item Erläutern Sie drei Techniken, mit denen ein Webserver einen Nutzer bei seinem zweiten (späteren) Besuch wiedererkennen kann.
  \end{enumerate}

  \begin{solution}
    % TODO
  \end{solution}
\end{exercise}

\setcounter{subsection}{2310}
\begin{exercise}{Tagging-Angriff auf anonyme Mixe und Kanäle}
  \begin{enumerate}
    \item Was ist ein Tagging-Angriff auf anonyme Kanäle?
    \item Wie verhindert man den Tagging-Angriff auf anonyme Kanäle?
  \end{enumerate}

  \begin{solution}
    % TODO
  \end{solution}
\end{exercise}

\begin{exercise}{Tracking auf der Anwendungsschicht}
  \begin{enumerate}
    \item Nennen Sie zwei Schutzziele, die beim Tracking auf der Anwendungsebene gebrochen werden.
    \item Was unterscheidet Tracking auf der Anwendungsebene von Tracking auf der Netzebene?
    \item Nennen Sie drei grundsätzliche Mechanismen, mit denen man Tracking auf der Anwendungsebene verhindern bzw. erschweren kann. Erläutern Sie für jeden Mechanismus, wie das Tracking erschwert wird und vor welchem Angreifer, und nennen Sie jeweils ein konkretes Beispiel, bei dem der Mechanismus umgesetzt wird.
  \end{enumerate}

  \begin{solution}
    % TODO
  \end{solution}
\end{exercise}

\begin{exercise}{Strafverfolgung}
  Die Polizei hat sich Zugriff auf einen Webserver verschafft, der zur unbefugten Verbreitung von urheberrechtlich geschütztem Material verwendet wird. Anhand des dort aufgezeichneten Access-Log sollen nun die Nutzer des Servers ausfindig gemacht werden. Bei der Ermittlung von Tatverdächtigen stößt man schnell auf Probleme: Die Zugriffe, die von der IP-Adresse A ausgingen, stammen von einem Kleinbetrieb, in dem es fünf Nutzer gibt. Der Inhaber des Anschlusses stellt der Polizei ein Protokoll des DSL-Routers zur Verfügung, in dem die Aktivitäten der einzelnen internen Nutzer anhand ihrer IP-Adressen aufgezeichnet wurden. Messungen ergeben, dass die Latenz des Kommunikationsnetzes zwischen DSL-Router und Server maximal 0,5 Sekunden beträgt. Auf Basis der verfügbaren Daten (Access-Log, IP-Adresse A, Protokoll des DSL-Routers und Latenz) soll nun die interne IP-Adresse ermittelt werden, von der die Zugriffe ausgingen.

  Bitte erläutern Sie Ihr Vorgehen im Detail. Unter welcher Begrifflichkeit lassen sich derartige Angriffe zusammenfassen?

  \begin{solution}
    % TODO
  \end{solution}
\end{exercise}



\section[2024]{Mock exam}

% TODO



\section[2020]{1st exam}
\begin{exercise}{Broadcast}
  \begin{enumerate}
    \item Who is protected and what protection goal is pursued?
    \item Name two examples of applications (apart from newspaper/TV/radio).
    \item A professor wants to receive feedback from students via a magazine and protect their identity. Instructions similar to: In addition to the feedback, a random number between 0 and 100 should be given. Answers are given in relation to this pseudonym. Which of the two types of implicit addressing was used here? What are possible problems with regard to this selection of pseudonyms?
    \item The professor does not like that the answers to the feedback can be read by everyone (and not only by the corresponding person). How could the problem be solved? Write a text with instructions to the students and justify the necessity of the individual components.
  \end{enumerate}

  \begin{solution}
    \begin{enumerate}
      \item The students are protected and the protection goal is to keep their identity secret.
      \item Two examples of applications are:
      \item The professor used the implicit addressing by pseudonyms. Possible problems with this selection of pseudonyms are that the pseudonyms are not unique and that the students could be identified by the professor.
      \item The problem could be solved by encrypting the feedback and only allowing the professor to decrypt it. The students should also use a secure connection to send the feedback.
    \end{enumerate}
  \end{solution}
\end{exercise}

\begin{exercise}{IPv6 prefixes}
  \begin{enumerate}
    \item Why is it not sufficient to use different addresses under one prefix?
    \item What are the weaknesses of prefix sharing with a constant prefix and a small user group?
    \item What are the weaknesses of prefix bouquets if different prefixes are used for different goals but not per connection?
    \item What is the problem with prefix hopping if the IP address changes during an existing connection?
    \item How can this problem be solved with prefix hopping?
    \item How can users be tracked apart from addressing? Name 3 examples.
  \end{enumerate}

  \begin{solution}
    \begin{enumerate}
      \item It is not sufficient to use different addresses under one prefix because the addresses could be linked to the same user.
      \item The weaknesses of prefix sharing with a constant prefix and a small user group are that the users can be identified by the prefix and that the users can be tracked by the prefix.
      \item The weaknesses of prefix bouquets are that the users can be identified by the prefix and that the users can be tracked by the prefix.
      \item The problem with prefix hopping is that the connection is interrupted when the IP address changes.
      \item This problem can be solved by using a VPN connection that keeps the IP address constant.
      \item Users can be tracked apart from addressing by using cookies, browser fingerprinting, and tracking pixels.
    \end{enumerate}
  \end{solution}
\end{exercise}

\begin{exercise}{Mixes}
  \begin{enumerate}
    \item Specify the attacker model.
    \item How can a mix-net be attacked by a strong attacker (e.g. intelligence service)? Name 2 examples.
    \item A mix-net is offered by only one operator, who advertises that 20 mixes are used in a row and only 10 participants per mix are assigned for higher performance. Evaluate the security of the presented product with justification.
  \end{enumerate}

  \begin{solution}
    \begin{enumerate}
      \item The attacker model is a strong attacker (e.g. intelligence service).
      \item A mix-net can be attacked by a strong attacker by using traffic analysis and by using a Sybil attack.
      \item The security of the presented product is low because the operator can see all the traffic and the participants can be identified by the operator.
    \end{enumerate}
  \end{solution}
\end{exercise}

\begin{exercise}{DC networks}
  Given were 4 local sums (A, B, C, D), a key graph kAB, kBC, kCD, and the key kBC. We are participant B and have sent an empty message.
  \begin{enumerate}
    \item Calculate the sent message N. Sketch your calculation.
    \item Limit who can have sent the message. Sketch your calculation.
    \item What needs to happen for B to make fewer statements?
  \end{enumerate}

  \begin{solution}
    \begin{enumerate}
      \item The sent message N is the sum of the local sums A, B, C, and D. Sketch your calculation.
      \item The message can only be sent by participants A and C. Sketch your calculation.
      \item For B to make fewer statements, the key kBC must be changed.
    \end{enumerate}
  \end{solution}
\end{exercise}

\end{document}