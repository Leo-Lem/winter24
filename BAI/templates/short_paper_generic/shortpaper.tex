\documentclass[12pt]{article}
%%%%%%%%%%%%%%%%%%%%%%%%%%%%%%%%%%%%%%%%%%%%%%%%%%%%%%%%%%%%%
% Meta informations:
\newcommand{\trauthor}{Horst Hansen}
\newcommand{\trtype}{Proposal} %{Expos\'{e}} %{Review}
\newcommand{\trtitle}{Neural Networks for Artificial Agents}
\newcommand{\trdate}{01.09.2010}

%%%%%%%%%%%%%%%%%%%%%%%%%%%%%%%%%%%%%%%%%%%%%%%%%%%%%%%%%%%%%
% Languages:

% Falls die Ausarbeitung in Deutsch erfolgt:
% \usepackage[german]{babel}
% \usepackage[T1]{fontenc}
% \usepackage[latin1]{inputenc}
% \usepackage[latin9]{inputenc}	 				
% \selectlanguage{german}

% If the thesis is written in English:
\usepackage[english]{babel} 						
\selectlanguage{english}

%%%%%%%%%%%%%%%%%%%%%%%%%%%%%%%%%%%%%%%%%%%%%%%%%%%%%%%%%%%%%
% Bind packages:
\usepackage{acronym}                    % Acronyms
\usepackage{algorithmic}								% Algorithms and Pseudocode
\usepackage{algorithm}									% Algorithms and Pseudocode
\usepackage{amsfonts}                   % AMS Math Packet (Fonts)
\usepackage{amsmath}                    % AMS Math Packet
\usepackage{amssymb}                    % Additional mathematical symbols
\usepackage{amsthm}
\usepackage{booktabs}                   % Nicer tables
%\usepackage[font=small,labelfont=bf]{caption} % Numbered captions for figures
\usepackage{color}                      % Enables defining of colors via \definecolor
\definecolor{uhhRed}{RGB}{254,0,0}		  % Official Uni Hamburg Red
\definecolor{uhhGrey}{RGB}{122,122,120} % Official Uni Hamburg Grey
\usepackage{fancybox}                   % Gleichungen einrahmen
\usepackage{fancyhdr}										% Packet for nicer headers
%\usepackage{fancyheadings}             % Nicer numbering of headlines

%\usepackage[outer=3.35cm]{geometry} 	  % Type area (size, margins...) !!!Release version
%\usepackage[outer=2.5cm]{geometry} 		% Type area (size, margins...) !!!Print version
%\usepackage{geometry} 									% Type area (size, margins...) !!!Proofread version
\usepackage[outer=3.15cm]{geometry} 	  % Type area (size, margins...) !!!Draft version
\geometry{a4paper,body={5.8in,9in}}

\usepackage{graphicx}                   % Inclusion of graphics
%\usepackage{latexsym}                  % Special symbols
\usepackage{longtable}									% Allow tables over several parges
\usepackage{listings}                   % Nicer source code listings
\usepackage{multicol}										% Content of a table over several columns
\usepackage{multirow}										% Content of a table over several rows
\usepackage{rotating}										% Alows to rotate text and objects
\usepackage[hang]{subfigure}            % Allows to use multiple (partial) figures in a fig
%\usepackage[font=footnotesize,labelfont=rm]{subfig}	% Pictures in a floating environment
\usepackage{tabularx}										% Tables with fixed width but variable rows
\usepackage{url,xspace,boxedminipage}   % Accurate display of URLs

%%%%%%%%%%%%%%%%%%%%%%%%%%%%%%%%%%%%%%%%%%%%%%%%%%%%%%%%%%%%%
% Configurationen:

\hyphenation{whe-ther} 									% Manually use: "\-" in a word: Staats\-ver\-trag

%\lstloadlanguages{C}                   % Set the default language for listings
\DeclareGraphicsExtensions{.pdf,.svg,.jpg,.png,.eps} % first try pdf, then eps, png and jpg
\graphicspath{{./src/}} 								% Path to a folder where all pictures are located
\pagestyle{fancy} 											% Use nicer header and footer

% Redefine the environments for floating objects:
\setcounter{topnumber}{3}
\setcounter{bottomnumber}{2}
\setcounter{totalnumber}{4}
\renewcommand{\topfraction}{0.9} 			  %Standard: 0.7
\renewcommand{\bottomfraction}{0.5}		  %Standard: 0.3
\renewcommand{\textfraction}{0.1}		  	%Standard: 0.2
\renewcommand{\floatpagefraction}{0.8} 	%Standard: 0.5

% Tables with a nicer padding:
\renewcommand{\arraystretch}{1.2}

%%%%%%%%%%%%%%%%%%%%%%%%%%%%
% Additional 'theorem' and 'definition' blocks:
\theoremstyle{plain}
\newtheorem{theorem}{Theorem}[section]
%\newtheorem{theorem}{Satz}[section]		% Wenn in Deutsch geschrieben wird.
\newtheorem{axiom}{Axiom}[section] 	
%\newtheorem{axiom}{Fakt}[chapter]			% Wenn in Deutsch geschrieben wird.
%Usage:%\begin{axiom}[optional description]%Main part%\end{fakt}

\theoremstyle{definition}
\newtheorem{definition}{Definition}[section]

%Additional types of axioms:
\newtheorem{lemma}[axiom]{Lemma}
\newtheorem{observation}[axiom]{Observation}

%Additional types of definitions:
\theoremstyle{remark}
%\newtheorem{remark}[definition]{Bemerkung} % Wenn in Deutsch geschrieben wird.
\newtheorem{remark}[definition]{Remark} 

%%%%%%%%%%%%%%%%%%%%%%%%%%%%
% Provides TODOs within the margin:
\newcommand{\TODO}[1]{\marginpar{\emph{\small{{\bf TODO: } #1}}}}

%%%%%%%%%%%%%%%%%%%%%%%%%%%%
% Abbreviations and mathematical symbols
\newcommand{\modd}{\text{ mod }}
\newcommand{\RS}{\mathbb{R}}
\newcommand{\NS}{\mathbb{N}}
\newcommand{\ZS}{\mathbb{Z}}
\newcommand{\dnormal}{\mathit{N}}
\newcommand{\duniform}{\mathit{U}}

\newcommand{\erdos}{Erd\H{o}s}
\newcommand{\renyi}{-R\'{e}nyi}

%%%%%%%%%%%%%%%%%%%%%%%%%%%%%%%%%%%%%%%%%%%%%%%%%%%%%%%%%%%%%
% Document:
\begin{document}
\renewcommand{\headheight}{14.5pt}

\fancyhead{}
\fancyhead[CO]{\trtitle}

%%%%%%%%%%%%%%%%%%%%%%%%%%%%
% Cover Header:
\title{\trtitle\\[0.3cm]{\normalsize\trtype}}
\author{\trauthor}
\date{\trdate}
\maketitle

%%%%%%%%%%%%%%%%%%%%%%%%%%%%

\thispagestyle{empty}
\pagenumbering{arabic}

% Abstract gives a brief summary of the main points of a paper:
\begin{abstract}
  This paper gives a brief overview and an example of how to write a short paper in \LaTeX. An example for a short paper could be a proposal or a review.
\end{abstract}

% the actual content, usually separated over a number of sections
% each section is assigned a label, in order to be able to put a
% crossreference to it

\section{Introduction}
\label{sec:introduction}

The introduction describes the problem, why it is important, the main
ideas of the following paper, what are the main contributions of the
paper, etc. 

If the paper has more then 4 pages, a readers guide is recommenced. In a short paragraph an outline of the content of the next 

\section{Optional: Background Information}
\label{sec:basics}
A brief section giving background information may be necessary, especially if your work spans two or more traditional fields. That means that your readers may not have any experience with some of the material needed to follow your thesis, so you need to give it to them. A different title than that given above is usually better; e.g., "A Brief Review of Frammis Algebra."

\section{Related Work}
\label{sec:relwork}

The related work section could describe other work that is in some respects relevant for the understanding of the problem outlined in Section~\ref{sec:introduction}, that offer competing solutions, etc.

All sources must be properly referenced, ideally by using the BiBTeX system. References can then be very conveniently made with the \textsc{cite} command. For example, reference
\cite{Leunen:Scholars:92} discusses some of the elementary rules on
writing scientific papers, amongst others how to correctly cite other
documents. Reference \cite{Taylor:SIGuide:95}, e.g., describes how to
correctly use the SI system of units and their correct typographical
representation. In general you organize this section by idea, and not by author or by publication.

\section{Model description}
\label{sec:model}

After the two (or maybe three) common sections Introduction (plus optional Background) and Related work, more sections with the actual content of a paper follow. The style and
structure of such sections varies by a large degree, no general rules
of thumb can be given. 

\subsection{Word processing with \LaTeX}
\label{sec:model:subsec:latex}

This document has already introduced the most important constructs of
\LaTeX. What is necessary to produce documents with \LaTeX is simple
any normal text editor and a \LaTeX distribution. This is commonly
installed on practically all UNIX-type systems; for Windows, an
excellent \LaTeX exists, called MikTeX, available from
\url{www.miktex.org}. Almost all distributions come with a large patch
of examples and introductory material; consult your local installation
for details. 

Lots of supplementary and background information, FAQs, etc.\ is
available from the Comprehensive TeX Archive Network (CTAN); the
German mirror of which is \url{www.dante.de}. 

\subsection{Tables in \LaTeX}
\label{sec:model:subsec:tables}
Tables should be centred and should always have a caption 
positioned above it. A caption in a sentence form as well as in a short form must end with a period as seen in table~\ref{tab:sample}.

\begin{table}[hbtp]
  \caption{This caption has one line so it is centred.}\label{tab:sample} 
  \centering
  \begin{tabular}{|c|c|}
    \hline
    Example column 1 & Example column 2 \\
    \hline
    Example text 1 & Example text 2 \\
    \hline
  \end{tabular}
\end{table}


\subsection{Figures in \LaTeX}
\label{sec:model:subsec:figures}
Note that a figure is a so-called floating object: it is moved around the actual
text in order to best fit on a page. This is in stark contrast to some
GUI-based word processing tools, where the placement of figures is
usually more associated with luck than principle.

As figures float around, expressions like ``the following figure''
must never be used. Instead, figures need a caption, a label, and must
be properly referenced in the main text. A figure caption is placed centred below the figure and describes the figure in (very) short. 

In general, only vector graphics in encapsulated postscript (eps) or a similar format should be included in any kind of text, as this allows arbitrary scaling, rotation etc.\ without any loss of quality. Bitmap formats (JPEG, GIF, \dots) should only be used if no other alternative exists
--- basically the only case where bitmaps can be justified is when scanned pictures need to be included in a text, however, this should be avoided as hard as possible as the quality in usually not satisfactory. If a screen shot is needed a high resolution picture without visible fragments of a jpeg compression is allowed. Figures like the figure~\ref{fig:samplefig} should always appear after the first referencing it.
\begin{figure}[hbtp]
	 \centerline{\includegraphics[width=0.6\textwidth]{samplefig.pdf}}
	 {\caption{Network of transputers and the structure of individual
processes.}\label{fig:samplefig}}
\end{figure}

\section{Model analysis}
\label{sec:analysis}
The analysis of a model is a much more free-form of a paper. Starting with a comparison of several approaches over some experimental settings and result up to a theoretical verification of the model, anything is allowed. But it all has only one purpose: to convince the reader of the right to exist of the described model.

\section{Conclusion}
\label{sec:concl}

At the end, there is a final section concluding and summarizing a
paper, putting the entire work into perspective and explaining, on a
larger level, what the consequence of this work are. Also, unexpected
results can be discussed here, etc.


%%%%%%%%%%%%%%%%%%%%%%%%%%%%%%%%%%%%%%
% hier werden - zum Ende des Textes - die bibliographischen Referenzen
% eingebunden
%
% Insbesondere stehen die eigentlichen Informationen in der Datei
% ``bib.bib''
%
\bibliography{bib}
\bibliographystyle{plain}

\end{document}


