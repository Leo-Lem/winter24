\documentclass[12pt]{article}



%%%%%%%%%%%%%%%%%%%%%%%%%%%%%%%%%%%%%%
% Metainformation
%%%%%%%%%%%%%%%%%%%%%%%%%%%%%%%%%%%%%%
\newcommand{\trauthor}{Leopold Lemmermann}
\newcommand{\trtype}{Paper}
\newcommand{\trtitle}{Image Captioning}
\newcommand{\trdate}{\today}



%%%%%%%%%%%%%%%%%%%%%%%%%%%%%%%%%%%%%%
% Language
%%%%%%%%%%%%%%%%%%%%%%%%%%%%%%%%%%%%%%
\usepackage[english]{babel}
\selectlanguage{english}



%%%%%%%%%%%%%%%%%%%%%%%%%%%%%%%%%%%%%%
% Bind packages
%%%%%%%%%%%%%%%%%%%%%%%%%%%%%%%%%%%%%%
\usepackage{acronym}                    % Acronyms
\usepackage{algorithmic}								% Algorithms and Pseudocode
\usepackage{algorithm}									% Algorithms and Pseudocode
\usepackage{amsfonts}                   % AMS Math Packet (Fonts)
\usepackage{amsmath}                    % AMS Math Packet
\usepackage{amssymb}                    % Additional mathematical symbols
\usepackage{amsthm}
\usepackage{booktabs}                   % Nicer tables
% \usepackage[font=small,labelfont=bf]{caption} % Numbered captions for figures
\usepackage{color}                      % Enables defining of colors via \definecolor
\definecolor{uhhRed}{RGB}{254,0,0}		  % Official Uni Hamburg Red
\definecolor{uhhGrey}{RGB}{122,122,120} % Official Uni Hamburg Grey
\usepackage{fancybox}                   % Gleichungen einrahmen
\usepackage{fancyhdr}										% Packet for nicer headers

%\usepackage[outer=3.35cm]{geometry} 	  % Type area (size, margins...) !!!Release version
%\usepackage[outer=2.5cm]{geometry} 		% Type area (size, margins...) !!!Print version
%\usepackage{geometry} 									% Type area (size, margins...) !!!Proofread version
\usepackage[outer=3.15cm]{geometry} 	  % Type area (size, margins...) !!!Draft version
\geometry{a4paper,body={5.8in,9in}}

\usepackage{graphicx}                   % Inclusion of graphics
%\usepackage{latexsym}                  % Special symbols
\usepackage{longtable}									% Allow tables over several parges
\usepackage{listings}                   % Nicer source code listings
\usepackage{multicol}										% Content of a table over several columns
\usepackage{multirow}										% Content of a table over several rows
\usepackage{rotating}										% Alows to rotate text and objects
\usepackage[hang]{subfigure}            % Allows to use multiple (partial) figures in a fig
%\usepackage[font=footnotesize,labelfont=rm]{subfig}	% Pictures in a floating environment
\usepackage{tabularx}										% Tables with fixed width but variable rows
\usepackage{url,xspace,boxedminipage}   % Accurate display of URLs



%%%%%%%%%%%%%%%%%%%%%%%%%%%%%%%%%%%%%%
% Configuration
%%%%%%%%%%%%%%%%%%%%%%%%%%%%%%%%%%%%%%
\hyphenation{whe-ther} 									% Manually use: "\-" in a word: Staats\-ver\-trag

%\lstloadlanguages{C}                   % Set the default language for listings
\DeclareGraphicsExtensions{.pdf,.svg,.jpg,.png,.eps} % first try pdf, then eps, png and jpg
\graphicspath{{./src/}} 								% Path to a folder where all pictures are located
\pagestyle{fancy} 											% Use nicer header and footer

% Redefine the environments for floating objects:
\setcounter{topnumber}{3}
\setcounter{bottomnumber}{2}
\setcounter{totalnumber}{4}
\renewcommand{\topfraction}{0.9} 			  %Standard: 0.7
\renewcommand{\bottomfraction}{0.5}		  %Standard: 0.3
\renewcommand{\textfraction}{0.1}		  	%Standard: 0.2
\renewcommand{\floatpagefraction}{0.8} 	%Standard: 0.5

% Tables with a nicer padding:
\renewcommand{\arraystretch}{1.2}



%%%%%%%%%%%%%%%%%%%%%%%%%%%%%%%%%%%%%%
% Additional 'theorem' and 'definition' blocks
%%%%%%%%%%%%%%%%%%%%%%%%%%%%%%%%%%%%%%
\theoremstyle{plain}
\newtheorem{theorem}{Theorem}[section]
\newtheorem{axiom}{Axiom}[section]
%Usage:%\begin{axiom}[optional description]%Main part%\end{fakt}

\theoremstyle{definition}
\newtheorem{definition}{Definition}[section]

%Additional types of axioms:
\newtheorem{lemma}[axiom]{Lemma}
\newtheorem{observation}[axiom]{Observation}

%Additional types of definitions:
\theoremstyle{remark}
\newtheorem{remark}[definition]{Remark}



%%%%%%%%%%%%%%%%%%%%%%%%%%%%%%%%%%%%%%
% Abbreviations and mathematical symbols
%%%%%%%%%%%%%%%%%%%%%%%%%%%%%%%%%%%%%%
\newcommand{\modd}{\text{ mod }}
\newcommand{\RS}{\mathbb{R}}
\newcommand{\NS}{\mathbb{N}}
\newcommand{\ZS}{\mathbb{Z}}
\newcommand{\dnormal}{\mathit{N}}
\newcommand{\duniform}{\mathit{U}}

\newcommand{\erdos}{Erd\H{o}s}
\newcommand{\renyi}{-R\'{e}nyi}



%%%%%%%%%%%%%%%%%%%%%%%%%%%%%%%%%%%%%%
% Document:
%%%%%%%%%%%%%%%%%%%%%%%%%%%%%%%%%%%%%%
\begin{document}
\renewcommand{\headheight}{14.5pt}

\fancyhead{}
\fancyhead[CO]{\trtitle}



%%%%%%%%%%%%%%%%%%%%%%%%%%%%%%%%%%%%%%
% Cover
%%%%%%%%%%%%%%%%%%%%%%%%%%%%%%%%%%%%%%
\title{\trtitle\\[0.3cm]{\normalsize\trtype}}
\author{\trauthor}
\date{\trdate}
\maketitle

\thispagestyle{empty}

\begin{abstract}
  This paper is about image captioning. We discuss the current state of the art and give an overview of the most important methods. We also present some of our own results.
\end{abstract}

\tableofcontents
\newpage
\pagenumbering{arabic}



%%%%%%%%%%%%%%%%%%%%%%%%%%%%%%%%%%%%%%
% Content (Outline_Lemmermann_Leopold_TITLE)
%%%%%%%%%%%%%%%%%%%%%%%%%%%%%%%%%%%%%%
\section{Introduction}
\label{sec:introduction}

Image captioning is the task of generating a textual description of an image. It is a challenging problem that requires understanding of both the visual and the textual domain. In recent years, deep learning has revolutionized the field of image captioning.

In this paper, we present an overview of the most important methods in image captioning. We also present some of our own results.

Comparable, if more detailed, papers include %TODO: Add references to related work.

\section{Methods}
\label{sec:methods}

Accuracy of image captioning is commonly measured using BLEU or CIDEr scores. BLEU is a precision-based metric that compares n-grams in the generated caption to those in the reference captions. CIDEr is a recall-based metric that compares the generated caption to the reference captions using cosine similarity.

We put our main focus on the flickr8k dataset. It consists of 8,000 images, each with five reference captions. Other commonly used datasets are flickr30k and MS COCO, but due to their size and our limited resources, we decided to focus on flickr8k.

Usually, to achieve image captioning an encoder-decoder architecture is used. The encoder is a convolutional neural network (CNN) that extracts features from the image. The decoder is a recurrent neural network (RNN) that generates the caption word by word.

We experiment with different architectures, such as VGG16, ResNet, and InceptionV3 as encoders, and LSTM and GRU as decoders. We also experiment with different attention mechanisms, such as soft attention and hard attention.

\section{Approach}
\label{sec:approach}

We follow the state-of-the-art approach to image captioning. We preprocess the images using the encoder and tokenize the captions. We then train the model on the training set and evaluate it on the validation set. We use beam search to generate the captions.

\section{Results}
\label{sec:results}

Comparison of different models, evaluation, and discussion.

\section{Conclusion}
\label{sec:concl}

Summary of the results, future work, and open questions.



%%%%%%%%%%%%%%%%%%%%%%%%%%%%%%%%%%%%%%
% References
%%%%%%%%%%%%%%%%%%%%%%%%%%%%%%%%%%%%%%
\newpage
\thispagestyle{empty}

\nocite{*}
\bibliographystyle{apalike}
\bibliography{bib}

\end{document}
