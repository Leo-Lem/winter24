\documentclass{article}

\usepackage[]{../.template/xrcise}

\subject{Bio-inspired Artificial Intelligence}
\semester{Winter 2024}
\author{Leopold Lemmermann}

\begin{document}\createtitle

\section[2024]{Mock exam}

% TODO



\section[2023]{1st exam}
% First questions were multiple choice

% Function to compute neighbourhood in SOM
% Meaning of grounded language
% Ventriloquism Effect

% Followed by some general questions

% multisensory integration of different strength stimuli (slide example)
% What is the cocktailparty effect


% Spiking neural networks

% label leaky and integrate+fire models
% Similarity angle, calculate for a weight and input vector
% Compute output from perceptron (similar to slide example)
% Visual

% what do the different layers on CNN learn (not every single layer but a brief overview)
% Compute filtering in CNN layer (5x5 matrix with 3x3 filter, stride=1, no padding)
% Draw orientation and firing rate diagram from slides
% Audio

% ITD and ILD: explain and what to use for which frequencies
% Behavior

% braitenberg vehicles
% Advantage of behavior based


% Evolutionary Computing

% Cycle Graph also arrow labels
% Example: find the shortest distance between cities only visit each city once
% Recombination and Mutation algorithm for example
% Optimisation strategies for this algorithm



\section[2022]{1st exam}
% Nr.1) Multiple choice: (one or multiple correct, you only get points if all ticks are completely correct, the last possibility of four was always "none of the above")

% a)

% ...

% b) In which Models is Backpropagation used for learning?

% MLP

% Convolutional Neural Network

% SOM

% None of the above

% c) what does "embodied" means for language?

% ...

% d)

% ...

% e)

% ...



% Nr.2) Attention

% 2.1) Name and explain three ways to measure the attention of a human.

% 2.2) Describe two models to measure or visualize human attention.

% 2.3) Name advantages and disadvantages of the two models you proposed.



% Nr.3) Vision

% 3.1) Name the two streams and their task that are present in the human vision.

% 3.2) Explain rods and cones

% 3.3) Explain the similarity between a convolutional neural network with convolution and max pooling layers and the visual cortex (not sure if visual cortex or some other brain area)

% 3.4) Visualize the firing rate of a simple cell for a line of different orientation in a diagram with the orientation on the x-axis and the firing rate on the y-axis

% 3.5) Given a 5x5 matrix and a 3x3 filter compute the convolution (no padding, stride = 1)



% Nr.4) Sound Localization

% 4.1) compute crosscorrelation steps on two signals until maximum crosscorrelation value is reached

% 4.2) name and explain the two cues used to perform sound localization. Which one works best for high/low frequencies?

% ...



% Nr.5) Behaviour

% 5.1) Explain a Braitenberg Vehicle with an example and name the behaviour of your example vehicle.

% 5.2) Flow chart of functional decomposition of the classical approach for a robot controller was given with steps left blank: (sensors => blank => blank => blank => blank => actuators)

% ...



% Nr.6) Continual Learning

% 6.1) Explain the stability-plasticity dilemma

% 6.2) Explain catastrophic forgetting and a model where this effect could arise

% 6.3) Name and describe two strategies of continual learning



% Nr.7) Language processing

% 7.1) Name the two streams important in language processing and their tasks

% 7.2) Describe CBOW (Continuous Bag-of-word) and Skip-gram

% 7.3) Describe difference between the earlier Wernickes model and the later Pulvermüllers model (question gave a few more details which I don't remember anymore)





% Nr.8) Computational Neural Networks

% 8.1) Explain 3 steps of Backpropagation algorithm and explain termination criteria

% 8.2) Name and shortly describe the three fundamental ANN Learning Paradigms

% 8.3) Explain how the learning rate parameter could be chosen

\end{document}