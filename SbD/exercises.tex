\documentclass{article}

\usepackage[]{../.template/xrcise}

\subject{Security by Design}
\semester{Winter 2024}
\author{Leopold Lemmermann}

\begin{document}\createtitle

\section{Assignments}
% Aufgabe 1.1 Skytala
% Bei der Skytala handelt es sich um eine Transpositionschiffre, bei der ursprünglich ein Zylinder mit einem
% Papierstreifen umwickelt wurde. Die Ver- und Entschlüsselung kann aber auch mithilfe einer Matrix dargestellt
% werden.
% a) Verschlüsseln Sie den folgenden Klartext mit einer Skytala, die durch eine Matrix mit sieben Zeilen und
% neun Spalten beschrieben werden kann:
% G L U E C K L I C H _ I S T _ D E R _ D E R _ D E N _ D I N G E N _ A U F _ D E N _ G R U N D _ G E H E N _ K O N N T E
% b) Was ist der geheim zu haltende symmetrische Schlüssel der Skytala aus Teilaufgabe a) und wie hängt
% dieser mit der ursprünglichen Variante mit Zylinder und Papierstreifen zusammen?
% c) Erläutern Sie kurz zwei sinnvolle Ansätze zur Kryptanalyse, d. h. Ansätze zum Brechen einer Skytala.

% Aufgabe 1.2 Spalten-Transpositionen
% Bei der Skytala wird ein Zylinder mit einem Papierstreifen umwickelt, beschrieben und wieder abgewickelt.
% a) Eine Verbesserung dieser Spalten-Transposition ergibt sich, wenn zusätzlich auch die Spalten in ihrer
% Reihenfolge vertauscht werden (was beim Papierstreifen natürlich nicht gelingt, wohl aber im Rechner).
% Um welchen Faktor vergrößert sich dadurch der Schlüsselraum bei s Spalten?
% b) Schreiben Sie eine Skytala mit 4 Zeilen und 3 Spalten (ohne Spaltenvertauschung) in Zyklenschreib-
% weise.

% Aufgabe 2.1 Unsicherheit der Vigènere-Chiffre
% Ein klassisches Verschlüsselungsverfahren ist die Vigenère-Chiffre. Sie arbeitet typischerweise auf einem Al-
% phabet aus den Großbuchstaben A-Z für den Klartext, den Schlüssel und den Schlüsseltext. Zudem kann
% ein Zeichen zur Worttrennung hinzugenommen werden. Für diese Aufgabe wird dazu der Unterstrich _ ver-
% wendet. Ein Schlüsseltextzeichen ergibt sich aus dem um das jeweilige Zeichen des Schlüssels verschobene
% Klartextzeichen. Diese Verschiebung kann durch das sogenannte Vigenère-Tableau in Abbildung 1 dargestellt
% werden.
% A B C D E F G H I J K L M N O P Q R S T U V W X Y Z _
% A A B C D E F G H I J K L M N O P Q R S T U V W X Y Z _
% B B C D E F G H I J K L M N O P Q R S T U V W X Y Z _ A
% C C D E F G H I J K L M N O P Q R S T U V W X Y Z _ A B
% D D E F G H I J K L M N O P Q R S T U V W X Y Z _ A B C
% E E F G H I J K L M N O P Q R S T U V W X Y Z _ A B C D
% F F G H I J K L M N O P Q R S T U V W X Y Z _ A B C D E
% G G H I J K L M N O P Q R S T U V W X Y Z _ A B C D E F
% H H I J K L M N O P Q R S T U V W X Y Z _ A B C D E F G
% I I J K L M N O P Q R S T U V W X Y Z _ A B C D E F G H
% J J K L M N O P Q R S T U V W X Y Z _ A B C D E F G H I
% K K L M N O P Q R S T U V W X Y Z _ A B C D E F G H I J
% L L M N O P Q R S T U V W X Y Z _ A B C D E F G H I J K
% M M N O P Q R S T U V W X Y Z _ A B C D E F G H I J K L
% N N O P Q R S T U V W X Y Z _ A B C D E F G H I J K L M
% O O P Q R S T U V W X Y Z _ A B C D E F G H I J K L M N
% P P Q R S T U V W X Y Z _ A B C D E F G H I J K L M N O
% Q Q R S T U V W X Y Z _ A B C D E F G H I J K L M N O P
% R R S T U V W X Y Z _ A B C D E F G H I J K L M N O P Q
% S S T U V W X Y Z _ A B C D E F G H I J K L M N O P Q R
% T T U V W X Y Z _ A B C D E F G H I J K L M N O P Q R S
% U U V W X Y Z _ A B C D E F G H I J K L M N O P Q R S T
% V V W X Y Z _ A B C D E F G H I J K L M N O P Q R S T U
% W W X Y Z _ A B C D E F G H I J K L M N O P Q R S T U V
% X X Y Z _ A B C D E F G H I J K L M N O P Q R S T U V W
% Y Y Z _ A B C D E F G H I J K L M N O P Q R S T U V W X
% Z Z _ A B C D E F G H I J K L M N O P Q R S T U V W X Y
% _ _ A B C D E F G H I J K L M N O P Q R S T U V W X Y Z
% Abbildung 1: Vigenère-Tableau
% a) Verschlüsseln Sie den nachfolgenden Klartext mit dem Schlüssel PRUEFUNG unter Verwendung einer
% Vigenère-Chiffre:
% G L U E C K L I C H _ I S T _ D E R _ D E R _ D E N _ D I N G E N _ A U F _ D E N _ G R U N D _ G E H E N _ K O N N T E
% b) Erläutern Sie kurz zwei sinnvolle Ansätze zur Kryptanalyse, d. h. zum Brechen einer Vigenère-Chiffre.
% c) Eine Verbesserung der Vigenère-Chiffre ergibt sich durch die Verlängerung des Schlüssels. Um wel-
% chen Faktor vergrößert sich dadurch der Schlüsselraum bei einem Alphabet mit 27 Elementen (A-Z und
% Unterstrich als Worttrennung) und einer Verlängerung um s Zeichen?
% d) Wie müsste die Vigenère-Chiffre verändert werden, um informationstheoretische Sicherheit im Bezug
% auf die Vertraulichkeit zu liefern? Begründen Sie, warum diese Veränderung zielführend ist.

% Aufgabe 2.2 Zufallszahlen bei der Schlüsselgenerierung
% Warum sind bei der Schlüsselgenerierung (echte) Zufallszahlen nötig? Weshalb eignet sich die XOR-
% Verknüpfung, wenn zur Erhöhung der Sicherheit mehrere Zufallszahlen zu einer einzelnen Zufallszahl ver-
% knüpft werden, etwa bei der Schlüsselgenerierung?

% Aufgabe 2.3 Falsche Verwendung des One-Time-Pads
% Ihnen sind eine Reihe verschlüsselter deutscher Substantive in die Hände gefallen. Sie gehen davon aus,
% dass fahrlässigerweise alle Wörter mit demselben One-Time-Pad verschlüsselt wurden. Versuchen Sie, den
% Schlüssel zu ermitteln, indem Sie einen geeigneten Angriff implementieren. Die Schlüsseltexte (der 6 Sub-
% stantive) in Dezimalschreibweise lauten:
% 09 00 04 10
% 10 20 28 09
% 10 16 02 02
% 10 20 05 08
% 26 26 03 00
% 28 16 03 17
% Hinweis: Das Passwort und die Substantive sind ASCII-codiert (Alphabet = {A, B, C, . . . , Z})

% Aufgabe 2.4 Grundlagen der Kryptographie
% a) Erläutern Sie zwei Ursachen, die dafür sorgen, dass fast alle kryptographischen Algorithmen mit der
% Zeit durch andere ersetzt werden müssen.
% b) Sie möchten einen Text so verschlüsseln, dass er auch in 99 Jahren nicht ohne Kenntnis des von Ihnen
% versteckten Schlüssels entschlüsselt werden kann. Welches Verfahren wählen Sie dazu? Begründen
% Sie Ihre Entscheidung. Warum wird das Verfahren heutzutage nicht überall benutzt?
% c) In der Vorlesung wurden verschiedene klassische Chiffren behandelt, darunter Substitutionschiffren
% wie die Caesar-Chiffre (Verschiebechiffre), das Schema von Polybios, das Vigenère-Verfahren und die
% Vernman-Chiffre. Zudem wurden Transpositionschiffren wie die Skytala betrachtet.
% Ihnen liegt ein Schlüsseltext vor, der mit einer dieser Chiffren erzeugt wurde. Ihnen ist bekannt, dass es
% sich um die Verschlüsselung eines deutschen Textes handelt. Wie können Sie – ohne Rechnerunter-
% stützung – vorgehen, um einzugrenzen, welche Chiffre verwendet wurde? Erläutern Sie die jeweiligen
% Ansätze.



\section{Catalogue}

% Aufgabe 4.7 Spoofing
% a) Inwiefern kann ein DNS-Spoofing-Angriff für einen Distributed Denial of Service (DDoS)-Angriff ausgenutzt werden? Erklären Sie anschaulich.
% b) Welchen Angriff kann man in der nachfolgenden Ausgabe erkennen? Begründen Sie.
% ?(192.168.1.7) at 00:1f:88:13:fe:f1 [ether] on eth0
% ?(192.168.1.4) at 00:1f:88:13:fe:f1 [ether] on eth0
% ?(192.168.1.1) at 00:1f:88:13:fe:f1 [ether] on eth0
% ?(192.168.1.3) at 00:1f:88:13:fe:f1 [ether] on eth0
% ?(192.168.1.2) at 00:1f:88:13:fe:f1 [ether] on eth0
% c) Warum funktioniert ARP-Spoofing nur in einem lokalen Netz?
% d) Sie vermuten, Opfer eines ARP-Spoofing-Angriffs zu sein. Was können Sie an Ihrem Rechner überprüfen, um einen solchen Angriff erkennen zu können?

% Aufgabe 4.8 Kommunikationsmanipulation
% Eve und Bob sind befreundet und wohnen im selben Haushalt. Ihre Computer sind an denselben Switch
% angeschlossen und über diesen mit dem Internet verbunden. Eve hat bald Geburtstag und vermutet, dass Bob
% ein Geschenk für Sie bei einem Onlineshop bestellen wird, der keine Verbindungsverschlüsselung einsetzt.
% Einerseits ist sie sehr neugierig und möchte gern so früh wie möglich erfahren, was sie geschenkt bekommt.
% Andererseits befürchtet sie, dass Bob ihr einen Stoffteddy schenken möchte, den sie nicht haben will. Sie
% möchte daher Bobs Kommunikation abhören und, wenn nötig, seine Bestellung manipulieren können.
% a) Nennen und erläutern Sie eine Angriffsart, mit der Eve den gesamten Netzverkehr von Bob abhören
% und manipulieren kann.
% b) Wie kann Bob sich vor dem in Teilaufgabe a) beschriebenen Angriff schützen?

% Aufgabe 10.5 Indeterministische Verschlüsselung
% a) Was bedeutet indeterministische Verschlüsselung?
% b) Wie realisieren Sie die indeterministische Verschlüsselung bei einem asymmetrischen Konzelations-
% system?
% c) Was ändert sich gegenüber b) bei einem symmetrischen Konzelationssystem?
% d) Sie sollen eine indeterministische Verschlüsselung realisieren. Zur Auswahl stehen die Algorithmen
% RSA und ElGamal. Wie würden Sie bei beiden Algorithmen vorgehen?
% e) Beim Einsatz asymmetrischer Verschlüsselung sollte eine Nachricht vor der Verschlüsselung noch um
% (Pseudo-)Zufallszahlen ergänzt werden (indeterministische Verschlüsselung). Welcher Angriff kann da-
% durch verhindert werden?

% Aufgabe 10.6 Informationstheoretische Sicherheit
% Das One-Time-Pad (Vernam-Chiffre) ist ein sehr einfaches, aber informationstheoretisch sicheres symme-
% trisches Verschlüsselungsverfahren: Die Klartextbits xi werden einzeln XOR-verknüpft mit einer zufälligen
% Schlüsselfolge ki gleicher Länge, die nur ein einziges Mal verwendet werden darf (Schlüsseltext si = xi ⊕ki
% mit i= 1,2,...).
% a) Was erfährt der Angreifer, der den Schlüsseltext abfängt, über den Klartext?
% b) Wie würden Sie die Sicherheit des Verfahrens beweisen?
% c) Warum darf die Schlüsselfolge nicht mehrmals verwendet werden? Überlegen Sie sich, was passieren
% würde, wenn der Angreifer zwei Schlüsseltexte s1 und s2 (oder gar noch weitere) abfangen würde, die
% unter dem gleichen Schlüssel k= k1 = k2 verschlüsselt wurden. Annahme: Es handelt sich um sinnvolle
% Klartexte.
% Es existiert auch ein informationstheoretisch sicheres (symmetrisches) Verfahren zur Authentifikation.
% d) Welche Eigenschaft muss ein informationstheoretisch sicheres Authentifikationssystem besitzen?
% e) Entwerfen Sie ein solches System und erläutern Sie dessen Funktionsweise sowie dessen Schutz
% vor einem Angreifer und damit dessen informationstheoretische Sicherheit. Legen Sie eine bitweise
% Authentisierung (ein MAC-Bit pro Klartextbit) zugrunde.
% f) Kann es informationstheoretisch sichere asymmetrische Verfahren zur Verschlüsselung bzw. Authenti-
% fikation geben? Begründen Sie Ihre Antwort.

% Aufgabe 11.3 Unsicherheit der klassischen Verschiebechiffre
% Ein klassisches Verschlüsselungsverfahren ist die Verschiebechiffre (Vigenère-Chiffre). Sie arbeitet typischer-
% weise auf einem Alphabet aus den Großbuchstaben A-Z für den Klartext, den Schlüssel und den Schlüssel-
% text. Zudem kann ein Zeichen zur Worttrennung hinzugenommen werden. Für diese Aufgabe wird dazu der
% Unterstrich _ verwendet. Ein Schlüsseltextzeichen ergibt sich aus dem um das Schlüsselzeichen verschobene
% Klartextzeichen. Dieses sogenannte Vigenère-Tableau ist in Abbildung 1 dargestellt.
% a) Verschlüsseln Sie den nachfolgenden Klartext mit dem Schlüsselwort PRUEFUNG unter Verwendung
% einer Vigenère-Chiffre:
% GLUECKLICH_IST_DER_DER_DEN_DINGEN_AUF_DEN_GRUND_GEHEN_KONNTE
% b) Erläutern Sie kurz zwei sinnvolle Ansätze zur Kryptanalyse, d. h. zum Brechen einer Vigenère-Chiffre.
% c) Eine Verbesserung der Vigenère-Chiffre ergibt sich durch die Verlängerung des Schlüsselwortes. Um
% welchen Faktor vergrößert sich dadurch der Schlüsselraum bei einem Alphabet mit 27 Elementen (A-Z
% und Unterstrich als Worttrennung) und einer Verlängerung um sZeichen?
% d) Wie müsste die Vigenère-Chiffre verändert werden, um informationstheoretische Sicherheit im Bezug
% auf die Vertraulichkeit zu liefern? Begründen Sie, warum dieser Veränderung zielführend ist.
% 23
% A B C D E F G H I J K L M N O P Q R S T U V W X Y Z _
% A A B C D E F G H I J K L M N O P Q R S T U V W X Y Z _
% B B C D E F G H I J K L M N O P Q R S T U V W X Y Z _ A
% C C D E F G H I J K L M N O P Q R S T U V W X Y Z _ A B
% D D E F G H I J K L M N O P Q R S T U V W X Y Z _ A B C
% E E F G H I J K L M N O P Q R S T U V W X Y Z _ A B C D
% F F G H I J K L M N O P Q R S T U V W X Y Z _ A B C D E
% G G H I J K L M N O P Q R S T U V W X Y Z _ A B C D E F
% H H I J K L M N O P Q R S T U V W X Y Z _ A B C D E F G
% I I J K L M N O P Q R S T U V W X Y Z _ A B C D E F G H
% J J K L M N O P Q R S T U V W X Y Z _ A B C D E F G H I
% K K L M N O P Q R S T U V W X Y Z _ A B C D E F G H I J
% L L M N O P Q R S T U V W X Y Z _ A B C D E F G H I J K
% M M N O P Q R S T U V W X Y Z _ A B C D E F G H I J K L
% N N O P Q R S T U V W X Y Z _ A B C D E F G H I J K L M
% O O P Q R S T U V W X Y Z _ A B C D E F G H I J K L M N
% P P Q R S T U V W X Y Z _ A B C D E F G H I J K L M N O
% Q Q R S T U V W X Y Z _ A B C D E F G H I J K L M N O P
% R R S T U V W X Y Z _ A B C D E F G H I J K L M N O P Q
% S S T U V W X Y Z _ A B C D E F G H I J K L M N O P Q R
% T T U V W X Y Z _ A B C D E F G H I J K L M N O P Q R S
% U U V W X Y Z _ A B C D E F G H I J K L M N O P Q R S T
% V V W X Y Z _ A B C D E F G H I J K L M N O P Q R S T U
% W W X Y Z _ A B C D E F G H I J K L M N O P Q R S T U V
% X X Y Z _ A B C D E F G H I J K L M N O P Q R S T U V W
% Y Y Z _ A B C D E F G H I J K L M N O P Q R S T U V W X
% Z Z _ A B C D E F G H I J K L M N O P Q R S T U V W X Y
% _ _ A B C D E F G H I J K L M N O P Q R S T U V W X Y Z
% Abbildung 1: Vigenère-Tableau

% Aufgabe 11.4 Klassische Chiffren und Gütekriterien
% a) Verschlüsseln Sie den Klartext PRUEFUNGSSPASS_OHNE_KOMPROMISSE mit der Skytala und dem Schlüs-
% sel k= 4 (4 Zeilen, 8 Spalten). Verschlüsseln Sie den Klartext STUDIUMSFREUDE mit der Caesar-Chiffre
% und dem Schlüssel k= 3.
% b) Die sogenannte Häufigkeitsanalyse kann als Angriff auf klassische Chiffren verwendet werden. Dabei
% werden die Buchstabenhäufigkeiten eines Chiffretextes gezählt und mit den normalerweise vorkom-
% menden Buchstabenhäufigkeiten von Klartexten verglichen. Damit können Rückschlüsse von den Chif-
% fretextbuchstaben auf die Klartextbuchstaben gezogen werden. Für welche der beiden in Teilaufgabe a)
% angesprochenen klassischen Chiffren ist dieser Angriff besonders schwerwiegend und kann zum direk-
% ten Brechen einer Chiffre führen? Begründen Sie Ihre Antwort.
% c) Sie kommen auf die Idee, beide Chiffren miteinander zu kombinieren, um eine größere Sicherheit zu
% erreichen. Begründen Sie für die Variante, dass zuerst die Caesar-Chiffre auf den Klartext und dann
% die Skytala auf den resultierenden Caesar-Chiffretext angewandt wird, ob diese Überlegung tatsächlich
% stimmt. Ändert sich an Ihrer Überlegung etwas, wenn erst die Skytala und dann die Caesar-Chiffre
% angewandt wird?
% d) Nennen und erläutern Sie die Gütekriterien, die moderne symmetrische Chiffren erfüllen müssen, um
% vor derartig einfachen Angriffen wie der Häufigkeitsanalyse zu schützen.

% Aufgabe 12.6 Hashfunktionen und Signaturen
% a) Handelt es sich bei der Identitätsfunktion H(x) = xum eine kryptographische Hashfunktion? Begründen
% Sie.
% b) Ist die Kollisionsfreiheit eine relevante Eigenschaft kryptographischer Hashfunktionen? Begründen Sie.
% c) In der Praxis wird die digitale Signatur einer Nachricht mmeist durch Signieren des (kryptographischen)
% Hashwertes der Nachricht H(m) und nicht der Nachricht selbst gebildet: Sigk(H(m)). Warum kann ein
% Empfänger sich in diesem Fall auf die Authentizität der Nachricht verlassen? Welche Vorteile bringt
% das Signieren des Hashwertes einer Nachricht mit sich? Betrachten Sie hierfür auch den Fall, dass
% klassische RSA-Signaturen ohne Padding zum direkten Signieren von kurzen Nachrichten eingesetzt
% würden.
% 25
% d) Eine Entwicklerin möchte eine von ihr entwickelte Software zur Verfügung stellen. Um den Datenverkehr
% auf ihrem eigenen Server zu reduzieren, stellt sie das Softwarepaket über ein kommerzielles Content
% Delivery Network (CDN) bereit. Sie möchte jedoch verhindern, dass ein böswilliger CDN-Betreiber die
% von ihr bereitgestellte Software durch eine mit einer Hintertür versehene Version ersetzt. Entwerfen Sie
% ein geeignetes Verfahren unter Nutzung von kryptographischen Hashfunktionen, das es Nutzenden er-
% laubt, die Integrität des Softwarepakets zu überprüfen. Sie können annehmen, dass der CDN-Betreiber
% keinen Zugriff auf den Server der Entwicklerin besitzt.
% • Wie geht die Entwicklerin beim Veröffentlichen einer neuen Software-Version vor?
% • Wie können Nutzer die Integrität überprüfen?
% • Welche Eigenschaft einer kryptographischen Hashfunktion ist für die Sicherheit des Verfahrens
% entscheidend? Begründen Sie.

% Aufgabe 15.1 Das Feistel-Prinzip
% a) Die nebenstehende Abbildung zeigt eine Runde der Feistel-
% Verschlüsselung. Berechnen Sie zwei Runden der Verschlüsse-
% lung und Entschlüsselung mit dem Feistel-Prinzip und den fol-
% genden Werten:
% • Klartext: 0011 0100
% • Funktion F: binäres UND
% • Rundenschlüssel K1: 1110
% • Rundenschlüssel K2: 0001
% Welche Eigenschaften sollte die Einwegfunktion F aufweisen, um
% mit ihrer Hilfe eine sichere Feistel-Chiffre zu erzeugen?
% Zeigen Sie, dass auch bei bei einer konstanten Einwegfunktion F
% sowohl die Ver- als auch die Entschlüsselung funktioniert.
% d) Benötigen Sie die Umkehrfunktion F−1 von F, damit die Ent-
% schlüsselung klappt? Begründen Sie Ihre Antwort.

% Aufgabe 16.3 RSA: Eigenschaften und Sicherheit
% Für die Erzeugung eines RSA-Schlüsselpaares wurde von Alice der Modulus n= p·q= 7·17= 119 gewählt.
% a) Was muss für den öffentlichen Verschlüsselungsexponenten c, der gemeinsam mit dem Modulus n
% veröffentlicht wird, gelten, damit dieser als Teil des RSA-Schlüssels verwendet werden kann?
% 34
% b) Warum kann eine mit RSA zu verschlüsselnde Nachricht m nicht gleichzeitig einen Faktor p und q
% enthalten?
% c) Alice wählt den öffentlichen Verschlüsselungsexponenten c= 5. Berechnen Sie den zugehörigen ge-
% heimen Entschlüsselungsexponenten dmithilfe des erweiterten euklidischen Algorithmus.
% d) Bei naiver Anwendung des RSA-Verfahrens wird eine Nachricht m, für die 1 <m<ngelten muss, ohne
% Veränderung mit dem öffentlichen Verschlüsselungsexponenten cpotenziert und modulo nreduziert.
% Problematisch ist hierbei der multiplikative Homomorphismus des RSA-Verfahrens. Erläutern Sie kurz
% diese Eigenschaft anhand zweier Nachrichten m1 und m2.
% Hinweis: Zeichnen Sie ggf. ein passendes Diagramm. Sie müssen keine Berechnung durchführen!
% e) Wie könnte die Angreiferin Eve die Eigenschaft des multiplikativen Homomorphismus ausnutzen, um an
% die Klartextnachricht einer abgefangenen verschlüsselten Nachricht s, die an Alice adressiert war, zu
% gelangen? Zeigen Sie das Vorgehen anhand der abgefangenen verschlüsselten Nachricht s= 101, der
% eigens von Eve gewählten Zufallszahl r= 22 sowie dem multiplikativen Inversen der Zufallszahl r−1 =
% 92 mod 119. Verwenden Sie auf Empfängerseite (Alice) den gegebenen RSA-Modulus, den öffentlichen
% Verschlüsselungsexponenten c= 11 und den geheimen Entschlüsselungsexponenten d= 35, der nur
% Alice, aber nicht Eve bekannt ist.
% f) Nennen Sie ein Anwendungsbeispiel, bei dem der multiplikative Homomorphismus des RSA-Verfahrens
% „ausgenutzt“ bzw. gewollt angewendet wird.

% Aufgabe 16.10 Würfeln mit RSA
% Bob möchte Alice über das Internet das Ergebnis eines Würfelwurfs senden. Bob verschlüsselt das Ergebnis
% vor der Übertragung mit dem deterministischen RSA-Verfahren. Alice hat bereits ein RSA-Schlüsselpaar er-
% zeugt und dabei die folgenden Parameter verwendet: pA = 5, qA = 11, eA = 3, dA = 27. Bob hat ebenfalls
% ein RSA-Schlüsselpaar erzeugt und dabei folgende Parameter verwendet: pB = 17, qB = 5, eB = 3, dB = 43.
% Die öffentlichen Schlüssel haben die beiden bereits über einen sicheren Kanal ausgetauscht.
% a) Bob hat gewürfelt und möchte das Ergebnis nun an Alice senden. Er überträgt dazu cB = 9. Zeigen Sie,
% dass eine passive Angreiferin (Eve), die lediglich die öffentlichen Schlüssel und cB kennt, mittels eines
% Chosen-Plaintext-Angriffs Bobs Wurfergebnis mB ermitteln kann. Welche Zahl hat Bob gewürfelt?
% b) Wie kann sich Bob vor diesem Angriff schützen?

% Aufgabe 16.11 Forward Secrecy mit RSA
% Zur Erreichung von Forward Secrecy erzeugt Alice auf ihrem Server mit dem RSA-Verfahren einen geheimen
% Schlüssel dsowie den dazu passenden öffentlichen Schlüssel c, den sie persönlich an ihre Kunden verteilt.
% Zusätzlich erzeugt Alices Server bei jedem Verbindungsaufbau ein neues kurzlebiges (engl. ephemeral) RSA-
% Schlüsselpaar, bestehend aus einem öffentlichen Schlüssel ce und einem geheimen Schlüssel de. Im Ver-
% gleich zum bisherigen Verfahren verteilt Alice ce nicht mehr persönlich. Stattdessen signiert ihr Server mit
% ihrem ursprünglichen privaten Schlüssel dden neuen öffentlichen Schlüssel ce und schickt diesen zusam-
% men mit der Signatur über die aufgebaute Verbindung an den Kunden. Der Kunde prüft anschließend die
% Signatur mithilfe des ursprünglichen von Alice erhaltenen öffentlichen Schlüssels cund verwendet bei korrek-
% ter Signatur ce zum Verschlüsseln des zufällig generierten symmetrischen Schlüssels k. Das Verschlüsseln
% der Nachrichten mit kerfolgt mit AES. Nach Verbindungsende löscht Alices Server de, ce und k.
% a) Welche Verbindungen könnte ein sehr starker Angreifer (etwa ein Nachrichtendienst) noch entschlüs-
% seln, wenn er an dgelangt? Begründen Sie kurz ihre Antwort.
% b) Skizzieren Sie einen Angriff, der möglich wäre, wenn der Kunde die Signatur nicht prüft.
% 37
% c) Alice ist noch nicht ganz zufrieden mit dem neuen Verfahren: Es ist ihr zu langsam. Sie nimmt nun
% folgende Änderung vor: Statt jedes mal bei der Erzeugung von ce und de neue Primzahlen pund qfür
% den Modulus Ne zu generieren, verwendet sie immer die gleichen Primzahlen. Nun wählt sie lediglich
% den öffentlichen Exponenten aus ce neu und berechnet einen dazugehörigen geheimen Exponenten
% de. Warum ist dies keine gute Idee? Zeigen Sie, wie sich dieses geänderte Verfahren angreifen lässt,
% um aufgezeichnete Verbindungen zu entschlüsseln.

% Aufgabe 16.16 Elliptische Kurven: EC-ElGamal
% Das ElGamal-Kryptosystem lässt sich auch auf elliptische Kurven übertragen. Dabei werden die Rechen-
% operationen mittels Punkten auf einer elliptischen Kurve vorgenommen. Betrachten Sie für die folgenden
% Teilaufgaben die elliptische Kurve E(GF(p)) mit der Form y2 ≡x3 + x+ 6 (mod 11).
% a) Passen Sie die Operationen des ElGamal-Kryptosystems zur Schlüsselgenerierung, zur Entschlüsse-
% lung sowie zur Verschlüsselung an, sodass diese auf der Basis elliptischer Kurven operieren.
% b) Generieren Sie mithilfe Ihres Verfahrens aus Teilaufgabe a) ein geeignetes Schlüsselpaar auf der oben
% angegebenen elliptischen Kurve. Berechnen Sie dann die Verschlüsselung cder Nachricht m= (5,2).
% c) Berechnen Sie die Entschlüsselung des Schlüsseltextes caus der vorherigen Teilaufgabe b) mit dem
% dort generierten geheimen Schlüssel.
% d) Zeigen sie anschaulich, auf welcher Annahme die Sicherheit des Verfahrens beruht. Welcher Ausdruck
% müsste berechnet werden, damit Angreifende Schlüsseltexte ohne Kenntnis des geheimen Schlüssels
% leicht entschlüsseln könnten?

% Aufgabe 22.9 Verbindungsverschlüsselung in GSM
% Welche Kommunikationsabschnitte sind in GSM verschlüsselt? Welche Kommunikationsabschnitte sind un-
% verschlüsselt?

% Aufgabe 22.27 Bluetooth-Authentifikation
% Bei Bluetooth wird auf Basis der Funktion E22 die Authentifikation durchgeführt. Was sind die In- und Outputs
% der Authentifikation?

% Aufgabe 22.29 Mehrfachnutzung des IV in einem WEP-verschlüsselten WLAN
% a) Der in WEP verwendete Initialisierungsvektor (IV) ist 24 Bit groß und steht in einem Paket P am Paketan-
% fang vor dem Schlüsseltext. Wie viele Pakete können maximal erzeugt werden, wenn kein IV mehrfach
% verwendet werden soll? Wie sind die IVs von einem Teilnehmer des WLANs zu erzeugen, sodass eine
% Kollision möglichst spät auftritt? Welches Problem gibt es bei diesem Verfahren, wenn mehrere Teilneh-
% mer kommunizieren?
% b) Nehmen Sie an, Sie haben die Pakete P1 und P2 empfangen:
% P1: 00010 00111 11111 11110 11001 01001 00101 ...
% P2: 00010 00111 11111 11110 11000 10011 00010 ...
% Die ersten 10 Klartextbits des ersten Paketes P1 lauten 10010 01100. Bestimmen Sie die ersten 10
% Klartextbits des zweiten Paketes P2.



\section[2024]{Mock exam}

% TODO



\section[2023]{2nd exam}

\begin{exercise}{Betriebsarten}
  Betrachten Sie die Betriebsarten für Blockchiffren ECB, CBC und CTR.

  \begin{enumerate}
    \item Bei welchen Arten braucht man eine invertierbare Verschlüsselungsfunktion, bei welchen nicht? Begründen Sie.
    \item Bei welchen Arten lässt sich die Verschlüsselung parallelisieren?
    \item Im zweiten Block gibt es ein Bit mit einem additiven Fehler. Was kann noch entschlüsselt werden?
    \item Ein Bit fehlt im zweiten Block (nicht bekannt, wo das Bit fehlt). Was kann noch entschlüsselt werden?
    \item Ein ganzer Block fehlt. Was kann noch entschlüsselt werden?
    \item Zwei verschiedene Klartexte wurden mit CTR mit dem gleichen IV verschlüsselt. Zeigen Sie, dass ein Angreifer den zweiten Klartext entschlüsseln kann, wenn dieser den ersten Klartext und die beiden Schlüsseltexte hat.
  \end{enumerate}

  \begin{solution}
    % TODO
  \end{solution}
\end{exercise}

\begin{exercise}{Elliptische Kurven}
  \begin{enumerate}
    \item Auf welcher mathematischen Schwierigkeit basiert der ECC?
    \item Berechnen Sie die Punkte auf der folgenden elliptischen Kurve: $a = 6$, $b = 1$, $p = 11$ (allgemeine Formel war gegeben).
    \item Nennen Sie zwei Vorteile, die ECC gegenüber sicherheitsmäßig vergleichbaren Verfahren hat.
  \end{enumerate}

  \begin{solution}
    % TODO
  \end{solution}
\end{exercise}

\begin{exercise}{Zertifikate}
  Das Zertifigierungsmodell ist transitiv anzunehmen.

  \begin{enumerate}
    \item Erklären und grenzen Sie die Begriffe Signatur und Zertifikat voneinander ab.
    \item N7 möchte N3 eine Nachricht schicken. Welchen CAs muss N7 vertrauen?
    \item N7 möchte zusätzlich N5 eine Nachricht schicken. Welchen CAs muss N7 vertrauen, so dass die Anzahl an CAs minimal ist?
    \item CA2 fällt aus. Wie können wir die Kommunikation aus b) trotzdem realisieren (Was \& Wie)? Was können wir für N1 tun?
    \item Trust on First Use: Keine CAs, Schlüssel wird bei erster Kommunikation an Kommunikationspartner geschickt, sobald und bei späteren Kommunikationen überprüft. Wenn Kommunikation und es gibt schon anderen Schlüssel, deny. Nennen und erklären Sie zwei Vorteile und zwei Nachteile.
  \end{enumerate}

  \begin{solution}
    % TODO
  \end{solution}
\end{exercise}

\begin{exercise}{GSM}
  Zur Erreichung von Forward Secrecy erzeugt Alice auf ihrem Server mit dem RSA-Verfahren einen privaten Schlüssel $d$ sowie den dazu passenden öffentlichen Schlüssel $c$, den sie persönlich an ihre Nutzerinnen und Nutzer verteilt. Zusätzlich erzeugt der Server von Alice bei jedem Verbindungsaufbau ein neues kurzlebiges (engl. ephemeral) RSA-Schlüsselpaar, das aus einem öffentlichen Schlüssel $c_e$ und einem geheimen Schlüssel $d_e$ besteht. Der Server signiert mit dem privaten Schlüssel $d$ den neuen öffentlichen Schlüssel $c_e$ und sendet diesen zusammen mit der Signatur über die aufgebaute Verbindung an die Kundin. Die Kundin prüft anschließend die Signatur mithilfe des zuvor von Alice erhaltenen öffentlichen Schlüssels $c$. Ist die Signatur korrekt, nutzt die Kundin $c_e$, um einen neuen, zufällig generierten symmetrischen AES-Schlüssel $k$ zu verschlüsseln. Die weitere Kommunikation in dieser Verbindung wird symmetrisch mit $k$ verschlüsselt. Nach Verbindungsende löscht Alices Server $d_e$, $c_e$ und $k$.

  \begin{enumerate}
    \item Welche Verbindungen könnten von einem sehr starken Angreifer wie einem Nachrichtendienst noch entschlüsselt werden, wenn dieser Zugriff auf den Server bekommt? Begründen Sie Ihre Antwort.
    \item Skizzieren Sie einen Angriff, der möglich wäre, wenn die Kundin die Signatur nicht prüft. Erläutern Sie die einzelnen nötigen Schritte.
    \item Alice ist das Verfahren zu langsam. Sie nimmt nun folgende Änderung vor: Statt jedes Mal bei der Erzeugung von $c_e$ und $d_e$ neue Primzahlen $p$ und $q$ für den Modulus $N_e$ zu generieren, verwendet sie immer die gleichen Primzahlen. Nun wählt sie lediglich den öffentlichen Exponenten $c_e$ neu und berechnet einen dazugehörigen geheimen Exponenten $d_e$. Warum ist dies keine gute Idee? Zeigen Sie, wie sich dieses geänderte Verfahren angreifen lässt, um aufgezeichnete Verbindungen zu entschlüsseln.
  \end{enumerate}

  \begin{solution}
    % TODO
  \end{solution}
\end{exercise}



\section[2022]{1st exam}

\begin{exercise}{Kryptographie}
 \begin{enumerate}
  \item Erläutern Sie zwei Ursachen, die dafür sorgen, dass fast alle kryptographischen Algorithmen mit der Zeit nicht mehr Standard sind.
  \item Sie möchten einen Text so verschlüsseln, dass er auch in 98 Jahren nicht ohne Kenntnis des von Ihnen versteckten Schlüssels entschlüsselt werden kann. Welches Verfahren wählen Sie dazu? Begründen Sie Ihre Entscheidung. Warum wird das Verfahren heutzutage nicht überall benutzt?
  \item Gegeben eine Chiffre, welche Buchstaben A-N mit Buchstaben N-Z vertauscht. Und gleichzeitig muss die Länge der Nachricht immer gleich sein. Bewerten Sie die Sicherheit der Chiffre und die beiden Eigenschaften (Polyalphabetische Verschlüsselung nur mit Vertauschen und die gleiche Länge der Nachricht).
 \end{enumerate}

  \begin{solution}
    % TODO
  \end{solution}
\end{exercise}

\begin{exercise}{Elliptische Kurven}
  \begin{enumerate}
    \item Berechnen Sie für die folgenden Parameter alle Punkte auf der Kurve: $a = 2$, $b = 3$, $p = 7$.
    \item Elgamal mit elliptischen Kurven wurde vorgestellt. Zeigen Sie, warum dieses Verfahren mit beliebigen Parametern funktioniert.
    \item Vergleichen Sie den Elgamal aus der Vorlesung mit dem gegebenen. Auf welchem Prinzip beruht die Sicherheit? Denken Sie, dass er sicherer ist? Warum?
  \end{enumerate}

  \begin{solution}
    % TODO
  \end{solution}
\end{exercise}

\begin{exercise}{Zertifikate}
  \begin{enumerate}
    \item Es wurden 3 verschiedene Möglichkeiten gegeben, einen Schlüssel von verschiedenen Schlüsselpartnern (S, V, C, T) zu kombinieren. Alle Teilschlüssel werden aneinander gehängt l/4. 3 der Teilschlüssel wird aneinander gehängt und der letzte Teilnehmer bestimmt die Reihenfolge. Alle Schlüssel werden XOR berechnet. Bewerten Sie die Sicherheit der Verfahren.
    \item Gegeben ist ein Zertifikatsbaum mit 5 Teilnehmern. $E \rightarrow D$, $E \rightarrow B$, $E \rightarrow A$, $E \rightarrow C$, $C \rightarrow B$, $A \rightarrow B$, $G \rightarrow \emptyset$. Betrachten Sie die Zertifizierungsrelation. $G$ möchte jetzt zusätzlich eine verschlüsselte Nachricht an $B$ senden. Welcher Zertifizierungsstelle bzw. welchen Zertifizierungsstellen muss $G$ mindestens vertrauen, um das Zertifikat von $B$ prüfen zu können?
    \item Wählen Sie Ihre Lösung so, dass die Anzahl an Zertifizierungsstellen, denen $N4$ (explizit) vertrauen muss, minimal ist.
    \item In der Zertifizierungsrelation aus Teilaufgabe c) wird die Zertifizierungsstelle $C$ kompromittiert und fällt deshalb aus. $G$ möchte eine verschlüsselte Nachricht an $C$ senden. Welche Maßnahme ermöglicht es $G$, das Zertifikat von $B$ trotz des Ausfalls von $CA4$ zu überprüfen?
  \end{enumerate}

  \begin{solution}
    % TODO
  \end{solution}
\end{exercise}

\begin{exercise}{GSM}
  \begin{enumerate}
    \item Sagen Sie zu folgenden Abkürzungen welche Funktion sie im GSM erfüllen: $Ki$, $IMSI$, $RAND$, $SRES$, $LAI$, $MSISDN$, $Kc$.
    \item Welche dieser Abkürzungen ist auch noch im UMTS enthalten bzw. mit ähnlicher Funktion aber anderen Namen?
  \end{enumerate}

  \begin{solution}
    % TODO
  \end{solution}
\end{exercise}
\end{document}