\documentclass{article}

\usepackage{../.template/summary}

\subject{Security by Design}
\semester{Winter 2024}
\author{Leopold Lemmermann}

\usepackage{pdfpages}

\begin{document}\createtitle

\section{Kryptographie}

\subsection{Angriffsarten (schwer zu leicht)}
\begin{enumerate}
  \item \textbf{Ciphertext Only}: Angreifer kennt nur den Ciphertext
  \item \textbf{Known Plaintext}: Angreifer kennt Ciphertext und Plaintext
  \item \textbf{Chosen Plaintext} (adaptiv v. nicht-adaptiv): Angreifer kann Plaintext wählen und erhält den Ciphertext (nicht-adaptiv: alle Plaintexte müssen vorher gewählt werden)
\end{enumerate}

\subsection{Brechen (stark zu schwach)}
\begin{enumerate}
  \item \textbf{Vollständiges Brechen}: Schlüssel gefunden
  \item \textbf{Universelles Brechen}: Äquivalenter Schlüssel gefunden
  \item \textbf{Selektives Brechen}: Bestimmte Nachricht entschlüsselt
  \item \textbf{Existenzielles Brechen}: Irgendeine Nachricht entschlüsselt
\end{enumerate}

\includepdf[pages={14-16}, nup=1x3]{slides.pdf}

\subsection{Klassische Chiffren}
\begin{itemize}
  \item \textbf{Transposition}: Veränderung der Reihenfolge
  \item \textbf{Subsitution}: Ersetzen von Zeichen
  \item \textbf{Produkt}: Kombination von Transposition und Substitution
\end{itemize}

\begin{enumerate}
  \item \textbf{Skytala}: Transposition mit Zylinderdurchmesser als Schlüssel
  \item \textbf{Polybios}: Substitution mit 5x5-Quadrat
  \item \textbf{Caesar}: Substitution mit Verschiebung
  \item \textbf{Vigenére}: Substitution mit Schlüsselwort
  \item \textbf{Beaufort}: involutorischer Vigenére (gleiche Verschlüsselung und Entschlüsselung)
  \item \textbf{Vernam} (one-time pad): Substitution mit zufälligem Schlüssel mit Klartextlänge
\end{enumerate}
\includepdf[pages={18-32,35,37-41}, nup=3x7]{slides.pdf}

\subsection{Moderne Kryptographie}



\end{document}