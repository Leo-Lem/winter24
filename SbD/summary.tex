\documentclass{article}

\usepackage{../.template/summary}

\subject{Security by Design}
\semester{Winter 2024}
\author{Leopold Lemmermann}

\usepackage{pdfpages}

\begin{document}\createtitle

\part{Kryptographie}


\includepdf[pages={14-16}, nup=1x3]{slides.pdf}
\section{Grundlagen}
\subsection{Angriffsarten (schwer zu leicht)}
\begin{enumerate}
  \item \textbf{Ciphertext Only}: Angreifer kennt nur den Ciphertext
  \item \textbf{Known Plaintext}: Angreifer kennt Ciphertext und Plaintext
  \item \textbf{Chosen Plaintext} (adaptiv v. nicht-adaptiv): Angreifer kann Plaintext wählen und erhält den Ciphertext (nicht-adaptiv: alle Plaintexte müssen vorher gewählt werden)
\end{enumerate}

\subsection{Brechen (stark zu schwach)}
\begin{enumerate}
  \item \textbf{Vollständiges Brechen}: Schlüssel gefunden
  \item \textbf{Universelles Brechen}: Äquivalenter Schlüssel gefunden
  \item \textbf{Selektives Brechen}: Bestimmte Nachricht entschlüsselt
  \item \textbf{Existenzielles Brechen}: Irgendeine Nachricht entschlüsselt
\end{enumerate}



\includepdf[pages={18-32,35,37-41}, nup=3x7]{slides.pdf}
\section{Klassische Chiffren}
\begin{itemize}
  \item \textbf{Transposition}: Veränderung der Reihenfolge
  \item \textbf{Subsitution}: Ersetzen von Zeichen
  \item \textbf{Produkt}: Kombination von Transposition und Substitution
\end{itemize}

\begin{enumerate}
  \item \textbf{Skytala}: Transposition mit Zylinderdurchmesser als Schlüssel
  \item \textbf{Polybios}: Substitution mit 5x5-Quadrat
  \item \textbf{Caesar}: Substitution mit Verschiebung
  \item \textbf{Vigenére}: Substitution mit Schlüsselwort
  \item \textbf{Beaufort}: involutorischer Vigenére (gleiche Verschlüsselung und Entschlüsselung)
  \item \textbf{Vernam} (one-time pad): Substitution mit zufälligem Schlüssel mit Klartextlänge
\end{enumerate}


\section{Moderne Kryptographie}
\includepdf[pages={44-108}, nup=4x9]{slides.pdf}
\subsection{Symmetrische Systeme}
\begin{enumerate}
  \item \textbf{One-Time-Pad (mod 2)}: Vernam-Chiffre mit zufälligem Schlüssel
  \item \textbf{Symmetrische Authentifikationscodes}: Erweiterung des One-Time-Pad mit Message Authentication Code (MAC)
  \item \textbf{DES (Data Encryption Standard)}: 56-Bit-Blockchiffre nach Feistel-Prinzip
  \item \textbf{IDEA (International Data Encryption Algorithm)}: 128-Bit-Blockchiffre mit XOR, Addition und Multiplikation
  \item \textbf{AES (Advanced Encryption Standard)}: 128-Bit-Blockchiffre mit 10, 12 oder 14 Runden
\end{enumerate}

\subsubsection{Gütekriterien}
\begin{itemize}
  \item \textbf{Vollständigkeit}: Jedes Output-Bit hängt von jedem Input-Bit ab
  \item \textbf{Avalanche}: Jede Änderung im Input führt zu vielen Änderungen im Output
  \item \textbf{Nichtlinearität}: Output ist nicht linear abhängig vom Input
  \item \textbf{Korrelationsimmunität}: Keine Korrelation zwischen Input und Output
  \item sekundäre Kriterien: Implementierbarkeit, Längentreue, Schnelligkeit
\end{itemize}

\subsubsection{Feistel-Prinzip}
\begin{itemize}
  \item \textbf{In Hälften teilen}: Klartext wird in zwei Hälften geteilt
  \item \textbf{Runden}: Rundenfunktion wird auf eine Hälfte angewendet und das Ergebnis mit der anderen Hälfte XOR-verknüpft
  \item \textbf{Austauschen der Hälften}: Die Hälften werden getauscht
  \item \textbf{Dekodierung}: Entschlüsselung ist identisch zur Verschlüsselung (Runden umgekehrt)
\end{itemize}

\subsubsection{DES (Data Encryption Standard)}
\begin{itemize}
  \item[+] vollständig
  \item[+] kein analytischer Zusammenhang zwischen Input und Output
  \item[+] invariant ggü. Komplementbildung
  \item[-] Designkriterien nicht öffentlich (inzwischen bekannt)
  \item[-] ineffiziente Implementierung wg. Permutationen
  \item[-] Schlüssellänge zu kurz (56 Bit)
  \item \textbf{möglicher Angriff}: Known-Plaintext-Angriff mit $2^{57}$ Aufwand
\end{itemize}

\subsubsection{IDEA (International Data Encryption Algorithm)}
\begin{itemize}
  \item \textbf{Schlüssellänge}: 128-Bit-Blockchiffre, entwickelt als sicherer Ersatz für DES.
  \item \textbf{Struktur}: Verwendet eine Kombination aus XOR-, Addition- und Multiplikationsoperationen für hohe Sicherheit.
  \item \textbf{Sicherheitsvorteile}: Resistenz gegen lineare und differentielle Kryptoanalyse.
  \item \textbf{Verwendung}: Besonders populär in Europa und wurde in PGP (Pretty Good Privacy) integriert.
\end{itemize}

\subsubsection{AES (Advanced Encryption Standard)}
\begin{itemize}
  \item \textbf{Schlüssellänge}: Unterstützt 128, 192 und 256 Bit, damit für verschiedene Sicherheitsniveaus geeignet.
  \item \textbf{Struktur}: Verwendet 10, 12 oder 14 Runden von Byte-Substitution, Zeilenverschiebung, Spaltenmischung und Rundenschlüssel-Addition.
  \item \textbf{Sicherheitsniveau}: Sicher gegen Brute-Force- und bekannte Kryptoangriffe, empfohlen von NIST als sicherer Standard.
  \item \textbf{Verwendung}: Weltweiter Standard für Verschlüsselung in Regierung, Industrie und dem Finanzsektor.
\end{itemize}

\subsubsection{Betriebsarten symmetrischer Blockchiffren}
\begin{enumerate}
  \item \textbf{ECB (Electronic Codebook Mode)}: Klartext wird in Blöcke geteilt und einzeln verschlüsselt
  \item \textbf{CBC (Cipher Block Chaining Mode)}: Klartext wird in Blöcke geteilt und XOR-verknüpft
  \item \textbf{CTR (Counter Mode)}: Klartext wird in Blöcke geteilt und mit Zähler verschlüsselt
  \item \textbf{OFB (Output Feedback Mode)}: Klartext wird in Blöcke geteilt und mit Zähler verschlüsselt
  \item \textbf{CFB (Cipher Feedback Mode)}: Klartext wird in Blöcke geteilt und mit Zähler verschlüsselt
\end{enumerate}



\includepdf[pages={44, 109-142}, nup=4x9]{slides.pdf}
\subsection{Asymmetrische Systeme}
\begin{enumerate}
  \item \textbf{Diffie-Hellman-Key-Exchange}: Schlüsselaustausch über unsicheren Kanal; Sicherheit durch diskretes Logarithmusproblem, anfällig für Man-in-the-Middle-Angriffe.
  \item \textbf{ElGamal Kryptosystem}: Asymmetrische Verschlüsselung und Signatur; Diffie-Hellman-basiert, stärkere Sicherheit, aber höhere Schlüsselanforderungen.
  \item \textbf{RSA zur Konzelation und Signatur}: Asymmetrische Verschlüsselung auf Faktorisierungsannahme; verbreitet in SSL/TLS, bietet Vertraulichkeit und Authentizität.
  \item \textbf{Blinde Signaturen mit RSA}: Signatur ohne Kenntnis des Inhalts für Anonymität; Anwendung in elektronischen Abstimmungen und digitalen Währungen.
  \item \textbf{Kryptosysteme auf Basis elliptischer Kurven}: Hohe Sicherheit bei kleinen Schlüsseln; effizient und ideal für mobile/räumlich beschränkte Geräte.
\end{enumerate}

\subsubsection{Mathematische Grundlagen}
\begin{itemize}
  \item \textbf{Modulo}: $a \mod b = a - b \cdot \lfloor \frac{a}{b} \rfloor$
  \item \textbf{Erweiterter Euklidischer Algorithmus}: $ggT(a, b) = a \cdot x + b \cdot y$
  \item \textbf{Euler-Phi-Funktion}: $\varphi(n) = |\{a \in \mathbb{N} \mid 1 \leq a < n, ggT(a, n) = 1\}|$
  \item \textbf{Primitive Wurzel}: $g$ ist primitive Wurzel modulo $p$, wenn $g^i \mod p \neq 1$ für $i < p-1$
  \item \textbf{Diskreter Logarithmus}: $a = g^x \mod p \Rightarrow x = \log_g a \mod p$
  \item \textbf{Faktorisierungsannahme}: $n = p \cdot q \Rightarrow p, q \in \mathbb{P}$
\end{itemize}

\subsubsection{Diffie-Hellman-Key-Exchange}
\begin{itemize}
  \item \textbf{Ziel}: Ermöglicht sicheren Schlüsselaustausch zwischen zwei Parteien über unsichere Kanäle ohne vorherigen geheimen Austausch.
  \item \textbf{Funktionsweise}: Berechnung gemeinsamer geheimer Schlüssel über Modulorechnung basierend auf einem öffentlichen Basiswert und Primzahl.
  \item \textbf{Sicherheitsprinzip}: Sicherheit basiert auf der Schwierigkeit des diskreten Logarithmusproblems.
  \item \textbf{Schwachstelle}: Anfällig für Man-in-the-Middle-Angriffe, wenn nicht authentifiziert.
\end{itemize}

\subsubsection{ElGamal Kryptosystem}
\begin{itemize}
  \item \textbf{Verwendung}: Für Verschlüsselung und digitale Signaturen geeignet.
  \item \textbf{Schlüssellänge}: Bietet längere Schlüssel für höhere Sicherheit, aber weniger Effizienz im Vergleich zu RSA.
  \item \textbf{Sicherheitsprinzip}: Beruht auf der Schwierigkeit des diskreten Logarithmusproblems.
  \item \textbf{Vorteile}: Flexibel und erweiterbar auf elliptische Kurven für mehr Effizienz.
\end{itemize}

\subsubsection{RSA zur Konzelation und Signatur}
\begin{itemize}
  \item \textbf{Verwendung}: Verschlüsselung und digitale Signatur mit öffentlichen und privaten Schlüsseln.
  \item \textbf{Schlüssellänge}: Variable Schlüssellänge (typisch 1024, 2048 oder 4096 Bit), längere Schlüssel bieten höhere Sicherheit.
  \item \textbf{Grundlage}: Faktorisierungsannahme, d.h., die Schwierigkeit der Primfaktorzerlegung großer Zahlen.
  \item \textbf{Anwendung}: Weltweit in SSL/TLS und anderen Sicherheitssystemen implementiert.
\end{itemize}

\subsubsection{Blinde Signaturen mit RSA}
\begin{itemize}
  \item \textbf{Prinzip}: Der Signierer erstellt eine Signatur ohne den Inhalt der Nachricht zu kennen (Anonymität).
  \item \textbf{Verfahren}: Der Verfasser maskiert die Nachricht, bevor sie zur Signatur übergeben wird, und entfernt die Maske nach der Signierung.
  \item \textbf{Anwendung}: Nützlich in Szenarien wie elektronischen Abstimmungen, bei denen die Vertraulichkeit wichtig ist.
\end{itemize}

\subsubsection{Kryptosysteme auf Basis elliptischer Kurven}
\begin{itemize}
  \item \textbf{Effizienz}: Höhere Sicherheit bei kleineren Schlüssellängen (z.B. ein 256-Bit-Schlüssel bietet Sicherheit ähnlich einem 3072-Bit-RSA-Schlüssel).
  \item \textbf{Grundlage}: Elliptische Kurven über endlichen Feldern, basierend auf dem diskreten Logarithmusproblem.
  \item \textbf{Verwendung}: Ideal für mobile und ressourcenbeschränkte Geräte, oft verwendet in modernen Sicherheitsprotokollen (z.B. ECDSA für digitale Signaturen).
\end{itemize}



\part{Public Key Infrastructure}
\includepdf[pages={1}, nup=3x7]{slides.pdf}




\part{Attacks and Defenses}
\includepdf[pages={1}, nup=3x7]{slides.pdf}




\part{Mobile Security}
\includepdf[pages={1}, nup=3x7]{slides.pdf}


\end{document}